\upaper{103}{Реальность религиозного опыта}
\author{Мелхиседек}
\vs p103 0:1 Все истинно религиозные реакции человека поддерживаются ранним служением духа\hyp{}помощника почитания и контролируются духом\hyp{}помощником мудрости. Первое сверхразумное дарование человека есть вовлечение личности в контур Святого Духа Вселенского Творческого Духа; задолго до пришествий божественных Сынов или всеобщего дарования Настройщиков это влияние действует, чтобы расширить взгляды человека на этику, религию и духовность. После пришествий Райских Сынов освобожденный Дух Истины вносит большой вклад в увеличение человеческой способности воспринимать религиозные истины. По мере продолжения эволюции в обитаемом мире Настройщики Мысли принимают все большее участие в развитии высших типов человеческого религиозного понимания. Настройщик Мысли --- это космическое окно, сквозь которое творение может благодаря вере мельком увидеть достоверные и божественные качества безграничного Божества, Отца Всего Сущего.
\vs p103 0:2 Стремление человеческих рас к религиям является врожденным; они проявляются повсеместно и, по\hyp{}видимому, имеют естественное происхождение; примитивные религии в своем генезисе всегда эволюционны. По мере продолжения развития естественного религиозного опыта периодические откровения истины прерывают ход планетарной эволюции, который в противном случае был бы медленным и плавным.
\vs p103 0:3 \pc Сегодня на Урантии существуют четыре вида религии:
\vs p103 0:4 \ublistelem{1.}\bibnobreakspace Естественная, или эволюционная религия.
\vs p103 0:5 \ublistelem{2.}\bibnobreakspace Сверхъестественная религия, или религия, данная откровением.
\vs p103 0:6 \ublistelem{3.}\bibnobreakspace Практическая, или современная религия, различные степени смешения религий естественных и сверхъестественных.
\vs p103 0:7 \ublistelem{4.}\bibnobreakspace Философские религии, созданные человеком или философски продуманные теологические доктрины и религии, сотворенные рассуждением.
\usection{1. Философия религии}
\vs p103 1:1 Единство религиозного опыта в среде социальной или расовой группы происходит от идентичной природы частицы Бога, пребывающей в индивидууме. Именно это божественное в человеке и порождает его бескорыстную заинтересованность в благополучии других людей. Однако поскольку личность уникальна --- не существует и двух одинаковых смертных, --- из этого неизбежно следует, что нет и двух человек, способных дать одинаковое толкование призывов и побуждений божественного духа, который живет в их разуме. Группа смертных может испытывать духовное единство, но философского однообразия не может достичь никогда. Причем об этом разнообразии толкований религиозной мысли и опыта свидетельствует тот факт, что теологи и философы двадцатого века сформулировали более пятисот различных определений религии. В действительности каждый человек определяет религию в терминах своего собственного, основанного на опыте толкования божественных побуждений, исходящих от божественного духа, который в нем пребывает, а потому такое толкование должно быть уникальным и полностью отличным от религиозной философии всех остальных людей.
\vs p103 1:2 Когда один смертный полностью согласен с религиозной философией своего собрата\hyp{}смертного, подобное явление указывает на то, что эти два существа имели сходный \bibemph{религиозный опыт} в отношении предметов, связанных со сходством их религиозного толкования.
\vs p103 1:3 Хотя ваша религия и является вопросом личного опыта, чрезвычайно важно, чтобы вам было дано узнать множество других религиозных опытов (различные толкования других и различных смертных), с тем, чтобы вы не дали вашей религиозной жизни стать эгоцентричной --- ограниченной, эгоистической и антиобщественной.
\vs p103 1:4 Рационализм неправ, когда предполагает, что религия --- это сначала примитивное верование во что\hyp{}то, за которым следует поиск ценностей. Религия --- это в первую очередь поиск ценностей; система же истолковывающих верований формируется позднее. Людям гораздо проще договориться о религиозных ценностях --- целях, --- чем о верованиях --- толкованиях. Это и объясняет то, как религия может достичь согласия о ценностях и целях, обнаруживая при этом такое вводящее в заблуждение явление, как сохранение веры в сотне противоречивых верований --- символов веры. Объясняет это и то, почему данный человек может сохранять свой религиозный опыт даже если он откажется от многих из своих религиозных верований или сменит их. Религия продолжает существовать вопреки революционным переменам в религиозных верованиях. Не теология создает религию, а религия создает теологическую философию.
\vs p103 1:5 То, что религиозные люди так сильно верили в то, что было ложным, отнюдь не делает религию несостоятельной, поскольку религия основана на признании ценностей и утверждается верой, основанной на личном религиозном опыте. Религия, следовательно, основана на опыте и религиозной мысли; теология же, философия религии, является честной попыткой этот опыт истолковать. Такие истолковывающие верования могут быть правильными или неправильными, либо смесью истины и ошибки.
\vs p103 1:6 Осознание признания духовных ценностей --- это опыт, выходящий за пределы способности формировать и воспринимать идеи. Ни в одном человеческом языке нет слова, которое бы можно было использовать для обозначения этого «ощущения», «чувства», «интуиции» или «опыта», которые мы решили назвать Богосознанием. Дух Бога, пребывающий в человеке, не является личностным --- Настройщик предличностен --- однако этот Наблюдатель являет собой ценность, выделяет божественный аромат, который личен в высшем и бесконечном смысле. Если бы Бог не был хотя бы личностным, то он бы не мог быть и сознающим, а если бы он не был сознающим, то не был бы и инфрачеловеческим.
\usection{2. Религия и человек}
\vs p103 2:1 Религия действует в человеческом разуме и до своего появления в человеческом сознании уже реализована в опыте. До того, как испытать \bibemph{рождение,} ребенок уже просуществовал девять месяцев. Однако «рождение» религии не внезапно; это вполне постепенное появление. Тем не менее рано или поздно наступает «день рождения». Нельзя войти в царство небесное, пока не «родишься заново» --- родишься от Духа. Многие духовные рождения сопровождаются сильным страданием духа и отмечены психологическими расстройствами, как и многие рождения в теле характеризуются «бурными родовыми муками» и другими аномалиями «родов». Другие духовные рождения представляют собой естественный и нормальный рост признания верховных ценностей с углублением духовного опыта, хотя ни одно религиозное развитие не происходит без сознательного усилия и явной, и индивидуальной решимости. Религия никогда не бывает пассивным опытом, отрицательным отношением. То, что называют «рождением религии», отнюдь не прямо связано с переживаниями так называемого обращения, которые, как правило, характеризуют религиозные эпизоды, возникающие в жизни позднее вследствие умственных конфликтов, подавления эмоций и бурных потрясений.
\vs p103 2:2 Однако тот, кто с детства воспитан родителями в сознании того, что он --- дитя любящего небесного Отца, не должен смотреть косо на своих смертных собратьев, которые смогли достичь такого сознания родства с Богом, лишь пережив психологический кризис, эмоциональный переворот.
\vs p103 2:3 Эволюционная почва в разуме человека, где прорастает семя данной откровением религии, и есть нравственная природа, которая так рано порождает общественное сознание. Первые побуждения моральной природы ребенка отнюдь не связаны с сексом, виной или гордыней, но с порывами справедливости, честности и со стремлением к доброте --- полезным служением своим собратьям. Когда же такие ранние моральные побуждения взращиваются, происходит постепенное становление религиозной жизни, сравнительно свободной от конфликтов, потрясений и кризисов.
\vs p103 2:4 Каждый человек весьма рано испытывает нечто, подобное конфликту между своекорыстием и альтруистическими порывами, и первый опыт Богосознания может быть неоднократно получен в процессе поиска сверхчеловеческой помощи в деле разрешения таких моральных конфликтов.
\vs p103 2:5 Психология ребенка по природе своей позитивна, а не негативна. Столь многие смертные негативны, потому что их так воспитали. Когда говорят, что ребенок позитивен, то речь идет о его моральных побуждениях, тех силах ума, появление которых подает сигнал о прибытии Настройщика Мысли.
\vs p103 2:6 Без неправильного обучения ум нормального ребенка при появлении религиозного сознания развивается позитивно в направлении моральной праведности и общественного служения, а не негативно --- по направлению от греха и вины. В развитии религиозного опыта может быть, а может и не быть конфликта, однако в нем постоянно присутствуют неизбежные решения, усилия и действия человеческой воли.
\vs p103 2:7 Нравственный выбор обычно сопровождается большим или меньшим нравственным конфликтом. Причем этот самый первый конфликт в уме ребенка происходит между побуждениями эгоизма и порывами альтруизма. Настройщик Мысли не игнорирует личностные ценности эгоистического побуждения, но действует, дабы некоторое предпочтение делалось в пользу альтруистического порыва как ведущего к цели человеческого счастья и к радостям царства небесного.
\vs p103 2:8 Когда, столкнувшись со стремлением быть эгоистичным, нравственное существо решает быть бескорыстным, это примитивное религиозное переживание. Ни одно животное такой выбор сделать не может; подобное решение --- и человеческое, и религиозное. Оно включает в себя факт Богосознания и являет собою порыв общественного служения, основу братства людей. Когда разум актом свободной воли выбирает правильное моральное суждение, такое решение образует религиозный опыт.
\vs p103 2:9 Однако еще до того, как ребенок достаточно разовьется, чтобы приобрести способность к моральному суждению и, следовательно, будет в состоянии выбрать альтруистическое служение, у него уже вырабатывается сильный и вполне целостный эгоистический характер. Именно эта фактическая ситуация и порождает теорию борьбы между «высшей» и «низшей» природами, между «старым человеком греха» и «новой природой» благодати. Нормальный ребенок весьма рано в своей жизни начинает узнавать, что «блаженнее давать, нежели принимать».
\vs p103 2:10 Человек склонен отождествлять побуждение служить самому себе со своим эго --- с самим собой. И, наоборот, желание быть альтруистом склонен отождествлять с некоторой силой вне самого себя --- с Богом. Причем такое суждение, действительно, правильное, ибо все подобные неэгоистические желания на самом деле обусловлены водительством пребывающего в человеке Настройщика Мысли, а этот Настройщик является частицей Бога. Побуждение духа\hyp{}Наблюдателя реализуется в человеческом сознании как стремление быть альтруистом, заботящимся о своих собратьях\hyp{}творениях. По крайней мере, таков ранний и основополагающий опыт детского разума. Когда растущему ребенку не удается объединить свою личность, альтруистическое побуждение может стать настолько чрезмерно развитым, что способно нанести серьезный ущерб благополучию собственного «я». Введенная в заблуждение совесть может стать причиной больших противоречий, беспокойства, печали и нескончаемого человеческого несчастья.
\usection{3. Религия и человечество}
\vs p103 3:1 Хотя вера в духов, в сны и разные другие предрассудки сыграли свою роль в эволюционном происхождении примитивных религий, нельзя пренебрегать и влиянием клана или духа племенной солидарности. В групповых отношениях была точно представлена общественная ситуация, которая отражала конфликт между альтруизмом и эгоизмом в нравственной природе древнего человека. Несмотря на свою веру в духов, примитивные австралийцы по\hyp{}прежнему сосредоточивают свою религию на клане. Со временем такие религиозные понятия имеют тенденцию персонализироваться --- вначале как животные, а позднее как сверхчеловек или как Бог. Даже такие низшие расы, как африканские бушмены, которые в своих верованиях даже не стоят на уровне тотемов, и те признают разницу между интересом собственного «я» и интересом группы, примитивно отличают ценности мирские от священных ценностей. Однако социальная группа источником религиозного опыта не является. Независимо от того, какое в целом влияние оказали примитивные верования на раннюю религию человека, несомненно, что истинное религиозное побуждение происходит от подлинных духовных присутствий, пробуждающих желание быть бескорыстным.
\vs p103 3:2 \pc Более поздняя религия предзнаменована в примитивной вере в природные чудеса и тайны, в неличностную ману. Однако рано или поздно развивающаяся религия требует того, чтобы индивидуум приносил какую\hyp{}нибудь личную жертву на благо своей социальной группы, делал бы что\hyp{}нибудь, дабы сделать счастливее и лучше других людей. В конечном итоге религии предназначено стать служением Богу и человеку.
\vs p103 3:3 Религии суждено изменить окружение человека, однако многое в религии, которую можно найти в среде смертных, сегодня не способно это сделать. Окружающая среда слишком часто подчиняет себе религию.
\vs p103 3:4 \pc Помните, что в религии всех эпох главное переживание --- это чувства, связанные с моральными ценностями и социальными значениями, а не мысли о теологических догмах или философских теориях. Религия развивается благоприятно по мере замены элемента магии понятиями морали.
\vs p103 3:5 Человек развивался, преодолев путь от суеверий, связанных с маной, магией, почитанием природы, боязнью духа и почитания животных до различных обрядов, посредством которых религиозная позиция индивидуума становилась коллективным образом действия клана. Затем эти церемонии сосредоточились и выкристаллизовались в племенные верования, а эти страхи и веры в итоге персонализировались в богов. Однако во всей этой религиозной эволюции моральный элемент всегда так или иначе присутствовал. Воздействие Бога на человека всегда было сильно. И эти мощные влияния (одно --- человеческое, а другое --- божественное) обеспечивали выживание религии на протяжении череды веков, и притом невзирая на то, что ей так часто угрожало быть уничтоженной тысячью разрушительных тенденций и враждебных антагонизмов.
\usection{4. Духовное общение}
\vs p103 4:1 Характерная разница между общественным событием и религиозным собранием заключается в том, что в отличие от светского религиозное пронизано атмосферой \bibemph{общения.} Таким образом, человеческое общение порождает чувство родства с божественным, а это и есть начало совместного богопочитания. Вкушение от общей трапезы было самым ранним типом социального общения, поэтому древние религии предусматривали, чтобы некоторая часть церемониальной жертвы съедалась почитающими. Даже в христианстве Вечеря Господня сохраняет этот вид общения. Атмосфера общения дает освежающий и успокаивающий период примирения в конфликте своекорыстного эго с альтруистическим побуждением духовного Наблюдателя, пребывающего в человеке. Это и есть прелюдия к истинному богопочитанию --- практике присутствия Бога, которое увенчивается братством людей.
\vs p103 4:2 Когда первобытный человек чувствовал, что его общение с Богом прервано, он, стараясь совершить искупление, восстановить доброжелательные отношения, прибегал к того или иного рода жертве. Алкание и жажда праведности ведет к открытию истины; истина же повышает идеалы, а это для отдельно взятого религиозного человека создает новые проблемы, ибо наши идеалы имеют тенденцию возрастать в геометрической прогрессии, тогда как наша способность жить в соответствии с ними возрастает лишь в арифметической прогрессии.
\vs p103 4:3 Чувство вины (не путайте с сознанием греха) происходит либо от прерванного духовного общения, либо от понижения нравственных идеалов человека. Избавление от подобного затруднения может прийти лишь через осознание того, что высшие нравственные идеалы человека не обязательно синонимичны воле Бога. Человек не может надеяться, что ему удастся жить согласно своим высшим идеалам, но он может быть верен своей цели отыскания Бога и все большего уподобления ему.
\vs p103 4:4 Иисус отверг все обряды жертвы и искупления. Объявив, что человек --- дитя Бога, он разрушил основы всей этой вымышленной вины и чувства одиночества во вселенной; в основу отношений между творением и Творцом были поставлены отношения между ребенком и родителем. Для своих смертных сыновей и дочерей Бог становится любящим Отцом. Все обряды, которые не являются законной частью таких интимных семейных отношений, упразднены навсегда.
\vs p103 4:5 Бог Отец общается со своим ребенком\hyp{}человеком не на основе его действительных добродетелей или достоинств, но на основе мотивировки ребенка --- цели и намерения создания. Эти отношения есть отношения родителя и ребенка и приводятся в действие божественной любовью.
\usection{5. Происхождение идеалов}
\vs p103 5:1 Ранний эволюционный разум дает начало чувству общественного долга и морального обязательства, происходящему, главным образом, от эмоционального страха. Более же позитивное стремление к общественному служению и альтруистический идеализм происходят от прямого побуждения божественного духа, пребывающего в человеческом разуме.
\vs p103 5:2 Эта идея\hyp{}идеал делания добра для других --- побуждение в чем\hyp{}то отказать самому себе ради блага своего ближнего --- вначале весьма ограничена. Первобытный человек считает ближними лишь тех, кто рядом с ним, тех, кто обращается с ним, как с соседом; по мере развития религиозной цивилизации понятие о ближнем человека разрастается и охватывает клан, племя, нацию. И затем Иисус расширил диапазон понятия о ближнем до включения в него всего человечества, до того, что мы должны любить даже своих врагов. Причем внутри каждого нормального человека есть нечто, говорящее ему о том, что это учение морально --- правильно. Даже те, кто на практике следует этому идеалу меньше всего, подтверждают, что теоретически он --- верен.
\vs p103 5:3 Все люди признают моральные качества этого универсального человеческого стремления быть бескорыстным и альтруистичным. Гуманист приписывает происхождение этого стремления естественному действию материального разума; человек же религиозный более прав, когда признает, что истинно бескорыстное побуждение разума смертного заключается в отклике на внутреннее духовное водительство Настройщика Мысли.
\vs p103 5:4 Однако человеческое толкование этих ранних противоречий между эго\hyp{}волей и бескорыстной волей, не всегда надежно. Только достаточно цельная личность может быть третейским судьей многообразных разногласий между стремлениями эго и пробуждающимся общественным сознанием. Собственное «я» имеет права так же, как и ближние человека. Ни то, ни другое не имеет исключительных прав на внимание и служение индивидуума. Неудача в решении этой проблемы породила самый ранний тип человеческого чувства вины.
\vs p103 5:5 Человеческое счастье достигается лишь тогда, когда эгоистическое желание собственного «я» и альтруистическое стремление высшего «я» (божественного духа) соотнесены и согласованы единой волей объединяющей и управляющей личности. Разум эволюционирующего человека постоянно сталкивается со сложной задачей решить спор между естественной экспансией эмоциональных порывов и моральным ростом бескорыстных стремлений, основанных на духовном понимании --- истинном религиозном размышлении.
\vs p103 5:6 Попытка добиться равного блага для самого себя и множества других личностей создает проблему, удовлетворительное решение которой в пространственно\hyp{}временных рамках можно найти не всегда. В условиях вечной жизни подобные антагонизмы могут быть разрешены, но в течение одной короткой человеческой жизни они неразрешимы. На такой парадокс и указывал Иисус, когда говорил: «Кто хочет жизнь свою сберечь, тот потеряет ее, а кто потеряет жизнь свою ради царства, тот обретет ее».
\vs p103 5:7 \pc Поиски идеала --- стремление уподобиться Богу --- это постоянное усилие до смерти и после нее. Жизнь после смерти, в сущности, не отличается от смертного бытия. Все, наши благие поступки в этой жизни благого, прямо способствует обогащению жизни будущей. Настоящая религия не благоприятствует моральной праздности и духовной лени, поощряя тщетную надежду обрести все замечательные добродетели, благородного характера в результате вхождения во врата естественной смерти. Истинная религия не умаляет усилий человека, направленных на совершенствование в течение отпущенной ему смертной жизни. Каждое достижение в смертной жизни, непосредственно содействует обогащению первых этапов опыта бессмертного продолжения существования.
\vs p103 5:8 \pc Когда человека учат, что все его альтруистические порывы --- всего лишь продолжение его природных стадных инстинктов, это фатально для его идеализма. Но человек облагораживается и получает мощный заряд энергии, когда узнает, что эти высшие стремления его души исходят от духовных сил, пребывающих в его смертном разуме.
\vs p103 5:9 Когда человек начинает полностью сознавать, что внутри у него живет и стремится к достижениям нечто вечное и божественное, это возвышает его над самим собой. Поэтому живая вера в сверхчеловеческое происхождение наших идеалов утверждает нашу веру в то, что мы --- сыновья Бога, и делает реальными наши альтруистические убеждения, чувства человеческого братства.
\vs p103 5:10 Человек в своей духовной сфере обладает свободной волей. Смертный человек не является ни беспомощным рабом неумолимого владычества всесильного Бога, ни жертвой безнадежной фатальности механистического космического детерминизма. Человек --- воистину архитектор своего собственного вечного предназначения.
\vs p103 5:11 \pc Однако принуждение человека не спасает и не облагораживает. Духовный рост происходит внутри совершенствующейся души. Нажим может деформировать личность, но роста не стимулирует никогда. Даже принуждение в области образования, и то полезно лишь в негативном смысле, поскольку может способствовать предотвращению гибельных ситуаций. Духовный рост максимален тогда, когда все нажимы извне минимальны. «Где дух Господень --- там свобода». Человек развивается лучше всего, когда давление со стороны семьи, общины, церкви и государства минимально. Однако это нельзя толковать в том смысле, что в прогрессивном обществе нет места семье, общественным институтам, церкви и государству.
\vs p103 5:12 Когда член общественно\hyp{}религиозной группы удовлетворяет требования такой группы, его нужно поощрять наслаждаться религиозной свободой, безвозбранно выражая свое собственное толкование истин религиозной веры и фактов религиозного опыта. Безопасность религиозной группы зависит от духовного единства, а не теологической однородности. Религиозная группа должна уметь пользоваться вольностью свободомыслия, но при этом не становиться «вольнодумцами». Существует великая надежда для любой церкви, которая почитает живого Бога, утверждает братство людей и не боится снять со своих членов давление символов веры.
\usection{6. Философское согласование}
\vs p103 6:1 Теология --- это исследование действий и реакций человеческого духа; наукой она не сможет стать никогда, поскольку в своем личном выражении должна всегда в большей или меньшей степени сочетаться с психологией, а в систематическом описании --- с философией. Теология --- это всегда исследование \bibemph{твоей} религии; исследование же религии другого --- это психология.
\vs p103 6:2 \pc Когда человек к исследованию и изучению своей вселенной подходит \bibemph{извне,} он порождает различные естественные науки; когда же к изучению самого себя и вселенной он подходит \bibemph{изнутри,} то дает начало теологии и метафизике. Более позднее искусство философии развивается в попытке согласовать множество несоответствий, которым вначале суждено появляться между открытиями и учениями этих двух диаметрально противоположных путей подхода к всеобъемлющей вселенной.
\vs p103 6:3 Религия связана с духовными воззрениями, сознанием \bibemph{внутренности} человеческого опыта. Духовная природа человека дает ему возможность «завернуть» вселенную внутрь. Поэтому несомненно, что если взглянуть на творение исключительно с точки зрения внутреннего опыта личности, то оно по своей природе кажется духовным.
\vs p103 6:4 Когда человек исследует вселенную аналитически через посредство материальных дарований своих физических чувств и связанного с ними умственного восприятия, космос кажется механическим и энергетически\hyp{}материальным. Подобный способ изучения реальности заключается в «выворачивании» вселенной изнутри наружу.
\vs p103 6:5 \pc Логическую и последовательную философскую концепцию вселенной нельзя построить на постулатах ни материализма, ни спиритизма, ибо обе эти системы мышления при универсальном их применении вынуждены рассматривать космос в искажении, поскольку первая из них соприкасается со вселенной, «вывернутой наружу», а вторая постигает природу вселенной, «завернутой внутрь». Поэтому ни наука, ни религия сами по себе в одиночку, без водительства человеческой философии и озарения божественного откровения, не могут надеяться на достижение адекватного понимания вселенских истин и отношений.
\vs p103 6:6 Внутренний дух человека для своего выражения и самореализации должен всегда полагаться на механизм и метод разума. Подобно тому, внешнее переживание человеком материальной реальности должно основываться на разумном сознании переживающей личности. Поэтому духовные и материальные, внутренние и внешние, переживания человека всегда соотнесены с функцией разума и в своей сознательной реализации деятельностью разума обусловлены. Человек постигает опытом материю в своем разуме, а духовную реальность --- в душе, но сознает этот опыт в своем разуме. Интеллект --- вот согласующий и всегда присутствующий фактор, который обусловливает и определяет всю совокупность опыта смертного. Как материальные вещи, так и духовные ценности окрашены толкованием их в рассудочной области сознания.
\vs p103 6:7 Трудность, которую испытываете вы в достижении более гармоничного согласования между наукой и религией, вызвана вашим полным незнанием промежуточной области моронтийного мира вещей и существ. Локальная вселенная состоит из трех ступеней, или этапов, проявления реальности: материи, моронтии и духа. Моронтийный подход стирает все расхождения между открытиями естественных наук и действием духа религии. Рассуждение --- это метод науки, основанный на понимании; вера --- это метод религии, основанный на проницательности; мота --- это метод моронтийного уровня. Мота --- это сверхматериальная чувствительность к реальности, которая начинает компенсировать неполный рост и в качестве своей субстанции имеет знание\hyp{}рассуждение, а в качестве сущности --- веру\hyp{}проницательность. Мота --- это сверхфилософское согласование различного восприятия реальности, для материальных личностей недостижимая; отчасти она основана на опыте продолжения существования после материальной жизни во плоти. Однако многие смертные уже признали желательность обладания некоторым методом согласования взаимодействия между далеко отстоящими друг от друга областями науки и религии, и метафизика --- это результат безуспешной попытки человека перебросить мост через эту общепризнанную пропасть. Однако оказалось, что человеческая метафизика больше запутывает, чем разъясняет. Метафизика представляет собой благонамеренную, но тщетную попытку человека возместить отсутствие моронтийной моты.
\vs p103 6:8 \pc Метафизика оказалась несостоятельной, а моту человек воспринимать не может. Откровение --- вот единственный способ, который может возместить отсутствие мотийной чувствительности к истине в материальном мире. Откровение уверенно проясняет путаницу, созданную рассуждением метафизики в эволюционном мире.
\vs p103 6:9 Наука --- это предпринятая человеком попытка исследовать свое физическое окружение, мир материи и энергии; религия --- это переживание человеком космоса духовных ценностей; философия же была создана усилием человеческого разума, дабы организовать и скоррелировать открытия этих глубоко различных понятий в нечто, подобное разумному и единому отношению к космосу. Философия, проясненная откровением, удовлетворительно действует в случае, когда отсутствует мота, а попытка подменить моту человеческим рассуждением --- метафизикой потерпела крах и неудачу.
\vs p103 6:10 \pc Древний человек не делал различия между уровнем энергии и уровнем духа. Фиолетовая раса и их потомки\hyp{}андиты были первыми, кто попытался отделить математическое от волевого. Цивилизованный человек все больше шел по следам древних греков и шумеров, которые отличали неодушевленное от одушевленного. И по мере развития цивилизации философии придется перекидывать мост через постоянно расширяющиеся пропасти между понятием духа и понятием энергии. Но во времени пространства эти расхождения соединяются в Верховном.
\vs p103 6:11 \pc Наука должна всегда основываться на рассуждении, хотя воображение и предположения полезны для расширения ее границ. Религия же всегда зависит от веры, хотя рассуждение и является стабилизирующей силой и полезной служанкой. И вводящие в заблуждение толкования явлений как естественного, так и духовного миров, как наук, так и религий (которые таковыми называются ложно), всегда были и будут всегда.
\vs p103 6:12 Из своего неполного понимания науки, своего слабого овладения религией и своих бесплодных попыток в метафизике человек пытался сочинить свои философские формулировки. И современный человек действительно построил бы достойную и привлекательную философию самого себя и своей вселенной, если бы не крах его важнейшей и незаменимой метафизической связи между мирами материи и духа --- неспособность метафизики перебросить мост через моронтийную пропасть между физическим и духовным. У смертного человека нет понятия о моронтийном разуме и материале; и \bibemph{откровение ---} это единственный способ возместить эту нехватку концептуальных данных, в которых человек столь остро нуждается, дабы сконструировать логичную философию вселенной и достичь удовлетворительного понимания своего надежного и устойчивого места в этой вселенной.
\vs p103 6:13 Откровение --- вот единственная надежда эволюционирующего человека на преодоление моронтийной пропасти. Вера и рассуждение без помощи моты не могут постичь и создать логическую вселенную. Без озарения моты смертный человек в явлениях материального мира не видит добродетель, любовь и истину.
\vs p103 6:14 Когда философия человека сильно тяготеет к миру материи, она становится рационалистической или \bibemph{натуралистической.} Когда философия особенно склоняется к духовному уровню, она становится \bibemph{идеалистической} или даже мистической. Когда же философия настолько неудачна, что опирается на метафизику, она неизменно становится \bibemph{скептической,} запутанной. В прошлые века в основном человеческие знания и интеллектуальные оценки были подвержены одному из этих трех искажений восприятия. Философия не должна вести свои толкования реальности по линейным законам логики и обязательно должна считаться с эллиптической симметрией реальности, а также с существенной кривизной всех представлений об отношении.
\vs p103 6:15 Высшая доступная смертному человеку философия должна быть логически основана на научном рассуждении, религиозной вере и озарении истиной, которое дает откровение. Благодаря этому союзу человек может отчасти возместить свою неудачу в создании адекватной метафизики и свою неспособность понять моту моронтии.
\usection{7. Наука и религия}
\vs p103 7:1 Наука поддерживается рассуждением, а религия --- верой. Вера, хоть и не основана на рассуждении, тем не менее разумна; хотя и независима от логики, тем не менее здравой логикой поощряется. Вера не может быть взлелеяна даже идеальной философией; на самом деле это она вместе с наукой является источником такой философии. Вера, человеческая религиозная проницательность, может быть верно обучена лишь откровением, может быть верно возвышена только личным переживанием смертным присутствия Бога (который есть дух) в виде духовного Настройщика.
\vs p103 7:2 \pc Истинное спасение --- это метод божественной эволюции разума смертного от отождествления себя с материей до сфер моронтийной связи, а затем и до высшего вселенского статуса духовной корреляции. И как в земной эволюции материальный интуитивный инстинкт предшествует появлению разумного знания, так и проявление духовной интуитивной проницательности предваряет более позднее появление моронтийного и духовного рассуждения и опыта в божественной программе небесной эволюции, в деле превращения потенциалов человека временного в действительность и божественность человека вечного, Райского финалита.
\vs p103 7:3 Однако как идущий по пути восхождения человек для постижения Бога опытом устремляется внутрь и по направлению к Раю, так и для энергетического понимания материального космоса он будет проникать наружу и в пространство. Прогресс науки земной жизнью человека не ограничен; опыт восхождения человека во вселенной и сверхвселенной в немалой степени будет изучением превращения энергии и материальной метаморфозы. Бог есть дух, но Божество есть единство, а единство Божества не только охватывает духовные ценности Отца Всего Сущего и Вечного Сына, но и признает энергетические факты Вселенского Контролера и Райского Острова при том, что эти две фазы вселенской реальности совершенно согласованы в отношениях, имеющих место в разуме Носителя Объединенных Действий, и объединены на конечном уровне в появляющемся Божестве Верховного Существа.
\vs p103 7:4 \pc Союз научной позиции и религиозной проницательности благодаря посредству эмпирической философии является частью человеческого опыта долгого восхождения к Раю. Математическая приблизительность и уверенность, которую дает проницательность, всегда будут требовать гармонизирующей функции разумной логики на всех уровнях опыта, где нет максимального достижения Верховного.
\vs p103 7:5 Однако логика никогда не сможет добиться согласования научных открытий и религиозной проницательности, пока как в научных, так и в религиозных аспектах личности не будет господствовать истина и эти аспекты не будут искренне желать следовать истине, куда бы та ни вела, независимо от заключений, к которым истина может прийти.
\vs p103 7:6 Логика --- это метод философии, ее способ выражения. В сфере истинной науки рассуждение всегда податливо подлинной логике; в сфере же истинной религии вера на основании внутренней точки зрения всегда логична, хотя такая вера и может казаться достаточно необоснованной с точки зрения научного подхода, направленной внутрь. При рассмотрении снаружи внутрь вселенная может представляться материальной; при взгляде же наружу изнутри та же вселенная представляется полностью духовной. Рассуждение происходит от материального сознания, а вера --- от духовного сознания, но через посредство философского размышления, усиленного откровением, логика может подтвердить как направленную внутрь, так и направленную наружу точку зрения и тем самым обеспечить устойчивость и науки, и религии. Таким образом, благодаря обычному соприкосновению с логикой философии, и наука, и религия могут становиться все более терпимыми друг к другу, все менее и менее скептичными.
\vs p103 7:7 И развивающаяся наука, и религия нуждаются в более глубокой и смелой самокритике, в большем осознании неполноты в эволюционном статусе. Учителя и науки, и религии довольно часто бывают слишком самоуверенными и догматичными. Наука и религия могут быть самокритичными только в отношении своих \bibemph{фактов.} Как только рассуждение отрывается от стадии фактов, оно перестает играть свою роль либо быстро вырождается, превращаясь в спутника ложной логики.
\vs p103 7:8 \pc Истина --- понимание космических отношений, вселенских фактов и духовных ценностей --- лучше всего достигается, благодаря служению Духа Истины и лучше всего критикуется \bibemph{откровением.} Но откровение не порождает ни науку, ни религию; его назначение --- согласовывать и науку, и религию с истиной реальности. При отсутствии откровения, либо при неспособности его принять или понять, смертный человек всегда обращался к бесполезному делу --- метафизике, так как она служит единственной человеческой заменой откровения истины или моты моронтийной личности.
\vs p103 7:9 Наука материального мира позволяет человеку управлять своим физическим окружением и до некоторой степени господствовать над ним. Религия, основанная на духовном опыте, --- это источник братского порыва, который позволяет людям жить вместе в сложных условиях цивилизации научной эпохи. Метафизика, но еще более, несомненно, --- откровение, создает общую почву для открытий и науки, и религии и делает возможной человеческую попытку логически увязать эти обособленные, но взаимозависимые области мысли в сбалансированную философию научной стабильности и религиозной уверенности.
\vs p103 7:10 \pc В смертном состоянии ничто не может быть абсолютно доказанным; и наука, и религия основываются на предположениях. На моронтийном уровне постулаты как науки, так и религии могут быть частично доказаны логикой моты. На духовном же уровне максимального статуса необходимость в конечном доказательстве постепенно исчезает перед лицом действительного переживания реальности и опыта, с ней связанного; но и тогда за пределами конечного остается много недоказанного.
\vs p103 7:11 Все разделы человеческой мысли основаны на определенных предположениях, принимаемых, хоть и бездоказательно, конструктивной чувствительностью к реальности, существующей в даре человеческого разума. Наука начинает свой превозносимый путь рассуждений, \bibemph{предполагая} реальность трех вещей: материи, движения и жизни. Религия же начинает с предположения о действительности трех вещей: разума, духа и вселенной --- Верховного Существа.
\vs p103 7:12 Наука становится мысленной сферой математики, материи энергии и времени в пространстве. Религия же берет на себя смелость иметь дело не только с конечным и временным духом, но и с духом вечности и верховенства. И лишь через посредство продолжительного опыта в моте эти две крайности восприятия вселенной можно заставить дать аналогичные толкования истоков, действий, отношений, реальностей и предназначений. Максимальное согласование расхождения между энергией и духом заключается в вовлечении в контур Семи Духов\hyp{}Мастеров; первое объединение этого расхождения происходит в Божестве Верховного, а завершающее единство --- в бесконечности Первоисточника и Центра, в Я ЕСТЬ.
\vs p103 7:13 \pc \bibemph{Рассуждение} есть акт признания умозаключений сознания в отношении опыта, связанного с физическим миром энергии и материи. \bibemph{Вера} есть акт признания действительности духовного сознания --- чего\hyp{}то такого, чему у смертного другого объяснения нет. \bibemph{Логика} является синтетическим стремящимся к истине возрастанием единства веры и рассуждения и основана на дарах разума смертных существ, природном признании вещей, значений и ценностей.
\vs p103 7:14 \pc Настоящее доказательство духовной реальности заключается в присутствии Настройщика Мысли, однако действительность этого присутствия для внешнего мира недоказуема, а доказуема лишь для того, кто таким образом переживает пребывание в нем Бога. Осознание Настройщика основано на интеллектуальном принятии истины, сверхразумном восприятии добродетели и стремлении личности любить.
\vs p103 7:15 Наука открывает материальный мир, религия его оценивает, а философия пытается истолковать его значения, одновременно координируя научную материальную точку зрения с религиозным духовным представлением. Однако история --- это область, в которой наука и религия к полному соглашению прийти не могут.
\usection{8. Философия и религия}
\vs p103 8:1 Хотя и наука, и философия своими логикой и рассуждением могут допускать вероятность существования Бога, только личный религиозный опыт ведомого духом человека может подтвердить достоверность такого верховного и личностного Божества. Благодаря методу такого воплощения живой истины, философская гипотеза о вероятности Бога становится религиозной реальностью.
\vs p103 8:2 Путаница, связанная с переживанием достоверности Бога, происходит от несходных толкований и описаний этого опыта отдельными индивидуумами и различными человеческими расами. Переживание Бога может быть вполне действительным, но рассуждения \bibemph{о} Боге, будучи интеллектуальными и философскими, бывают противоречивыми и часто запутанно ошибочными.
\vs p103 8:3 Хороший и благородный человек может быть совершенно влюблен в свою жену, но абсолютно не способен создать удовлетворительно написанное исследование психологии супружеской любви. Другой человек, любящий свою супругу мало или не любящий ее совсем, может создать такое исследование прекрасным образом. Неспособность любящего человека осмыслить истинную природу любимого ничуть не обесценивает ни реальность, ни искренность его любви.
\vs p103 8:4 \pc Если ты искренне веришь в Бога --- благодаря вере знаешь и любишь его, --- то не позволяй, чтобы реальность такого опыта в какой\hyp{}то мере была ослаблена или принижена порождающими сомнения инсинуациями науки, придирками логики, постулатами философии или ловкими предположениями благонамеренных душ, которые хотели бы создать религию без Бога.
\vs p103 8:5 Уверенность религиозного человека, который знает Бога, не должна нарушаться неуверенностью сомневающегося материалиста; наоборот, глубокая вера и непоколебимая уверенность развивающегося с ростом опыта верующего должна бросать вызов неуверенности неверующих.
\vs p103 8:6 \pc Чтобы принести величайшую пользу и науке, и религии, философия должна избегать крайностей и материализма, и пантеизма. Лишь философия, признающая реальность личности --- постоянство в присутствии изменения, --- может представлять для человека моральную ценность, может служить связью между теориями материальной науки и духовной религии. Откровение --- вот возмещение недостатков развивающейся философии.
\usection{9. Сущность религии}
\vs p103 9:1 Теология занимается интеллектуальным содержанием религии, а метафизика (откровение) --- ее философскими аспектами. Религиозный опыт \bibemph{есть} духовное содержание религии. Невзирая на мифологические причуды и психологические иллюзии интеллектуального содержания религии, ошибочные предположения метафизики и методы самообмана, политическое искажение и социально\hyp{}экономические извращения философского содержания религии, духовный опыт личной религии остается подлинным и действительным.
\vs p103 9:2 Религия занимается чувствами, делами и жизнью, а не просто мышлением. Мышление гораздо более тесно связано с материальной жизнью, и в нем главным образом (но не полностью) должны господствовать рассуждение и научные факты, а в его нематериальном стремлении к духовным сферам --- царить истина. Независимо от того, насколько иллюзорна или ошибочна теология человека, его религия может быть совершенно подлинной и вечно истинной.
\vs p103 9:3 Буддизм в своей исходной форме является одной из лучших религий без Бога, возникших за всю эволюционную историю Урантии, хотя по мере развития этой веры безбожной она не оставалась. Религия без веры --- это противоречие, а без Бога --- философская несостоятельность и интеллектуальный абсурд.
\vs p103 9:4 Магическое и мифологическое происхождение естественной религии отнюдь не умаляет реальность и истину более поздних данных откровением религий и совершенного спасительного евангелия религии Иисуса. Жизнь и учения Иисуса окончательно отделили религию от предрассудков магии, иллюзий мифологии и рабства традиционного догматизма. Однако допуская существование и реальность сверхматериальных ценностей и существ, эта древняя магия и мифология весьма эффективно готовили путь для более поздней и высшей религии.
\vs p103 9:5 Хотя религиозный опыт представляет собой чисто духовное субъективное явление, такой опыт охватывает собой основанное на позитивной и живой вере отношение к высшим сферам объективной реальности вселенной. Идеал религиозной философии --- это такое основанное на вере доверие, которое побуждает человека безгранично полагаться на абсолютную любовь бесконечного Отца вселенной вселенных. Такой подлинно религиозный опыт намного превосходит философское воплощение идеалистического желания; он действительно принимает спасение как должное и занимается лишь постижением и исполнением воли Райского Отца. Отличительные особенности подобной религии таковы: вера в верховное Божество, надежда на вечное продолжение существования и любовь --- особенно к своим собратьям.
\vs p103 9:6 \pc Когда теология подчиняет себе религию, религия умирает, становясь вместо жизни доктриной. Теология призвана лишь способствовать самосознанию личного духовного опыта. Теология представляет собой попытку религии определить, прояснить, развить и оправдать основанные на опыте утверждения религии, которые в конечном итоге могут быть подтверждены только живой верой. В высшей философии вселенной мудрость, подобно рассуждению, становится союзницей веры. Рассуждение, мудрость и вера --- вот высшие человеческие достижения. Рассуждение знакомит человека с миром фактов, с вещами; мудрость знакомит его с миром истины, с отношениями; вера же посвящает человека в мир божественного, приобщает его к духовному опыту.
\vs p103 9:7 Вера в высшей степени охотно увлекает за собой рассуждение и ведет его за собой, пока оно может идти, а потом до полного философского предела продолжает путь вместе с мудростью, после чего решается отправиться в безграничное и бесконечное путешествие по вселенной в сопровождении одной только ИСТИНЫ.
\vs p103 9:8 \pc Наука (знание) основана на присущем (дух\hyp{}помощник) предположении о том, что рассуждение действенно, что вселенная может быть понята. Философия (согласованное понимание) основана на присущем (дух мудрости) предположении, что мудрость действенна, что материальная вселенная может быть согласована с духовной. Религия (истина личного духовного опыта) основана на присущем (Настройщик Мысли) предположении о том, что вера действенна, что Бог познаваем и достижим.
\vs p103 9:9 Полное осознание реальности смертной жизни заключается в возрастающей готовности верить этим предположениям рассуждения, мудрости и веры. Такова жизнь, мотивированная истиной, жизнь, в которой царит любовь; таковы идеалы объективной космической реальности, существование которых нельзя продемонстрировать материально.
\vs p103 9:10 Когда рассуждение отличает правильное от неправильного, оно тем самым проявляет мудрость; когда мудрость выбирает между правильным и неправильным, истиной и ошибкой, она тем самым обнаруживает духовное руководство. Таким образом функции разума, души и духа прочно объединены и функционально взаимосвязаны. Рассуждение занимается фактическим знанием; мудрость --- философией и откровением; вера же --- живым духовным опытом. Через посредство истины человек достигает красоты, а благодаря духовной любви восходит к добродетели.
\vs p103 9:11 Вера ведет к познанию Бога, а не просто к мистическому ощущению божественного присутствия. Вера не должна подвергаться чрезмерному влиянию своих эмоциональных последствий. Истинная религия есть опыт веры и познания, а также удовлетворения чувством.
\vs p103 9:12 \pc Существует реальность религиозного опыта, которая пропорциональна духовному содержанию; такая реальность превосходит рассуждение, науку, философию, мудрость и все прочие человеческие достижения. Убеждения, основанные на таком опыте, --- неопровержимы; логика религиозной жизни --- неоспорима; достоверность такого знания --- сверхчеловеческая; удовлетворение --- в высшей степени божественно; смелость --- неукротима; преданность --- безусловна; верность --- верховна, а предназначения --- окончательны, вечны, предельны и универсальны.
\vsetoff
\vs p103 9:13 [Представлено Мелхиседеком из Небадона.]
