\upaper{124}{Отрочество Иисуса}
\author{Комиссия срединников}
\vs p124 0:1 Хотя Иисус мог бы получить в Александрии лучшее образование, чем в Галилее, однако там у него не было бы столь замечательного окружения, помогавшего ему решать собственные жизненные задачи при минимальном влиянии со стороны процесса обучения; к тому же постоянное общение с большим числом мужчин и женщин всех сословий, происходивших из всех частей цивилизованного мира, давало ему огромное преимущество. Останься Иисус в Александрии, его образованием руководили бы евреи, и оно проходило бы исключительно в русле еврейской культуры. Обучение и образование, полученное в Назарете, наилучшим образом подготовило его к пониманию неевреев и позволило более правильно и взвешено оценивать сравнительные достоинства восточного, или вавилонского, и западного, или эллинского, направления иудейской теологии.
\usection{1. Девятый год жизни Иисуса (3 г. н.э.)}
\vs p124 1:1 Было бы неверным утверждать, что Иисус когда\hyp{}либо серьезно болел, однако в тот год он вместе со своими братьями и малюткой сестрой перенес несколько неопасных детских болезней.
\vs p124 1:2 Учеба его продолжалась, и он по\hyp{}прежнему оставался учеником, который пользовался привилегией ежемесячно в течение недели не посещать школу. Он по\hyp{}прежнему делил это время между поездками с отцом по соседним городам, посещением фермы дяди к югу от Назарета и поездками на рыбную ловлю в Магдалу.
\vs p124 1:3 \pc Самая серьезная проблема, связанная со школой, возникла поздней зимой, когда Иисус осмелился оспорить учителя, считавшего, что все изображения, картины и рисунки являются идолопоклонством по своей сути. Иисус с наслаждением рисовал пейзажи и лепил из гончарной глины самые разнообразные вещи. По еврейским законам все подобные занятия были строжайшим образом запрещены, но к этому времени он настолько уже сумел умиротворить все возражения родителей, что они позволили ему продолжить эти занятия.
\vs p124 1:4 Однако проблема возникла снова, когда один из самых нерадивых учеников увидел, как Иисус рисует древесным углем изображение учителя на полу классной комнаты. Там оно и оставалось для всеобщего обозрения, и многие старейшины видели его, прежде чем совет отправился к Иосифу и потребовал, чтобы он сделал что\hyp{}нибудь, чтобы пресечь беззакония старшего сына. Хотя это был не первый случай, когда к Иосифу и Марии приходили с жалобами на поведение их разностороннего и беспокойного ребенка, это обвинение было наиболее серьезным из всех, которые когда\hyp{}либо ему предъявлялись. Иисус, сидевший снаружи у черного хода, на большом камне, некоторое время слушал обвинения в адрес своих художественных упражнений. Вознегодовав, что в совершенных им проступках обвиняют отца, он вошел внутрь, бесстрашно представ перед обвинителями. Старейшины пришли в замешательство. Некоторые склонны были отнестись к происшедшему с юмором, в то время как один или двое, казалось, считали, что мальчик совершил святотатство, если вообще не богохульство. Иосиф был недоволен, Мария негодовала, но Иисус настоял на том, чтобы его выслушали. Он сказал свое слово, мужественно защитил свою точку зрения и с полным самообладанием объявил, что будет твердо следовать решению своего отца как в этом, так и в других спорных вопросах. И совет старейшин в молчании удалился.
\vs p124 1:5 Мария пыталась уговорить Иосифа разрешить Иисусу лепить из глины дома, при условии, что он пообещает не продолжать этих неприемлемых занятий в школе, но Иосиф счел нужным решить, что в данном случае раввинское толкование второй заповеди должно взять верх. И с того дня и до тех пор, пока Иисус жил в доме своего отца, он больше никогда не рисовал и не лепил никаких изображений. Но он не пришел к убеждению, что это занятие --- дурно, и отказ от столь любимого времяпровождения был одним из величайших испытаний его отрочества.
\vs p124 1:6 \pc В конце июня Иисус впервые вместе со своим отцом поднялся на вершину горы Фавор. День был ясный, и вид с горы открывался величественный. Девятилетнему мальчику казалось, что перед его глазами действительно раскинулся весь мир, кроме Индии, Африки и Рима.
\vs p124 1:7 \pc Вторая сестра Иисуса, Марфа, родилась 13 сентября, в четверг вечером. Три недели спустя Иосиф, в течение какого\hyp{}то времени не покидавший дом, начал строительство пристройки, которая должна была служить одновременно мастерской и спальней. Небольшой станок был сделан и для Иисуса, впервые у него появились собственные инструменты. Многие годы он в свободное время работал за этим станком и стал большим мастером по изготовлению ярм.
\vs p124 1:8 \pc Эта зима и следующая были самыми холодными в Назарете за многие десятилетия. Иисус видел снег в горах, несколько раз он выпадал и в Назарете, но быстро таял. Но до этой зимы он ни разу не видел льда. Тот факт, что вода может пребывать в твердом, жидком и газообразном состоянии --- Иисус долго размышлял, наблюдая, как пар поднимается над кипящими горшками, --- навел мальчика на серьезные размышления о физическом мире и о его устройстве; и вместе с тем личность, воплощенная в подрастающем юнце, была истинным создателем и устроителем всех этих вещей во всей бескрайней вселенной.
\vs p124 1:9 Климат Назарета не был суровым. Январь являлся самым холодным месяцем со средней температурой около 50 F. В июле и августе, самых жарких месяцах, температура колебалась от 75 F до 90 F. От гор до Иордана и долины Мертвого моря климат Палестины менялся от холодного к знойному. Благодаря этому, иудеи были некоторым образом подготовлены к жизни в любом из разнообразных климатов земли.
\vs p124 1:10 Даже во время самых теплых летних месяцев прохладный морской ветерок обычно дул с запада с 10 утра примерно до 10 вечера. Но то и дело ужасные горячие ветры из восточной пустыни проносились через Палестину. Эти обжигающие порывы ветра обычно приходились на февраль и март, незадолго до конца сезона дождей. В этот период с ноября по апрель дождь проливался освежающими потоками, но дожди не были постоянными. В Палестине было всего два времени года --- лето и зима, сезон засухи и сезон дождей. В январе начинали цвести цветы, и к концу апреля вся земля превращалась в огромный цветущий сад.
\vs p124 1:11 \pc В мае того года Иисус впервые помогал с уборкой зерна в хозяйстве своего дяди. К своему тринадцатилетию он уже знал кое\hyp{}что практически обо всех работах, которыми мужчины и женщины занимались в окрестностях Назарета, кроме работ с металлом, а после смерти отца, став постарше, провел несколько месяцев и в кузнице.
\vs p124 1:12 Когда в работах и передвижении караванов наступало затишье, Иисус совершал с отцом множество деловых поездок или прогулок в близлежащие Кану, Ендор и Наин. Еще мальчиком он часто бывал в Сефорисе, находившемся немного более чем в трех милях к северо\hyp{}западу от Назарета, где с 4 г. до н.э. примерно до 25 г. н.э. располагалась столица Галилеи и одна из резиденций Ирода Антипы.
\vs p124 1:13 Иисус продолжал развиваться физически, интеллектуально, социально и духовно. Эти поездки значительно помогли ему лучше и глубже понять собственную семью, и к тому времени даже родители начали уже не только учить его, но и учиться у него. Уже в юности Иисус был оригинальным мыслителем и искусным учителем. Он находился в постоянном противоречии с так называемым «изустным законом», но всегда старался приспособиться к обычаям своей семьи. Он легко находил общий язык с детьми своего возраста, но часто бывал обескуражен медлительностью их мышления. Ему еще не было и десяти, когда он стал лидером группы из семи мальчиков, которые образовали союз, чтобы способствовать достижению зрелости --- физической, интеллектуальной и религиозной. Среди них Иисус успешно ввел много новых игр и разнообразных улучшенных способов активного отдыха.
\usection{2. Десятый год жизни Иисуса (4 г. н.э.)}
\vs p124 2:1 Было 5 июля, первая суббота месяца, когда Иисус, прогуливаясь с отцом в окрестностях Назарета, впервые выразил чувства и мысли, свидетельствовавшие о том, что он начинает осознавать необычный характер своей жизненной миссии. Иосиф внимательно выслушал эти знаменательные слова, но почти ничего не сказал в ответ; не сообщил ему никаких сведений. На следующий день у Иисуса был похожий, но более длинный разговор с матерью. Мария тоже внимательно выслушала слова мальчика, но и она ничего не рассказала ему. С тех пор прошло почти два года до того, как Иисус снова заговорил с родителями о растущем в его сознании откровении о природе собственной личности и характере его земной миссии.
\vs p124 2:2 \pc В августе Иисус поступил в высшую школу при синагоге. В школе у него постоянно возникали осложнения из\hyp{}за вопросов, которые он настойчиво продолжал задавать. Он все больше и больше будоражил весь Назарет. Его родители ни в коем случае не хотели мешать ему задавать эти беспокоившие всех вопросы, а его старший учитель был сильно заинтригован его любопытством, проницательностью и жаждой знаний.
\vs p124 2:3 Приятели Иисуса не находили в его поведении ничего сверхъестественного; во многих отношениях он был точно таким же, как они. Его интерес к учебе был выше среднего, но все же не представлял собой чего\hyp{}то совершенно необычного. Но он задавал в школе больше вопросов, чем другие ученики его класса.
\vs p124 2:4 \pc Его самой необычной и выдающейся чертой было, пожалуй, полное нежелание бороться за свои права. Он был прекрасно развит для своих лет, и его друзьям казалось странным, что он не склонен защищать себя даже от явной несправедливости или в тех случаях, когда подвергался личным оскорблениям. Если же подобное случалось, эта черта Иисуса доставляла ему не слишком большие страдания благодаря дружбе с Иаковом, соседским мальчиком, который был на год его старше. Иаков был сыном каменщика, связанного с Иосифом по работе. Он восхищался Иисусом и взял на себя заботу о том, чтобы никто не смел его обидеть, пользуясь отвращением Иисуса к физическим потасовкам. Несколько раз более взрослые и невоспитанные мальчики нападали на него, полагаясь на всем известную кротость, но они всегда получали быстрое и неотвратимое возмездие от рук добровольного телохранителя Иисуса и постоянного защитника, Иакова, сына каменщика.
\vs p124 2:5 Иисус был общепризнанным главой назаретских мальчиков, которые отстаивали высшие идеалы своего времени и поколения. Он был по\hyp{}настоящему любим своими юными сподвижниками не только за справедливость, но и за то, что обладал даром исключительного, понимающего сочувствия, которое означало любовь и граничило с тихим состраданием.
\vs p124 2:6 В этом году он начал отдавать определенное предпочтение обществу старших. Он с видимым наслаждением говорил о вопросах культуры, образования, социальной жизни, экономики, политики и религии с людьми более зрелыми, и глубина его рассуждений и проницательность наблюдений настолько очаровывали взрослых собеседников, что они всегда охотно проводили с ним время. Пока на Иисуса не легла ответственность за содержание семьи, родители пытались воздействовать на него, желая, чтобы он общался со своими сверстниками или с людьми, близкими по возрасту, а не со старшими и более знающими, общению с которыми он отдавал столь явное предпочтение.
\vs p124 2:7 Позже в том же году он в течение двух месяцев рыбачил со своим дядей на Галилейском море и очень преуспел в этом. Еще не достигнув окончательной зрелости, он стал прекрасным рыбаком.
\vs p124 2:8 Его физическое развитие продолжалось. Он был успевающим и пользующимся привилегиями учеником в школе; дома достаточно хорошо ладил с младшими братьями и сестрами, имея то преимущество, что был на три с половиной года старше самого старшего из остальных детей. К нему хорошо относились в Назарете, не считая родителей некоторых наименее развитых детей, которые часто говорили, что Иисус слишком дерзок, что ему не хватает истинного смирения и юношеской сдержанности. Он проявлял все большую и большую склонность к тому, чтобы ориентировать своих юных приятелей на более серьезные и вдумчивые игры. Он был прирожденным учителем и просто не мог поступать иначе, даже будучи вовлеченным в игру.
\vs p124 2:9 Иосиф рано начал обучать Иисуса различным способам добывания хлеба насущного и объяснял ему преимущества земледелия перед ремеслами и торговлей. Галилея была более красивой и процветающей областью, чем Иудея, и жизнь там была примерно в четыре раза дешевле, чем в Иерусалиме и в Иудее. Это была область земледельческих деревень и процветающих ремесленных городов --- более двухсот городов имели почти пятитысячное население, тридцать --- население более 15 тысяч.
\vs p124 2:10 Когда Иисус впервые поехал с отцом на Галилейское море, чтобы изучить рыболовное дело, он почти решил стать рыбаком, но находясь под большим влиянием профессии отца, позже избрал ремесло плотника, а еще позже сочетание различных влияний привело его к окончательному выбору --- он станет религиозным учителем нового строя.
\usection{3. Одиннадцатый год жизни Иисуса (5 г. н.э.)}
\vs p124 3:1 Весь этот год мальчик продолжал совершать загородные поездки с отцом, часто также ездил на ферму к дяде и иногда --- в Магдалу, чтобы вместе с дядей, который обосновался неподалеку от этого города, заняться рыбной ловлей.
\vs p124 3:2 Иосиф и Мария нередко испытывали искушение проявить некое особенное расположение к Иисусу или как\hyp{}нибудь иначе обнаружить знание того, что он является обетованным ребенком, судьбоносным сыном. Но оба родителя были необычайно мудры и проницательны в этом отношении. Несколько раз, когда они хоть в малейшей степени выказывали свое предпочтение к Иисусу, мальчик немедленно отвергал проявления подобного внимания.
\vs p124 3:3 Долгое время Иисус проводил в лавке, торговавшей всем, что необходимо для караванов, и из разговоров с путешественниками со всех частей света почерпнул удивительно много для его возраста сведений о жизни в других странах. Этот год был последним, когда он свободно наслаждался играми и юношеским весельем. Начиная с этого времени число трудностей и забот в жизни юноши стало резко возрастать.
\vs p124 3:4 \pc В 5 г. н.э., 24 июня, в среду вечером родился Иуда. Рождение этого седьмого ребенка сопровождалось серьезными осложнениями. В течение нескольких недель Мария была так тяжко больна, что Иосиф оставался дома. Иисус был очень занят поручениями отца и заботами, возникшими в связи с серьезной болезнью его матери. Юноше так больше никогда и не удалось вернуться к тому детскому состоянию, которое было свойственно ему в более ранние годы. Со времени болезни матери --- как раз перед тем, как ему должно было исполниться одиннадцать лет, --- он был вынужден взять на себя все обязанности старшего сына в семье, и ему пришлось сделать это на год или на два раньше срока, когда это бремя легло бы на его плечи в обычном случае.
\vs p124 3:5 Хазан раз в неделю проводил один вечер с Иисусом, помогая ему совершенствоваться в священном писании иудеев. Он был очень заинтересован в развитии своего многообещающего ученика, поэтому был готов помогать ему всеми возможными способами. Этот еврейский педагог оказал большое влияние на развивающийся ум Иисуса, но он никак не мог понять, почему все его предложения отправиться в Иерусалим и продолжить образование под руководством ученых раввинов оставляют мальчика столь равнодушным.
\vs p124 3:6 \pc Примерно в середине мая Иисус сопровождал отца во время деловой поездки в Скифополь, главный греческий город Десятиградия, в древности --- иудейский город Бесан. По дороге Иосиф рассказывал много древних историй о царе Сауле, о филистимлянах и о последующих событиях бурной истории Израиля. Чистота и упорядоченное устройство этого так называемого «языческого» города произвели огромное впечатление на Иисуса. Он удивлялся театру под открытым небом и восхищался красивым мраморным храмом, предназначенным для служения «языческим» божествам. Восторги мальчика взволновали Иосифа, он пытался свести на нет эти благоприятные впечатления, превознося красоту и величие иудейского храма в Иерусалиме. Иисус часто с любопытством смотрел на этот великолепный греческий город с Назаретского холма и много раз расспрашивал о его просторных общественных сооружениях и прекрасных зданиях, но его отец всегда старался избегать ответов на эти вопросы. Теперь они оказались лицом к лицу с красотами языческого города, и Иосифу было нелегко оставлять вопросы Иисуса без ответа.
\vs p124 3:7 Случилось, что как раз в это время в амфитеатре Скифополя происходили ежегодные соревнования и состязания в физической силе среди греческих городов Десятиградия, и Иисус настоял, чтобы отец взял его посмотреть игры; он был так настойчив, что Иосиф не решился отказать. Мальчик трепетал от восторга, наблюдая за играми, и всем сердцем проникся духом показа физической красоты и атлетического мастерства. Иосиф был невыразимо потрясен, видя, что его сын восторгается, рассматривая эту выставку «языческого» тщеславия. После того, как игры закончились, Иосиф испытал величайшее изумление в своей жизни, услышав, что Иисус выражает одобрение по поводу игр и считает, что было бы хорошо, если бы и молодые люди Назарета могли бы извлекать пользу из таких здоровых физических упражнений на свежем воздухе. Иосиф долго и проникновенно говорил с Иисусом о пагубной природе подобных занятий, но он понял, что убедить мальчика не удалось.
\vs p124 3:8 Единственный раз в жизни Иисус видел, чтобы отец так рассердился на него, и это было в тот вечер в их комнате в гостинице, когда в пылу спора мальчик настолько забыл о складе еврейского мышления, что предложил вернуться домой и начать работы по строительству амфитеатра в Назарете. Когда Иосиф услышал, что его первенец выражает столь нееврейские чувства, самообладание покинуло его, и он, забывшись, схватил Иисуса за плечо и сердито воскликнул: «Сын мой, чтобы, пока жив, я никогда больше не слышал от тебя столь греховных мыслей!» Иисус был поражен проявлением таких чувств со стороны отца. До этого ему никогда не доводилось испытывать на себе всей силы отцовского негодования, и он был невыразимо удивлен и потрясен. Иисус ответил только: «Хорошо, отец, так и будет». Больше ни разу мальчик не позволил себе даже малейшего намека на игры и другие атлетические упражнения греков до тех пор, пока отец был жив.
\vs p124 3:9 Позже Иисус увидел греческий амфитеатр в Иерусалиме и узнал, насколько подобные вещи, с точки зрения евреев, были отвратительны. Тем не менее всю свою жизнь он пытался внедрить идею здорового отдыха в свой собственный распорядок и, насколько позволяли еврейские обычаи, в дальнейшую программу регулярных занятий для своих двенадцати апостолов.
\vs p124 3:10 К концу одиннадцатого года Иисус был сильным, хорошо развитым, жизнерадостным юношей с чувством юмора, но с этого года он стал все более и более подвержен особым периодам глубокого созерцания и серьезных размышлений. Иисус часто предавался размышлениям о том, как ему удастся выполнить свои обязательства перед семьей и в то же время следовать зову своей миссии в мире; к этому времени он уже постиг, что его служение не должно ограничиться только улучшением жизни еврейского народа.
\usection{4. Двенадцатый год жизни Иисуса (6 г. н.э.)}
\vs p124 4:1 Этот год жизни Иисуса был богат событиями. Он продолжал успешно учиться в школе и был неутомим в изучении природы; кроме того, он настойчиво продолжал изучать различные виды деятельности, с помощью которых люди зарабатывали себе на жизнь. Он начал регулярно работать в домашней плотницкой мастерской, и ему позволили самому распоряжаться заработками --- весьма необычное для иудейской семьи соглашение. В этом же году он понял, насколько мудро хранить в тайне подобные семейные дела. Он начал осознавать, чем именно он вызывал неприятности в деревне, и в дальнейшем стал в высшей степени осмотрительно скрывать все то, что могло бы дать повод считать его отличным от других.
\vs p124 4:2 В течение этого года он много раз переживал периоды неуверенности, а то и сомнений относительно природы своей миссии. Его естественно развивавшийся человеческий ум еще не вполне постиг реальности его дуальной природы. То, что он был наделен единой личностью, затрудняло для его сознания обнаружение двойственного происхождения тех факторов, которые составляли природу, присущую самой его личности.
\vs p124 4:3 Начиная с этого времени он стал лучше ладить со своими братьями и сестрами. Он был все более тактичен, всегда полон сострадания и заботы об их благополучии и счастье и сохранял хорошие отношения с ними вплоть до начала своего общественного служения. Чтобы быть более точным, он лучше всего ладил с Иаковом, Мириам и с двумя младшими детьми (к тому времени еще не родившимися), Амосом и Руфью. Он всегда достаточно легко находил общий язык с Марфой. Все сложности, возникавшие дома, происходили из\hyp{}за трений с Иосифом и Иудой, в особенности с последним.
\vs p124 4:4 \pc Для Иосифа и Марии воспитание этого небывалого сочетания божественного и человеческого было серьезным испытанием, и они заслуживают похвалы за столь добросовестное и успешное выполнение родительских обязанностей. Родители Иисуса все больше убеждались, что старшему сыну действительно присуще нечто сверхчеловеческое, но даже и помыслить не могли, что этот обетованный сын и в самом деле есть подлинный творец всей локальной вселенной со всем, что в ней находится и живет. Иосиф и Мария жили и умерли, так и не узнав, что их сын Иисус действительно был создателем Вселенной, облекшимся в плоть человека.
\vs p124 4:5 В этом году Иисус уделял музыке больше, чем когда бы то ни было, времени; он продолжал учить в домашних условиях братьев и сестер. Примерно в это время мальчик тонко почувствовал разницу во взглядах Иосифа и Марии на характер его миссии. Он часто слышал беседы родителей, когда они думали, что сын крепко спит, и много размышлял над различием мнений своих родителей. Иисус все больше и больше склонялся к точке зрения отца, так что матери суждено было испытать огорчение от сознания, что сын постепенно отвергает ее руководство в вопросах, имеющих отношение к его жизненному пути. С годами брешь во взаимопонимании между ними становилась все шире. Все меньше и меньше понимала Мария значимость миссии Иисуса, и все больше эту заботливую мать задевало то, что сын не стремится воплощать ее горячие надежды.
\vs p124 4:6 Иосиф питал все возраставшую веру в духовную природу миссии Иисуса, и если бы не другие и более важные причины, можно было бы считать несчастьем, что ему не суждено было дожить до дня, когда бы он смог увидеть на земле исполнение своих представлений о пришествии Иисуса.
\vs p124 4:7 \pc В этот последний год учебы, в возрасте двенадцати лет, Иисус высказал отцу свои соображения по поводу еврейского обычая каждый раз при входе или выходе из дома, прикасаться к кусочку пергамента, прибитого к дверному косяку, и затем целовать палец, который прикасался к пергаменту. Частью ритуала были слова: «Да сохранит нас Господь, входящих и выходящих, ныне и во веки веков». Иосиф и Мария неоднократно говорили Иисусу, что нельзя создавать статуи и рисовать картины, объясняя, что подобные творения могут быть использованы в идолопоклоннических целях. Хотя Иисусу не удалось до конца понять смысл запрета создавать статуи и писать картины, он чутко ощущал непоследовательность запретов, и поэтому указал своему отцу на, по существу, идолопоклоннический характер этого традиционного поклона пергаменту, прибитому к косяку. После того, как Иисус таким образом убедил отца, Иосиф снял пергамент.
\vs p124 4:8 По мере того, как шло время, Иисус многое делал, чтобы изменить форму религиозных обрядов, включавших семейные молитвы и другие обычаи. Многое можно было сделать в этом отношении в Назарете, так как городская синагога находилась под влиянием либеральной школы раввинов, примером для которой служил знаменитый учитель из Назарета Иосий.
\vs p124 4:9 В течение этого и двух последующих лет Иисус находился в сильном психическом напряжении, вследствие постоянных усилий примирить собственные взгляды на религиозную практику и социальные обычаи со взглядами родителей. Он был смущен противоречием между необходимостью быть верным своим собственным убеждениям и с уважением подчиняться родителям; две важнейшие заповеди, занимавшие основное место в его юношеском сознании, вступали в сильнейший конфликт. Одна из них гласила: «Будь верен велению своих высочайших убеждений истины и праведности». Второй же была: «Почитай отца и мать твою, ибо они дали тебе жизнь и вскормили тебя». Однако он никогда не уклонялся от ответственности за необходимое ежедневное примирение этих двух сфер --- верности личным убеждениям и долга перед семьей. Он почувствовал себя удовлетворенным только тогда, когда все более гармоничное сочетание личных убеждений и семейных обязательств вылилось в глубокую идею общей солидарности, основанной на верности, справедливости, терпимости и любви.
\usection{5. Тринадцатый год жизни Иисуса (7 г. н.э.)}
\vs p124 5:1 В этом году мальчик из Назарета перешел от детства к началу юности; его голос начал меняться, и другие черты ума и тела также свидетельствовали о приближении возмужания.
\vs p124 5:2 В 7 г. н.э., 9 января, в воскресенье ночью родился его младший брат Амос. Иуде не было еще и двух лет, а его младшей сестре Руфи еще только предстояло родиться; итак, можно видеть, что в семье Иисуса было много маленьких детей, оставшихся на его попечении после внезапной смерти их отца в следующем году.
\vs p124 5:3 \pc Примерно к середине февраля Иисус по\hyp{}человечески убедился, что на земле целью миссии, к исполнению которой он призван, было просвещение людей и откровение Бога. Важные решения и далеко идущие планы зрели в уме юноши, который внешне был обычным иудейским парнем из Назарета. Интеллектуальная жизнь всего Небадона зачарованно и с изумлением наблюдала за тем, как это все стало проявляться в мыслях и поступках сына плотника, теперь уже юноши.
\vs p124 5:4 \pc В 7 г. н.э., 20 марта, в первый день недели Иисус закончил курс обучения в местной школе при Назаретской синагоге. Это был великий день в жизни всякой честолюбивой еврейской семьи, день, когда ее сына\hyp{}первенца объявляли «сыном заповеди» и выкупленным первенцем Господа Бога Израиля, «ребенком Всевышнего» и слугой Господа Всей земли.
\vs p124 5:5 В предыдущую пятницу Иосиф вернулся из Сефориса, где ему были поручены работы по строительству нового общественного здания, чтобы присутствовать при этом торжественном событии. Учитель Иисуса твердо верил, что сообразительному и усердному ученику уготована некая выдающаяся карьера, особая миссия. Старейшины, несмотря на трудности, возникавшие в связи с определенными неортодоксальными воззрениями Иисуса, были горды юношей и уже начинали строить планы, которые позволят ему отправиться в Иерусалим, чтобы продолжить образование там, в знаменитых академиях Иудеи.
\vs p124 5:6 По мере того, как Иисус время от времени слышал обсуждение этих планов, он все более и более убеждался, что он никогда не поедет в Иерусалим учиться у раввинов. Но он и не подозревал о трагедии, которая должна была разразиться так скоро и которая вынудит его отказаться от всех планов и заставит взять на себя ответственность за пропитание и благополучие большой семьи, где вскоре будет уже пять братьев и трое сестер, а также его мать и он сам. Опыт Иисуса в воспитании семьи был более длительным, чем тот, который выпал на долю Иосифа, его отца; и он достиг того уровня, который впоследствии сам себе поставил: быть мудрым, терпеливым, понимающим и эффективным учителем и старшим братом для всей семьи --- семьи, так внезапно пораженной горем и так внезапно понесшей утрату.
\usection{6. Путешествие в Иерусалим}
\vs p124 6:1 Достигнув порога молодости и официально закончив школу при синагоге, Иисус получил право отправиться с родителями в Иерусалим, чтобы вместе с ними участвовать в праздновании своей первой Пасхи. В том году праздник Пасхи пал на субботу, 9 апреля 7 г. н.э. В понедельник, 4 апреля, ранним утром довольно большая компания (103) была готова отправиться из Назарета в Иерусалим. Они направились на юг, к Самарии, но, достигнув Изрееля, повернули на восток и, пройдя мимо горы Гелвуй, спустились в долину Иордана, чтобы миновать Самарию. Иосиф и его семья были бы рады попасть в Иерусалим через Самарию, по дороге через колодец Иакова и Бетель, но так как евреи не любили иметь дело с самарянами, они решили двигаться вместе с соседями по дороге через долину Иордана.
\vs p124 6:2 Вселявший великий страх Архелай был смещен, и они могли больше не бояться появиться с Иисусом в Иерусалиме. Двенадцать лет прошло с тех пор, как первый Ирод искал погубить младенца из Вифлеема, и теперь уже никому не пришло бы в голову связать те события с этим никому не известным юношей из Назарета.
\vs p124 6:3 Прежде чем достигнуть перекрестка дорог у Изрееля, они, продолжая свое путешествие, очень скоро оставили с левой стороны старинную деревню Сунем, и Иисус вновь услышал историю о самой красивой девушке всего Израиля, которая жила там, и о замечательных деяниях, которые там совершал Елиша. Проходя через Изреель, родители Иисуса рассказывали ему о деяниях Ахава и Иезавели и о подвигах Ииуя. Огибая гору Гелвуй, они много говорили о Сауле, который окончил свою жизнь на склоне этой горы, о царе Давиде и событиях, связанных с этим историческим местом.
\vs p124 6:4 Когда пилигримы обошли подножие горы Гелвуй, справа их взорам открылся греческий город Скифополь. Они издали смотрели на мраморные строения, но не приближались к языческому городу, дабы таким образом не оскверниться и вследствие этого не иметь права участвовать в приближавшейся торжественной и священной церемонии празднования Пасхи в Иерусалиме. Мария не могла понять, почему ни Иосиф, ни Иисус не заговаривали о Скифополе. Она не знала об их прошлогодних разногласиях, так как они никогда не говорили ей об этом эпизоде.
\vs p124 6:5 Теперь дорога шла прямо вниз в тропическую долину Иордана, и вскоре изумленному взору Иисуса открылся извилистый, со множеством излучин Иордан, с его сверкающими и журчащими водами, текущий вниз к Мертвому морю. Продвигаясь к югу по этой тропической долине, они постепенно скидывали свои верхние одежды, а перед их восхищенным взором представали изобильные поля пшеницы и прекрасные олеандры, усыпанные розовыми цветами, в то время как массивная, с покрытой снегом вершиной гора Ермон возвышалась далеко на севере, величественно взирая на историческую долину. Спустя чуть больше трех часов пути от места, находящегося напротив Скифополя, они подошли к бурному ручью и там остановились на ночлег, прямо под сияющим звездами небом.
\vs p124 6:6 \pc На второй день путешествия они прошли то место, где Иабок с востока впадает в Иордан, и глядя на восток, в долину вверх по течению реки, они вспомнили о днях Гедеона, когда мадианитяне ворвались в этот край, чтобы опустошить его. К концу второго дня пути они разбили лагерь у подножия самой высокой горы, возвышающейся над долиной Иордана, горы Сартабе, на чьей вершине стояла Александрийская крепость, в которую Ирод заточил одну из своих жен и в которой были похоронены два задушенных им сына.
\vs p124 6:7 На третий день они прошли мимо двух селений, которые были недавно построены Иродом, и отметили их превосходную архитектуру и красоту их финиковых садов. К наступлению ночи они достигли Иерихона, и остались до утра. В тот вечер Иосиф, Мария и Иисус прошли примерно полторы мили до старинного Иерихона, в котором Иешуа, в честь которого был назван Иисус, совершил, согласно иудейскому преданию, свои знаменитые подвиги.
\vs p124 6:8 К четвертому и последнему дню путешествия дорога превратилась в непрерывную процессию паломников. Теперь они начали взбираться на холмы, ведущие к Иерусалиму. Когда они приблизились к вершине, их взору открылись горы за Иорданом и медлительные воды Мертвого моря. Примерно на полпути к Иерусалиму Иисус впервые увидел Масличную гору (место, которому предстояло стать столь важной частью его последующей жизни), и Иосиф сказал ему, что Святой Город лежит как раз за этим хребтом. Сердце юноши забилось сильнее от радостного предвосхищения --- вскоре ему предстояло увидеть город и дом своего Небесного Отца.
\vs p124 6:9 На восточных склонах Масличной горы они остановились на отдых в деревушке Вифания. Гостеприимные жители высыпали из домов, чтобы услужить паломникам. Случилось так, что Иосиф с семьей остановились у дома некоего Симона, у которого было трое детей примерно того же возраста, что Иисус, Марфа, Мария и Лазарь. Они пригласили семью из Назарета войти и отдохнуть, и между двумя семьями завязалась дружба, продлившаяся всю жизнь. Позже на протяжении своей богатой событиями жизни Иисус часто останавливался в этом доме.
\vs p124 6:10 Они продолжали двигаться дальше и вскоре оказались на краю Масличной горы, и Иисус впервые (на своей памяти) увидел Святой Город, великолепные дворцы и поразительный храм своего Отца. Никогда в жизни Иисус не испытывал такого чисто человеческого трепета, подобного тому, что с такой силой пронзил его, когда он стоял на Масличной горе, тем апрельским днем, впервые упиваясь видом Иерусалима. И в последующие годы он стоял на том же самом месте и оплакивал город, который был готов отвергнуть еще одного пророка, последнего и величайшего из своих небесных учителей.
\vs p124 6:11 Они поспешно продолжали свой путь к Иерусалиму. Был уже вечер четверга. Достигнув города, они прошли мимо храма, и никогда прежде Иисус не видел подобного сборища людей. Он погрузился в глубокие размышления над тем, как эти евреи собрались здесь из самых отдаленных концов цивилизованного мира.
\vs p124 6:12 Вскоре путники достигли места, приготовленного для их проживания в течение Пасхальной недели, большого дома одного из преуспевающих родственников Марии, человека, знавшего от Захарии кое\hyp{}что из ранней истории и Иоанна, и Иисуса. Следующий день, день приготовлений, они готовились к достойному празднованию Пасхальной субботы.
\vs p124 6:13 В то время как весь Иерусалим бурлил, готовясь к Пасхе, Иосиф нашел время сводить Иисуса в академию, где ему через два года, по достижении требуемого пятнадцатилетнего возраста, предстояло продолжить свое образование. Иосиф был искренне удивлен, увидев, как мало интереса проявляет Иисус к этим столь тщательно вынашиваемым им планам.
\vs p124 6:14 И сам храм, и все происходившее в нем службы и прочая деятельность вокруг храма произвели на Иисуса сильное впечатление. Впервые с тех пор, как ему исполнилось четыре года, он был так погружен в свои собственные размышления, что почти ни о чем не спрашивал. Все же Иисус задал отцу несколько смутивших его вопросов (как задавал и в предыдущих случаях) о том, почему Отец Небесный требует убийства столь многих невинных и беспомощных животных. И отец ясно понял по выражению лица юноши, что его ответы и попытки объяснить неубедительны; они не удовлетворяли его глубокомыслящего и проницательного сына.
\vs p124 6:15 \pc В канун Пасхальной Субботы поток духовных озарений пронесся через смертный разум Иисуса и наполнил его человеческое сердце, исполненное любви, жалостью к духовно слепым и морально невежественным толпам людей, собравшихся здесь, чтобы отпраздновать воспоминание о древней Пасхе. Это был один из самых необычных дней, проведенных Сыном Бога во плоти; и ночью, впервые за время земного пути, ему явился вестник, посланный из Спасограда Иммануилом, который сказал: «Час настал. Пришло тебе время быть в том, что принадлежит Отцу твоему».
\vs p124 6:16 Таким образом, прежде, чем тяжелая ответственность за Назаретскую семью легла на его юные плечи, Иисусу явился посланник, чтобы напомнить юноше, которому не было еще и тринадцати лет, что настал час вспомнить об ответственности за Вселенную. Это был первый случай в длинной череде событий, которые в конце концов достигли своей кульминации тогда, когда завершилось пришествие Сына на Урантии и «управление Вселенной вновь было возложено на его богочеловеческие плечи».
\vs p124 6:17 С течением времени тайна воплощения стала для всех нас все более и более необъяснимой. Мы с трудом могли осознать, что этот юноша из Назарета был создателем всего Небадона. И даже теперь, мы так же не понимаем, как дух этого Сына\hyp{}Творца и дух его Райского Отца связаны с душами человечества. По прошествии времени мы увидели, что его человеческий разум все яснее осознает то, что, пока он проживает эту жизнь во плоти, в духе продолжает нести бремя ответственности за всю вселенную.
\vs p124 6:18 \pc Так кончается история Назаретского мальчика и начинается повествование о мужающем юноше --- все более осознающем себя богочеловеке, --- который начинает размышлять о своей миссии в мире по мере того, как старается объединить свои все более расширяющиеся представления о цели своей жизни с желаниями своих родителей и обязанностями перед семьей и перед обществом своего времени и своей эпохи.
