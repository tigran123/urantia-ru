\upaper{7}{Связь Вечного Сына со вселенной}
\author{Божественный Советник}
\vs p007 0:1 Изначальный Сын всегда заботится об осуществлении духовных сторон вечной цели Отца по мере того, как она все больше и больше раскрывается в феномене развивающихся вселенных с их разнообразными группами живых существ. Этот вечный план мы не во всем понимаем, но, без сомнения, Вечный Сын его понимает.
\vs p007 0:2 Сын подобен Отцу в том, что он стремится наделить всем, чем может, своих равноправных Сынов и подчиненных им Сынов. И Сын разделяет с Отцом его самораспределяющуюся природу в неограниченном даровании себя Бесконечному Духу, их общему объединенному делателю.
\vs p007 0:3 \pc Как вседержитель духовных реальностей Второй Источник и Центр является вечным противовесом Райскому Острову, который так великолепно вершит в себе все материальные вещи. Таким образом, именно Первоисточник и Центр всегда проявляется в материальной красоте утонченных паттернов центрального Острова и в духовных ценностях совершенной личности Вечного Сына.
\vs p007 0:4 \pc Вечный Сын есть настоящий вседержитель безбрежного творения реальностей духа и духовных существ. Мир духа есть характерная особенность Сына, его личный образ действий, и неличностные реальности духовной природы всегда отзываются на волю и цель совершенной личности Абсолютного Сына.
\vs p007 0:5 Сын, однако, лично не ответственен за поведение всех духовных личностей. Воля личностного создания относительно свободна, и поэтому она определяет действия таких волевых существ. Следовательно, свободно\hyp{}волевой мир духа не всегда правильно представляет характер Вечного Сына так же, как и природа на Урантии не является истинным откровением совершенства и неизменности Рая и Божества. Но неважно, что может характеризовать действие свободной воли человека или ангела, вечная власть Сына над вселенским гравитационным контролем всех духовных реальностей остается абсолютной.
\usection{1. Духовно\hyp{}гравитационный контур}
\vs p007 1:1 Все, чему учат об имманентности Бога, о его вездесущности, всемогуществе и всеведении, в духовных областях, равно справедливо и для Сына. Чистая и вселенская духовная гравитация всего творения, этот исключительно духовный контур, непосредственно приводит к лицу Второго Источника и Центра в Раю. Он контролирует и управляет этим всепроникающим и безошибочным духовным обладанием всеми истинными духовными ценностями. Таким образом, Вечный Сын осуществляет абсолютное духовное владычество. Он буквально держит все духовные реальности и все одухотворенные ценности, как если бы они находились на ладони его руки. Контроль над вселенской духовной гравитацией есть вселенское духовное владычество.
\vs p007 1:2 Этот гравитационный контроль духовных вещей действует независимо от времени и пространства; поэтому духовная энергия не уменьшается при передаче. Духовная гравитация никогда не претерпевает изменений во времени и в пространстве. Она не уменьшается пропорционально квадрату расстояния, на которое передается; контуры чисто духовной мощи не замедляются массой материального творения. И этот выход чисто духовных энергий за пределы времени и пространства свойственен абсолютности Сына; это не обусловлено вмешательством антигравитационных сил Третьего Источника и Центра.
\vs p007 1:3 Духовные реальности откликаются на притягивающую силу центра духовной гравитации в соответствии с их качественной ценностью, реальной степенью их духовной природы. Духовная субстанция (качество) так же реагирует на духовную гравитацию, как организованная энергия физической материи (количество) реагирует на физическую гравитацию. Духовные ценности и духовные силы реальны. С точки зрения личности, дух есть душа творения; материя есть призрачное физическое тело.
\vs p007 1:4 Реакции и флуктуации духовной гравитации всегда точно соответствуют содержанию духовных ценностей, качественному духовному статусу индивидуума или мира. Эта притягивающая сила мгновенно реагирует на междуховные и внутридуховные ценности любой вселенской ситуации или планетарного состояния. Каждый раз, когда духовная реальность актуализируется во вселенных, такое изменение неизбежно влечет за собой немедленную и мгновенную перенастройку духовной гравитации. Такой новый дух в действительности является частью Второго Источника и Центра, и совершенно очевидно, что, как смертный человек становится одухотворенным существом, точно так же и он достигнет духовного Сына, центра и источника духовной гравитации.
\vs p007 1:5 \pc Духовная притягивающая сила Сына свойственна --- в меньшей степени --- и многим Райским чинам сыновства. Ибо внутри абсолютного духовно\hyp{}гравитационного контура существуют такие локальные системы духовной притягательности, которые функционируют в меньших единицах творения. Такие субабсолютные средоточия духовной гравитации являются частью божественности личностей Творцов времени и пространства и согласованы с возникающим опытным сверхконтролем Верховного Существа.
\vs p007 1:6 Духовно\hyp{}гравитационное притяжение и ответ на него действуют не только во вселенной, взятой как целое, но также и среди индивидуумов и групп индивидуумов. Между духовными и одухотворенными личностями любого мира, нации или группы верующих индивидуумов существует духовное сцепление. Между духовными индивидуумами, имеющими одинаковые вкусы и стремления, несомненно существует притягательность духовного свойства. Выражение \bibemph{родственные души} не есть просто фигура речи.
\vs p007 1:7 \pc Как и материальная гравитация Рая, духовная гравитация Вечного Сына является абсолютной. Грех и бунт могут препятствовать действию контуров локальной вселенной, но ничто не может приостановить духовную гравитацию Вечного Сына. Бунт Люцифера породил множество изменений в вашей системе обитаемых миров, а также и на Урантии, но мы не видим, чтобы последовавший в результате этого духовный карантин вашей планеты хоть в малой степени отразился на присутствии и функционировании вездесущего духа Вечного Сына или связанного с ним духовно\hyp{}гравитационного контура.
\vs p007 1:8 \pc Все реакции духовно\hyp{}гравитационного контура великой вселенной предсказуемы. Мы осознаем все действия вездесущего духа Вечного Сына и реакции на эти действия и считаем их надежными. Мы можем, в соответствии с хорошо известными законами, измерить духовную гравитацию (и мы делаем это), точно так же, как человек пытается рассчитать действия конечной физической гравитации. Дух Сына неизменно откликается на все духовные вещи, существа и индивидуумы, и этот отклик всегда находится в соответствии со степенью актуальности (качественной степенью реальности) всех таких духовных ценностей.
\vs p007 1:9 Но наряду с этим очень надежным и предсказуемым функционированием духовного присутствия Вечного Сына встречаются явления, реакции которых не столь предсказуемы. Такие явления, вероятно, означают согласованные действия Божественного Абсолюта в сферах появляющихся духовных потенциалов. Мы знаем, что духовное присутствие Вечного Сына есть влияние величественной и бесконечной личности, но едва ли мы можем считать личностными реакции, связанные с предполагаемой деятельностью Божественного Абсолюта.
\vs p007 1:10 \pc Если посмотреть с личностной точки зрения и персонально, то окажется, что Вечный Сын и Божественный Абсолют связаны следующим образом: Вечный Сын господствует в области актуальных духовных ценностей, в то время как Божественный Абсолют заполняет, по\hyp{}видимому, безграничную область потенциальных духовных ценностей. Всякая актуальная ценность духовной природы находит приют в гравитационном объятии Вечного Сына, но если речь идет о потенциальных ценностях, то они, по\hyp{}видимому, находятся в присутствии Божественного Абсолюта.
\vs p007 1:11 Надо полагать, что дух появляется из потенциалов Божественного Абсолюта; развивающийся дух обретает корреляцию в опытных и незавершенных владениях Верховного и Предельного; окончательное же предназначение дух обретает, в конце концов, в абсолютной власти духовной гравитации Вечного Сына. Таким представляется цикл опытного духа, но экзистенциальный дух присущ бесконечности Второго Источника и Центра.
\usection{2. Администрация Вечного Сына}
\vs p007 2:1 Присутствие и личностная деятельность Изначального Сына в Раю являются полными, абсолютными в духовном смысле. Когда мы проходим из Рая --- через Хавону --- в области семи сверхвселенных, мы обнаруживаем все меньшую и меньшую личностную активность Вечного Сына. Во вселенных, появившихся после Хавоны, присутствие Вечного Сына персонализировано в Райских Сынах, обусловлено опытными реальностями Верховного и Предельного и согласовано с неограниченным духовным потенциалом Божественного Абсолюта.
\vs p007 2:2 В центральной вселенной личностную деятельность Изначального Сына можно распознать в изумительной духовной гармонии вечного творения. Хавона так изумительно совершенна, что духовный статус и энергетические состояния этой вселенной паттерна находятся в вечном и совершенном равновесии.
\vs p007 2:3 В сверхвселенных Сын лично не присутствует и не пребывает; в этих творениях он осуществляет только сверхличностное представительство. Эти духовные выражения Сына не являются личностными; они не содержатся в личностном контуре Отца Всего Сущего. Мы не знаем лучшего термина, чем \bibemph{сверхличности} для того, чтобы их обозначить; и они конечные существа; они не являются ни абсонитными, ни абсолютными.
\vs p007 2:4 Администрация Вечного Сына в сверхвселенных, будучи исключительно духовной и сверхличностной, не распознаваема созданиями\hyp{}личностями. Все же всеобъемлющее духовное побуждение, обусловленное личностным влиянием Сына, ощущается в каждой фазе деятельности всех секторов, которые принадлежат областям Древних Дней. Однако мы видим, что в локальных вселенных Вечный Сын лично присутствует в лице Райских Сынов. Здесь бесконечный Сын духовно и творчески функционирует в личностях величественного отряда равноправных Сынов\hyp{}Творцов.
\usection{3. Связь Вечного Сына с индивидуумом}
\vs p007 3:1 В локальной вселенной восходящие смертные, живущие во времени, смотрят на Сына\hyp{}Творца как на личного представителя Вечного Сына. Но когда они вступают на стезю восхождения в сверхвселенную, путешествующие во времени начинают все больше и больше замечать возвышенное присутствие вдохновляющего духа Вечного Сына, и они могут воспользоваться восприятием этого служения передачи духовной энергии. В Хавоне восходящие еще больше сознают любящее объятье всеобъемлющего духа Изначального Сына. Ни на одном из этапов всего смертного восхождения дух Вечного Сына никогда не пребывает в разуме или душе путешественника во времени, но его благодеяние всегда рядом и постоянно заботится о благосостоянии и духовной безопасности продвигающихся вперед сынов, живущих во времени.
\vs p007 3:2 Духовно\hyp{}гравитационное притяжение Вечного Сына составляет тайну, присущую Райскому восхождению человеческих душ, переживших смерть. Все истинные духовные ценности и все по\hyp{}настоящему одухотворенные индивидуумы удерживаются в пределах неизменной духовной гравитации Вечного Сына. Например, смертный разум начинает свое продвижение как материальный механизм и в конце концов достигает уровня Отряда Финалистов --- почти совершенного духовного существования, причем в течение всех этих жизненных переживаний он постепенно все менее подчиняется материальной гравитации и, соответственно, все более откликается на внутренне притягивающее воздействие духовной гравитации. Духовно\hyp{}гравитационный контур буквально тянет душу человека по направлению к Раю.
\vs p007 3:3 \pc Духовно\hyp{}гравитационный контур является основным каналом передачи искренних молитв, исходящих от сердца верующего человека, с уровня человеческого сознания на уровень актуального сознания Божества. То, что в ваших просьбах представляет духовную ценность, будет отобрано вселенским контуром духовной гравитации и немедленно и одновременно передано всем божественным личностям, которых это касается. Каждая займется тем, что принадлежит области его личного ведения. Следовательно, в практике вашего религиозного опыта несущественно, имеете ли вы в виду, адресуя вашу просьбу, Сына\hyp{}Творца вашей локальной вселенной или Вечного Сына, находящегося в центре всех вещей.
\vs p007 3:4 \pc Различающее действие духовно\hyp{}гравитационного контура, возможно, можно сравнить с функционированием нервных контуров в материальном теле человека: ощущения проходят вовнутрь по нервным волокнам; некоторые задерживаются, и ответом на них является автоматическая реакция спинномозговых центров; другие доходят до центров нижних отделов мозга, реакция которых не в такой степени является автоматической, а определяется обучением и навыком; в то время как наиболее жизненно важные поступающие сообщения передаются этими низшими центрами далее и немедленно регистрируются на самых высших уровнях человеческого сознания.
\vs p007 3:5 Но насколько более совершенны замечательные методы духовного мира! Если в вашем сознании возникает что\hyp{}либо, представляющее верховную духовную ценность, то как только вы это выразили, никакая сила во вселенной не может помешать его передаче непосредственно Абсолютной Духовной Личности всего творения.
\vs p007 3:6 И наоборот, если ваши просьбы исключительно материальные и полностью эгоцентричные, не существует способа, посредством которого такие недостойные молитвы могут попасть в духовный контур Вечного Сына. Для любой просьбы, содержание которой не «продиктовано духом», не может быть места во вселенском духовном контуре; такие чисто эгоистичные и материальные просьбы рассеиваются как дым; они не восходят в контуры истинных духовных ценностей. Такие слова --- лишь «медь звенящая или кимвал звучащий».
\vs p007 3:7 Лишь мотивирующая мысль, духовное содержание придают ценность просьбе смертного. Слова ничего не стоят.
\usection{4. Планы божественного совершенства}
\vs p007 4:1 Вечный Сын постоянно взаимодействует с Отцом в процессе успешного выполнения \bibemph{божественного плана продвижения ---} вселенского плана творения, эволюции, восхождения и совершенствования созданий, обладающих волей. И в божественной верности Сын есть вечно равный Отцу.
\vs p007 4:2 Отец и его Сын выступают как единое целое в формулировании и выполнении этого гигантского плана достижения, плана продвижения материальных существ, живущих во времени, к совершенству вечности. Этот проект духовного возвышения восходящих душ, обитающих в пространстве, есть совместное творение Отца и Сына, и они --- вместе с Бесконечным Духом --- совместно воплощают свою божественную цель.
\vs p007 4:3 \pc Этот божественный план достижения совершенства охватывает три уникальных, но замечательно согласованных предприятия вселенского восхождения:
\vs p007 4:4 \ublistelem{1.}\bibnobreakspace \bibemph{План прогрессивного достижения.} Это план Отца Всего Сущего, план эволюционного восхождения; программа, безоговорочно принятая Вечным Сыном, когда он согласился с предложением Отца, гласившим: «Сотворим человека по образу нашему». Это условие для перехода созданий, живущих во времени, на более высокий уровень предполагало дарование Отцом Настройщиков Мысли и наделение материальных созданий прерогативами личности.
\vs p007 4:5 \ublistelem{2.}\bibnobreakspace \bibemph{План пришествия.} Следующий вселенский план --- это великое начинание Вечного Сына и его равноправных сыновей, заключающееся в откровении Отца. Это предложение Вечного Сына, и оно состоит в пришествии Сынов Бога на эволюционные творения, чтобы персонализировать, фактуализировать, воплотить и сделать реальной любовь Отца и милосердие Сына к созданиям всех вселенных. Действия Райских Сынов восстанавливают то, что введенная в заблуждение воля создания подвергла духовной опасности ибо такие действия присущи плану пришествия и являются временными выражениями этой любовной помощи. Когда бы и где бы ни произошла задержка в функционировании плана достижения, если случается, что бунт искажает или осложняет этот проект, тогда немедленно вводятся в действие экстренные меры плана пришествия. Райские Сыны стоят на страже, готовые действовать: идти в сущность самого бунта и там восстанавливать духовный статус сфер. И такую героическую службу сослужил равноправный Сын\hyp{}Творец на Урантии, осуществляя опытное продвижение по достижению владычества во время своего пришествия.
\vs p007 4:6 \ublistelem{3.}\bibnobreakspace \bibemph{План милосердного служения.} Когда план достижения и план пришествия были сформулированы и возвещены, Бесконечный Дух --- один и сам по себе --- составил и привел в действие потрясающее вселенское предприятие милосердного служения. Эта служба очень важна для практической и эффективной работы по осуществлению как достижения, так и пришествия, и все духовные личности Третьего Источника и Центра принимают участие в духе милосердного служения, который в столь большой степени является частью природы Третьего Лица Божества. Не только в творении, но и в управлении Бесконечный Дух действует истинно и в точности как объединенный распорядитель и Отца, и Сына.
\vs p007 4:7 \pc Вечный Сын --- личный опекун, божественный хранитель вселенского Отцовского плана восхождения созданий. Провозгласив вселенское установление: «Будьте совершенны, как и я совершенен», Отец поручил осуществление этого грандиозного дела Вечному Сыну; и Вечный Сын разделяет попечение об этом величественном предприятии со своим божественным соратником --- Бесконечным Духом. Таким образом, Божества эффективно сотрудничают в деле творения, контроля, эволюции, откровения и служения, а если требуется --- реконструкции и восстановления.
\usection{5. Дух пришествия}
\vs p007 5:1 Вечный Сын без колебаний присоединился к Отцу Всего Сущего, провозгласив это потрясающее предписание всему творению: «Будьте совершенны, как и совершенен Отец ваш в Хавоне». И с той поры этот приказ\hyp{}приглашение лежит в основе всех планов продолжения существования и проектов пришествия Вечного Сына и его огромной семьи равноправных Сынов и Сынов\hyp{}сподвижников. И в самих этих пришествиях Сыны Бога стали для всех эволюционирующих созданий «путем, истиной и жизнью».
\vs p007 5:2 \pc Вечный Сын не может непосредственно сносится с людьми, как это делает Отец с помощью дара предличностных Настройщиков Мысли, но Вечный Сын близко приближается к сотворенным личностям посредством ряда идущих вниз градаций божественного сыновства до тех пор, пока он не получит возможность находиться в человеческом присутствии, а временами --- и возможность самому быть человеком.
\vs p007 5:3 Чисто личностная природа Вечного Сына неспособна к фрагментации. Вечный Сын осуществляет служение как духовное влияние или как личность, и никак иначе. Сын не может стать частью опыта создания в том смысле, в каком ею становится Отец\hyp{}Настройщик, но Вечный Сын компенсирует эти ограничения посредством пришествия. Что опыт фрагментированных сущностей значит для Отца Всего Сущего, то опыт воплощения Райских Сынов значит для Вечного Сына.
\vs p007 5:4 Вечный Сын не приходит к смертному человеку как божественная воля, как Настройщик Мысли, пребывающий в человеческом разуме, но Вечный Сын пришел к смертному человеку на Урантии, когда божественная \bibemph{личность} его Сына, Михаила из Небадона, воплотилась в человеческое естество Иисуса из Назарета. Чтобы разделить опыт сотворенных личностей, Райские Сыны Бога должны принять саму природу таких созданий и воплотить свои божественные личности в сами эти создания. Воплощение, секрет Сынограда, является способом Сынов избежать в ином случае всеохватывающих оков абсолютизма личности.
\vs p007 5:5 \pc Давным\hyp{}давно Вечный Сын одарил своим собственным пришествием каждый из контуров центрального творения для просвещения и продвижения всех обитателей и пилигримов Хавоны, включая восходящих путешественников времени. Ни в одном из этих семи пришествий он не действовал ни как восходящий, ни как житель Хавоны. Он существовал, как он есть. Его опыт был уникален; это не был ни тот опыт, который связан \bibemph{с} человеком или другим пилигримом, ни \bibemph{подобный} их опыту, но в некотором отношении ассоциативный в сверхличностном смысле.
\vs p007 5:6 Он не отдыхал в том месте, между внутренним контуром Хавоны и берегами Рая. Для него, существа абсолютного, невозможно отключить сознание личности, ибо в нем сходятся все линии духовной гравитации. И во времена этих пришествий центральная Райская обитель духовного сияния была неомраченной, а обладание Сына вселенской духовной гравитацией --- неослабным.
\vs p007 5:7 \pc Пришествия Вечного Сына в Хавону не вмещаются в рамки человеческого воображения; они --- трансцендентальны. И тогда, и впоследствии он приумножил опыт всей Хавоны, но мы не знаем, усугубил ли он предполагаемую опытную способность своей экзистенциальной природы. Это есть тайна пришествия Райских Сынов. Однако мы полагаем, что бы ни приобрел Вечный Сын во время этих миссий пришествия, он все это сохраняет в себе; но мы не знаем, что это такое.
\vs p007 5:8 \pc Как бы ни было нам трудно постигать пришествие Второго Лица Божества, мы все\hyp{}таки постигаем пришествие в Хавону Сына Вечного Сына, который буквально прошел через контуры центральной вселенной и актуально разделил тот опыт, который составляет приготовление восходящего к достижению Божества. Это был изначальный Михаил, первородный Сын\hyp{}Творец, и он прошел через опыт жизни восходящих пилигримов от контура к контуру, лично преодолевая с ними ступень каждого контура во дни Грандфанды, первого из всех смертных, достигших Хавоны.
\vs p007 5:9 Что бы еще ни открыл этот изначальный Михаил, он сделал реальным для созданий Хавоны необыкновенное пришествие Изначального Сына\hyp{}Матери. Настолько реальным, что впоследствии всегда каждый пилигрим во времени, который прилагает усилия, чтобы осуществить восхождение по контурам Хавоны, ободряется и укрепляется сознанием того, что Вечный Сын Бога семь раз отрекался от мощи и славы Рая, чтобы принять участие в опыте пилигримов пространства\hyp{}времени на семи контурах последовательного достижения Хавоны.
\vs p007 5:10 \pc Вечный Сын --- образец вдохновения для всех Сынов Бога в их службе пришествия во всех вселенных пространства и времени. Равноправные Сыны\hyp{}Творцы и их сподвижники Сыны\hyp{}Повелители вместе с другими нераскрытыми чинами сыновства --- все разделяют эту замечательную готовность осуществить свое пришествие к разнообразным чинам живых созданий и в облике созданий. Следовательно, очевидно (по духу и вследствие родственности натуры, а также по происхождению), что в пришествии каждого Сына Бога в миры, существующие в пространстве, и посредством этих пришествий Вечный Сын сам нисходит к разумным созданиям вселенных, обладающим волей.
\vs p007 5:11 По духу и по природе, если не по всем атрибутам, каждый Райский Сын является совершенным изображением Изначального Сына. И абсолютно верно, что любой, кто видел Райского Сына, видел Вечного Сына Бога.
\usection{6. Райские Сыны Бога}
\vs p007 6:1 Недостаток знания о многочисленных Сынах Бога является источником большой путаницы на Урантии. И это неведение продолжает существовать перед лицом таких утверждений, как запись совещания этих духовных личностей: «Когда Сыны Бога возвещают радость, все Утренние Звезды поют хором». Каждое тысячелетие по времени, стандартному для сектора, различные чины божественных Сынов собираются на свои периодические совещания.
\vs p007 6:2 Вечный Сын есть личностный источник восхитительных атрибутов милосердия и служения, которые так полно характеризуют все чины нисходящих Сынов Бога, когда они функционируют по всему творению. Всю божественную природу, если не всю бесконечность атрибутов, Вечный Сын неизменно передает Райским Сынам, которые отправляются из вечного Острова открывать его божественный характер для вселенной вселенных.
\vs p007 6:3 \pc Изначальный и Вечный Сын является потомком\hyp{}лицом «первой» завершенной и бесконечной мысли Отца Всего Сущего. Каждый раз когда Отец Всего Сущего и Вечный Сын совместно выдвигают новую, оригинальную, идентичную, уникальную и абсолютную мысль, в этот же самый момент эта творческая идея совершенно и окончательно персонализируется в существо и личность нового и оригинального \bibemph{Сына\hyp{}Творца.} По духовной природе, божественной мудрости и творческой мощи эти Сыны\hyp{}Творцы потенциально равны с Богом Отцом и с Богом Сыном.
\vs p007 6:4 Сыны\hyp{}Творцы выходят из Рая во вселенные, развивающиеся во времени, и во взаимодействии с контролирующими и творческими силами Третьего Источника и Центра завершают организацию локальных вселенных, для которых характерна прогрессивная эволюция. Эти Сыны не придаются центральному и вселенскому контролю материи, разума и духа и они с ним не связаны. Поэтому они ограничены в своих творческих актах предсуществованием, старшинством и первенством Первоисточника и Центра и ему равноправных Абсолютов. Эти Сыны способны управлять только тем, чему они сами дают существование. Абсолютное управление присуще старшинству существования и неотделимо от вечности присутствия. Отец остается главным во вселенных.
\vs p007 6:5 \pc Почти так же, как Сыны\hyp{}Творцы персонализированы Отцом и Сыном, \bibemph{Сыны\hyp{}Повелители} персонализированы Сыном и Духом. Они являются теми Сынами, которые своим опытом воплощения в создания зарабатывают право служить в качестве судей, определяющих продолжение существования в посмертии в творениях времени и пространства.
\vs p007 6:6 \pc Отец, Сын и Дух также объединяются, чтобы персонализировать разносторонних \bibemph{Сынов\hyp{}Учителей Троицы,} которые странствуют по великой вселенной как возвышенные учителя всех личностей, человеческих и божественных. Существуют также многочисленные другие чины Райского сыновства, о которых смертным Урантии не сообщалось.
\vs p007 6:7 \pc Между Изначальным Сыном\hyp{}Матерью и этими множествами Райских Сынов, разбросанных по всему творению, существует прямой и исключительный канал связи, канал, чья функция присуща качеству духовного родства, которое объединяет их узами почти абсолютного духовного союза. Такой межсыновний контур совершенно отличается от вселенского контура духовной гравитации, который также сосредоточен в лице Второго Источника и Центра. Все Сыны Бога, которые происходят от лиц Райских Божеств, находятся в прямой и постоянной связи с Вечным Сыном\hyp{}Матерью. И такая связь мгновенна, она не зависит от времени, хотя иногда обусловлена пространством.
\vs p007 6:8 Вечный Сын не только обладает во все времена совершенным знанием о статусе, мыслях и многообразной деятельности всех чинов Райского сыновства, но он также обладает во все времена совершенством знания обо всем, что касается духовных ценностей, которые существуют в сердцах всех созданий в первичном центральном творении вечности и во вторичных творениях времени равноправных Сынов\hyp{}Творцов.
\usection{7. Верховное откровение Отца}
\vs p007 7:1 Вечный Сын есть полное, исключительное, вселенское и окончательное откровение духа и личности Отца Всего Сущего. Все знание и вся информация относительно Отца должна исходить от Вечного Сына и его Райских Сынов. Вечный Сын --- от вечности, и он тот, кто есть одно с Отцом полностью и без духовного ограничения. В божественной личности они равноправны, в духовной природе они равны, в божественности они идентичны.
\vs p007 7:2 Характер Бога не может быть улучшен в лице Сына, ибо божественный Отец бесконечно совершенен, но этот характер и личность усилены в результате освобождения от безличностного и недуховного для откровения созданным существам. Первоисточник и Центр много больше, чем личность, но все духовные качества отцовской личности Первоисточника и Центра духовно присутствуют в абсолютной личности Вечного Сына.
\vs p007 7:3 Изначальный Сын и его Сыны осуществляют вселенское откровение духовной и личностной природы Отца всему творению. В центральной вселенной, сверхвселенных, локальных вселенных или на обитаемых планетах Райский Сын есть тот, кто открывает Отца Всего Сущего людям и ангелам. Вечный Сын и его Сыны открывают творениям доступ к Отцу Всего Сущего. И даже мы, имеющие более высокое происхождение, понимаем Отца гораздо более полно, когда мы изучаем откровение его характера и личности в Вечном Сыне и в Сынах Вечного Сына.
\vs p007 7:4 Отец приходит к вам как личность только благодаря божественным Сынам Вечного Сына. И вы достигаете Бога именно таким живым путем; вы восходите к Отцу под водительством этой группы божественных Сынов. И это истинно несмотря на то, что ваша личность является непосредственным даром Отца Всего Сущего.
\vs p007 7:5 \pc Рассматривая всю эту широко распространенную деятельность широко раскинувшегося управления Вечного Сына, не забывайте, что Сын является личностью так же истинно и актуально, как Отец. Действительно, для существа, принадлежащего когда\hyp{}то человеческому чину, Вечный Сын будет более доступным, чем Отец Всего Сущего. В продвижении пилигримов, живущих во времени, через контуры Хавоны вы обретете способность достичь Сына задолго до того, как вы будете подготовлены к тому, чтобы распознать Отца.
\vs p007 7:6 Вы больше поймете относительно характера и милосердной природы Вечного Сына милосердия, если сосредоточитесь в размышлении об откровении этих божественных атрибутов, которые обнаруживаются в любовном служении вашего собственного Сына\hyp{}Творца --- когда\hyp{}то Сына Человеческого на земле, а ныне возвышенного владыки вашей локальной вселенной --- Сына Человеческого и Сына Божьего.
\vsetoff
\vs p007 7:7 [Выражено в словах Божественным Советником, призванным сформулировать это заявление, запечатлевшее Вечного Сына Рая.]
