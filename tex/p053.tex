\upaper{53}{Бунт Люцифера}
\author{Мановандет Мелхиседек}
\vs p053 0:1 Люцифер был блестящим первичным Сыном\hyp{}Ланонандеком Небадона. Он прошел служение во многих системах, был высоким советником своей группы, был отличен за мудрость, проницательность и результативность. Люцифер был номером 37 своего чина и, когда он назначался Мелхиседеками, был назван среди ста самых способных и блестящих личностей из более чем семисот тысяч себе подобных. Имея такое великолепное начало, он встал на путь зла и заблуждения, впал в грех и ныне числится одним из трех Владык Систем в Небадоне, которые поддались побуждениям эгоизма и оказались в плену у софистики ложной личной свободы --- они отвергли вселенскую верность и пренебрегли братскими обязательствами, обнаружив свою слепоту по отношению к космическим связям.
\vs p053 0:2 Во вселенной Небадона --- владении Христа\hyp{}Михаила --- десять тысяч систем обитаемых миров. За всю историю Сынов\hyp{}Ланонандеков во всей их работе на всех этих тысячах систем и во вселенских центрах только трое Владык Систем были когда\hyp{}либо признаны виновными в оскорблении власти Сына\hyp{}Творца.
\usection{1. Вожди бунта}
\vs p053 1:1 Люцифер не был восходящим созданием, он был сотворенным Сыном локальной вселенной, и о нем было сказано: «Со дня твоего сотворения ты был совершенным во всех отношениях, пока неправедность не обнаружилась в тебе». Множество раз он был в совете с Всевышними Эдентии. И Люцифер правил «на святой горе Бога», административной вершине Иерусема, поскольку был главным распорядителем огромной системы, состоящей из 607 обитаемых миров.
\vs p053 1:2 Люцифер был великолепным существом, блестящей личностью; он стоял следующим после Всевышних Отцов созвездий в ряду вселенских властей. Несмотря на преступление Люцифера, до пришествия Михаила на Урантию находящиеся в иерархии на более низкой ступени, воздерживались выказывать ему неуважение и презрение. Даже архангел Михаил во время воскресения Моисея «не высказывал против него осуждения, но просто сказал: „Судья воздаст тебе должное“». Выносить решение в таких делах надлежит Древним Дней --- правителям сверхвселенной.
\vs p053 1:3 Ныне Люцифер --- падший и свергнутый Владыка Сатании. Самомнение в высшей степени губительно, даже для высокопоставленных личностей небесного мира. О Люцифере было сказано: «Твое сердце взыграло из\hyp{}за твоей красоты; ты погубил свою мудрость из\hyp{}за своего блеска». Ваш древний пророк видел его отчаянное состояние, когда писал: «Ныне ты пал с небес, о Люцифер, сын утра! Ныне ты свергнут, ты, кто осмелился привести миры в беспорядок!»
\vs p053 1:4 На Урантии о Люцифере очень мало было известно из\hyp{}за того, что он назначил вести свое дело на вашей планете своего первого помощника --- Сатану. Сатана был членом той же самой первичной группы Ланонандеков, но он никогда не был Владыкой Системы; он полностью присоединился к восстанию Люцифера. «Дьявол» --- это никто иной как Калигастия, свергнутый Планетарный Принц и Сын вторичного чина Ланонандеков. В то время, когда Михаил был во плоти на Урантии, Люцифер, Сатана и Калигастия объединились, чтобы обречь его миссию пришествия на неудачу. Но они потерпели сокрушительное поражение.
\vs p053 1:5 Аваддон был главой штата Калигастии. Он примкнул к бунту вслед за своим руководителем и с тех пор является главным распорядителем урантийских бунтовщиков. Вельзевул был вождем вероломных срединных созданий, которые присоединились к войску предателя Калигастии.
\vs p053 1:6 \pc Символом всех этих сил зла, в конце концов, стал дракон. К моменту триумфа Михаила «Гавриил снизошел из Спасограда и навек заключил в узы дракона (всех вождей бунта)». О бунтовщиках\hyp{}серафимах Иерусема написано: «И ангелов, которые не хранили первое свое состояние, но покинули свои обиталища, он оставил в надежных оковах тьмы до великого судного дня».
\usection{2. Причины бунта}
\vs p053 2:1 Люцифер и его первый помощник Сатана правили в Иерусеме свыше пятисот тысяч лет, пока не стали настраивать себя в сердце своем против Отца Всего Сущего и его тогдашнего Сына\hyp{}наместника --- Михаила.
\vs p053 2:2 В системе Сатании не было никаких специфических или особых условий, которые провоцировали бы или способствовали бунту. Наше мнение таково, что идея бунта возникла и сформировалась в уме Люцифера и что он мог бы спровоцировать такой бунт, где он ни находился. Люцифер сначала сообщил о своих планах Сатане, но, чтобы развратить разум этого талантливого и блестящего помощника, потребовалось несколько месяцев. Однако однажды признав бунтовщические теории, тот стал дерзким и убежденным сторонником «самоутверждения и свободы».
\vs p053 2:3 \pc Никто никогда не склонял Люцифера к бунту. Идея самоутверждения в противоположность воле Михаила и планам Отца Всего Сущего в том виде, как они были воплощены в Михаиле, возникла у него самостоятельно. Его отношения с Сыном\hyp{}Творцом были всегда близкими и сердечными. До тех пор, пока Люцифер не начал превозносить свой разум, он никогда открыто не выражал свое неудовлетворение вселенским управлением. Несмотря на его молчание, в течение более ста лет стандартного времени Объединяющие Дней в Спасограде сообщали на Уверсу, используя феномен отражательности, что нет спокойствия в мыслях Люцифера. Эта информация также передавалась Сыну\hyp{}Творцу и Отцам Созвездий Норлатиадека.
\vs p053 2:4 В течение всего этого периода Люцифер становился все более критически настроенным по отношению ко всей системе вселенского управления, но при этом всегда заявлял о своей искренней преданности Верховным Правителям. Его откровенная нелояльность впервые выразилась в момент визита Гавриила в Иерусем, всего за несколько дней до открытого официального объявления Декларации Свободы Люцифера. Гавриил был столь глубоко потрясен очевидностью надвигающегося восстания, что он отправился прямо в Эдентию, чтобы обсудить с Отцами Созвездий те меры, которые необходимо принять в случае открытого бунта.
\vs p053 2:5 Очень трудно указать точную причину или причины, кульминацией которых, в конце концов стал бунт Люцифера. Мы уверены лишь в одном, а именно: каковы бы ни были первопричины, они родились в уме Люцифера. Это, должно быть, была его собственная гордыня, которая и привела его к самообману, так что Люцифер за короткое время действительно убедил себя в том, что его бунт на самом деле пойдет на пользу системе, если не вселенной. Со временем он разочаровался в своих планах, однако он зашел слишком далеко в своей первоначальной и порождающей зло гордыне, чтобы позволить себе остановиться. В какой\hyp{}то момент этого процесса он стал неискренним и зло развилось в преднамеренный и своевольный грех. Что подтверждается последующим поведением этого блестящего распорядителя. Ему давно была предоставлена возможность раскаяться, но лишь некоторые из его подчиненных все же приняли предложенную милость. Верный Дней Эдентии по просьбе Отцов Созвездия лично представил план Михаила по спасению этих позорных бунтовщиков, но милосердие Сына\hyp{}Творца всегда отвергалось и отвергалось со все возрастающими презрением и надменностью.
\usection{3. Манифест Люцифера}
\vs p053 3:1 Каковы бы ни были первые ростки смуты в сердцах Люцифера и Сатаны, окончательный взрыв принял форму Декларации Свободы Люцифера. Взгляды бунтовщиков были изложены в трех пунктах:
\vs p053 3:2 \ublistelem{1.}\bibnobreakspace \bibemph{Реальность Отца Всего Сущего.} Люцифер утверждал что Отец Всего Сущего реально не существует, что физическая гравитация и пространство\hyp{}энергия присущи вселенной и что Отец --- это миф, выдуманный Райскими Сынами, чтобы получить возможность продолжать править вселенными от имени Отца. Он отрицал, что личность есть дар Отца Всего Сущего. Он даже намекал на то, что финалиты состоят в сговоре с Райскими Сынами, чтобы ввести в обман все творение, потому что они никогда не давали совершенно четкого представления о той реальной личности Отца, которая видима в Раю. Он считал поклонение невежеством. Обвинение было огульным, ужасным и богохульным. Именно этот замаскированный выпад против финалитов, несомненно, заставил восходящих граждан Иерусема дать решительный отпор всем предложениям бунтовщиков и стоически им противодействовать.
\vs p053 3:3 \ublistelem{2.}\bibnobreakspace \bibemph{Вселенское правительство Сына\hyp{}Творца --- Михаила.} Люцифер заявлял, что локальные системы должны быть автономны. Он протестовал против права Михаила, Сына\hyp{}Творца, владычествовать в Небадоне от имени гипотетического Отца Всего Сущего и требовать, чтобы все личности присягнули на верность этому невидимому Отцу. Он утверждал, что весь замысел богопочитания есть не что иное как хитроумная система возвеличивания Райских Сынов. Он готов был признать Михаила как своего отца\hyp{}Творца, но не как своего Бога и законного правителя.
\vs p053 3:4 Наиболее ожесточенно он нападал на право Древних Дней --- «иноземных властелинов» --- вмешиваться в дела локальных систем и вселенных. Этих правителей он объявил тиранами и узурпаторами. Он призывал своих последователей верить в то, что никто из этих правителей не смог бы ничего предпринять, чтобы помешать полному самоуправлению, если бы только у людей и ангелов нашлось мужество заявить о себе и твердо отстаивать свои права.
\vs p053 3:5 Он декларировал, что исполнители Древних Дней могли бы быть лишены права действовать в локальных системах, если бы только исконные существа боролись за свою независимость. Он утверждал, что бессмертие присуще личностям системы, что воскресение является естественным и автоматическим и что все существа жили бы вечно, если бы не произвольные и несправедливые действия исполнителей Древних Дней.
\vs p053 3:6 \ublistelem{3.}\bibnobreakspace \bibemph{Нападки на вселенский план обучения восходящих смертных.} Люцифер утверждал, что чересчур много времени и энергии потрачено на программу столь основательного обучения восходящих смертных принципам вселенского управления, принципам, которые, как он считает, являются неэтичными и ошибочными. Он протестовал против рассчитанной на века программы подготовки смертных пространства к некоему неизвестному предназначению и указывал на присутствие отряда финалитов в Иерусеме как на доказательство того, что эти смертные потратили века на подготовку к некоторому предназначению, которое, на поверку, есть чистая фикция. С насмешкой он подчеркивал, что ничего не может быть более славного для предназначения финалитов, как вернуться в скромные миры, подобные тем, где они родились. Он намекал, что их испортили излишняя дисциплина и продолжительное обучение, и что в действительности они --- предатели своих смертных собратьев, потому что ныне содействуют плану порабощения всего творения с помощью выдумок о мифическом вечном предназначении восходящих смертных. Он отстаивал ту точку зрения, что восходящие должны пользоваться свободой личного самоутверждения. Он оспаривал и осуждал весь план смертного восхождения, который выдвигался Райскими Сынами Бога и поддерживался Бесконечным Духом.
\vs p053 3:7 \pc И именно с такой Декларации Свободы Люцифер начал свою оргию тьмы и смерти.
\usection{4. Начало бунта}
\vs p053 4:1 Манифест Люцифера был оглашен на ежегодном конклаве Сатании, происходившем на стеклянном море, в присутствии собравшихся сонмов Иерусема в последний день года примерно двести тысяч лет назад по урантийскому времени. Сатана провозгласил, что богопочитание может оказываться вселенским силам --- физическим, интеллектуальным и духовным, --- но верность может выражаться только по отношению к реальному сегодняшнему правителю Люциферу --- «другу людей и ангелов» и «Богу свободы».
\vs p053 4:2 Самоутверждение было боевым девизом бунта Люцифера. Одним из его главных аргументов было то, что, если самоуправление хорошо и справедливо для Мелхиседеков и других групп, оно так же хорошо и для всех чинов интеллекта. Он был дерзок и напорист в своем отстаивании «равенства разума» и «братства интеллекта». Он утверждал, что власть всех правительств должна быть ограничена рамками локальных планет и их добровольного конфедеративного объединения в локальные системы. Все другие виды руководства он отвергал. Он обещал Планетарным Принцам, что они будут править мирами как верховные распорядители. Он осуждал размещение законодательных органов в центрах созвездий и ведение судебных дел в столице вселенной. Он заявлял, что вся такая деятельность правительства должна быть сосредоточена в столицах систем, и пришел к тому, что учредил свою собственное законодательное собрание и организовал свои собственные суды под юрисдикцией Сатаны. И приказал, чтобы принцы на поддержавших его мирах делали то же самое.
\vs p053 4:3 Весь кабинет администрации Люцифера в полном составе перешел на его сторону, и они были приведены к присяге как служащие администрации нового главы «освобожденных миров и систем».
\vs p053 4:4 \pc Хотя до этого в Небадоне и было два бунта, но они произошли на далеких созвездиях. Люцифер считал, что эти восстания потерпели неудачу, потому что большинство не поддержало своих вождей. Он заявлял, что «правит большинство», что «разум непогрешим». Свобода, предоставленная ему вселенскими правителями, явно подтверждала многие из его нечестивых заявлений. Он игнорировал всех своих начальников, но те казалось бы не обращали внимания на его действия. Ему была дана полная возможность без помех и препятствий осуществлять свой план обольщения.
\vs p053 4:5 \pc На все милосердные отсрочки правосудия Люцифер указывал как на свидетельство неспособности правительства Райских Сынов подавить бунт. Он открыто высказывал неповиновение и высокомерно бросал вызов Михаилу, Иммануилу и Древним Дней, указывая на факт отсутствия противодействий как на подтверждение бессилия правительств вселенной и сверхвселенной.
\vs p053 4:6 Гавриил лично присутствовал во время всех этих изменнических событий и лишь заявил, что он в должное время выступит от лица Михаила и что все существа должны оставаться свободными и самостоятельными в своем выборе; что «правительство Сынов от лица Отца желает только такой верности, которая является добровольной, чистосердечной и может противостоять софистике».
\vs p053 4:7 \pc Люциферу было позволено окончательно установить и тщательно организовать свое бунтовщическое правление прежде, чем Гавриил предпринял хоть какую\hyp{}то попытку оспорить права раскольников или противодействовать пропаганде бунтовщиков. Но Отцы Созвездий немедленно ограничили действия этих предателей границами Сатании. Тем не менее этот период бездействия был временем великого испытания и проверки для верных существ всей Сатании. На несколько лет все погрузилось в хаос, и в мирах\hyp{}обителях было великое смятение.
\usection{5. Природа конфликта}
\vs p053 5:1 Уже в начале бунта Сатаны Михаил держал совет со своим Райским братом Иммануилом. После этого важного совещания Михаил объявил, что он будет следовать той же политике, которой он придерживался при подобных переворотах в прошлом, --- позиции невмешательства.
\vs p053 5:2 \pc Во время этого бунта и тех двух, которые ему предшествовали, во вселенной Небадона не существовало абсолютной и личной власти владыки. Михаил правил по божественному праву как наместник Отца Всего Сущего, но не в силу своего собственного личного права. Он не завершил свой путь пришествия; он еще не был облечен «всей властью на земле и на небе».
\vs p053 5:3 С начала восстания до дня своего воцарения как владыки\hyp{}правителя Небадона Михаил никогда не мешал силам бунтовщиков Люцифера; им было позволено идти своим путем в течение почти двухсот тысяч лет урантийского времени. В настоящее время Христос\hyp{}Михаил обладает достаточной силой и властью, чтобы быстро и даже немедленно подавить такие вспышки предательства, но мы сомневаемся, чтобы эта власть владыки заставила бы его действовать по\hyp{}другому, случись еще один такой переворот.
\vs p053 5:4 \pc Поскольку Михаил решил остаться в стороне во время борьбы, вызванной бунтом Люцифера, Гавриил созвал свой личный штат на Эдентии и, посоветовавшись со Всевышними, решил взять на себя командование верным воинством Сатании. Михаил остался в Спасограде, а Гавриил отправился в Иерусем и, расположившись на сфере, посвященной Отцу --- тому самому Отцу Всего Сущего, в личности которого Люцифер и Сатана сомневались, --- в присутствии собравшегося воинства верных личностей развернул знамя Михаила, материальную эмблему правительства Троицы, правительства всего творения --- три лазурно\hyp{}голубых концентрических круга на белом фоне.
\vs p053 5:5 Эмблемой Люцифера было белое знамя с одним красным кругом, в центре которого располагался сплошной черный круг.
\vs p053 5:6 «В небе происходила война; военачальник Михаила и его ангелы сражались против дракона (Люцифера, Сатаны и принцев\hyp{}отступников); и дракон, и его взбунтовавшиеся ангелы сражались, но не достигли цели». Эта «война в небе» не была физическим сражением, как обыкновенно это понимается на Урантии. В первые дни бунта Люцифер непрерывно разглагольствовал в планетарном амфитеатре. Гавриил непрестанно разоблачал софизмы бунтовщиков из своего центра, находящегося в непосредственной близости. Различные личности, присутствующие на сфере, которые еще не определили своей позиции, могли ходить и туда, и обратно на оба эти выступления до тех пор, пока не принимали окончательное решение.
\vs p053 5:7 Но эта война была очень страшной и действительно реальной. Хотя и не было проявлений жестокости, столь характерных для физических войн в неразвитых мирах, этот конфликт был гораздо более беспощадным: в материальной битве подвергается смертельной опасности материальная жизнь, а война на небе ведется за жизнь вечную.
\usection{6. Верный серафим\hyp{}военачальник}
\vs p053 6:1 Множество благородных и вдохновляющих поступков, полных преданности и верности, было совершено многочисленными личностями в промежутке между началом военных действий и прибытием нового правителя системы и его штата. Но наиболее захватывающим среди этих отважных подвигов было мужественное поведение Манотии --- второго по старшинству начальника серафимов центра Сатании.
\vs p053 6:2 В начале бунта на Иерусеме глава сонма серафимов перешел на сторону Люцифера. Это, без сомнения, и объясняет, почему такое большое число членов четвертого чина, серафимов\hyp{}администраторов системы, сбились с пути. Вождь серафимов был ослеплен блеском личности Люцифера; низшие чины небесных существ --- заворожены его обольстительными приемами. Они просто не могли поверить в возможность того, что такая поразительная личность может идти по ложному пути.
\vs p053 6:3 \pc Вскоре после этого, описывая треволнения, связанные с началом бунта Люцифера, Манотия сказал: «Но самым незабываемым моментом для меня было достопамятное событие, произошедшее во время бунта Люцифера, когда, будучи вторым командиром\hyp{}серафимом, я отказался принимать участие в готовящемся нанесении оскорбления Михаилу; и могущественные бунтовщики пытались меня уничтожить с помощью сил связи, которые они подготовили. В Иерусеме произошел ужасный переворот, но ни один верный серафим не пострадал.
\vs p053 6:4 После предательства моего непосредственного начальника обязанности командира ангельских сонмов в Иерусеме перешли ко мне как занимающему должность управителя делами серафимов системы, пришедшими в беспорядок. Меня морально поддерживали Мелхиседеки, мне помогало большинство Материальных Сынов, и хотя меня покинула огромная группа из моего собственного чина, но меня замечательно поддерживали восходящие смертные Иерусема.
\vs p053 6:5 Будучи автоматически выведены из контуров созвездий в результате отпадения Люцифера, мы оказались зависимы от верности нашего разведывательного отряда, который из соседней системы Рантулии передал в Эдентию призывы о помощи; и мы увидели, что царство порядка, разум верности и дух истины всегда побеждают бунт --- самоутверждение и так называемую личную свободу; нам удалось продержаться до прибытия нового Владыки Системы, преемника Люцифера, достойного занимать этот пост. И сразу же после этого я был назначен в отряд Мелхиседеков\hyp{}исполнителей Урантии, приняв под свою юрисдикцию верные чины серафимов в мире изменника Калигастии, объявившего свою сферу частью новой планируемой системы „освобожденных миров и эмансипированных личностей“, которая провозглашалась в постыдной Декларации, выпущенной Люцифером в его обращении к „любящим свободу, свободомыслящим и дальновидным разумным существам миров Сатании, которыми плохо правят и управляют“».
\vs p053 6:6 \pc Этот ангел все еще находится на службе на Урантии, функционируя как сподвижник главы серафимов.
\usection{7. История бунта}
\vs p053 7:1 Бунт Люцифера охватил всю систему. Тридцать семь раскольников\hyp{}Планетарных Принцев в значительной степени склонили администрацию своих миров на сторону архибунтовщика. Только на Паноптии Планетарному Принцу не удалось повести за собой свой народ. В этом мире --- под руководством Мелхиседеков --- народ выступил в поддержку Михаила. Элланора, молодая женщина этого смертного мира взяла на себя руководство человеческими расами, и ни одна душа в этом раздираемом противоречиями мире не стала под знамя Люцифера. И с тех пор эти верные Паноптийцы служат в седьмом переходном мире Иерусема сторожами и строителями на сфере Отца и окружающих ее семи мирах заключения. Паноптийцы не только действуют как настоящие опекуны этих миров, но и выполняют личные поручения Михаила по украшению этих сфер для какого\hyp{}то пока неизвестного использования в будущем. Они исполняют эту работу, когда задерживаются на пути в Эдентию.
\vs p053 7:2 В течение всего этого периода Калигастия отстаивал на Урантии дело Люцифера. Мелхиседеки действенно противостояли Планетарному Принцу\hyp{}отступнику, но софизмы необузданной свободы и заблуждения самоутверждения использовали каждую возможность для обмана первобытных народов молодого и неразвитого мира.
\vs p053 7:3 Вся пропаганда раскола должна была проводиться личными усилиями, поскольку служба возвещений и все другие пути межпланетной связи были перекрыты действиями руководителей контуров системы. С момента реального начала восстания вся система Сатании были изолирована от контуров созвездий и вселенных. В течение этого времени все передаваемые и принимаемые сообщения посылались серафимами\hyp{}посредниками и Одиночными Вестниками. Контуры, ведущие к падшим мирам, были также отрезаны, так что Люцифер не мог использовать этот путь для развития своего нечестивого плана. И эти контуры не будут восстановлены до тех пор, пока архибунтовщик находится в пределах Сатании.
\vs p053 7:4 Это было восстание Ланонандеков. Более высокие чины сыновства локальных вселенных не примкнули к расколу Люцифера, хотя бунт вероломных Принцев в какой\hyp{}то степени повлиял на незначительное число Носителей Жизни, размещенных на взбунтовавшихся планетах. Никто из Тринитизированных Сынов не сбился с пути. Мелхиседеки, архангелы и Блестящие Вечерние Звезды --- все остались верными Михаилу и вместе с Гавриилом героически сражались за волю Отца и правление Сына.
\vs p053 7:5 Ни одно существо Райского происхождения не встало на путь измены. Вместе с Одиночными Вестниками они заняли центр в мире Духа и оставались под руководством Верных Дней Эдентии. Не стал отступником и никто из примирителей, и ни один из Небесных Протоколистов не сбился с пути. Но большие потери понесли Моронтийные Компаньоны и Учителя Миров\hyp{}Обителей.
\vs p053 7:6 Из верховного чина серафимов не пропал ни один ангел, но значительная группа следующего, чина --- высшего, была обманута и попала в ловушку. Также и незначительная часть третьего чина --- чина серафимов\hyp{}руководителей --- пошла по ложному пути. Но самый сильный ущерб был нанесен четвертой группе, группе серафимов\hyp{}администраторов, тех серафимов, которые обычно назначаются на должности в столицах систем. Манотия спас почти две трети из них, но чуть более одной трети все\hyp{}таки последовало за своим главой по пути бунтовщиков. Одна треть всех иерусемских херувимов, приданных ангелам\hyp{}администраторам, пропала вместе со своими серафимами\hyp{}изменниками.
\vs p053 7:7 Из планетарных ангельских помощников, которые приданы Материальным Сынам, около одной трети было обмануто и почти десять процентов служителей перехода попали в ловушку. Иоанн символически видел это когда писал о великом красном драконе: «Хвост его увлек с неба третью часть звезд и поверг их на землю».
\vs p053 7:8 Самый большой урон понесли ангельские чины, однако и большинство низших чинов интеллекта стали предателями. Из 681\,217 Материальных Сынов, погибших в Сатании, девяносто пять процентов составили потери в бунте Люцифера. Большое число срединных созданий было потеряно на тех планетах, Планетарные Принцы которых примкнули к делу Люцифера.
\vs p053 7:9 \pc Во многих отношениях из всех подобных событий в Небадоне этот бунт был самым широко распространившимся и гибельным. В это восстание было вовлечено больше личностей, чем в два других. И вечный позор эмиссарам Люцифера и Сатаны, которые не пощадили даже воспитательных яслей на планете культуры финалитов, пытаясь развратить эти развивающиеся умы, из милосердия спасенные с эволюционных миров.
\vs p053 7:10 \pc Восходящие смертные были уязвимы, но они противостояли софизмам бунта лучше, чем низшие духи. Хотя многие в низших мирах\hyp{}обителях, те, кто не достиг окончательного слияния со своими Настройщиками, пали, надо отдать должное мудрости плана восхождения, ибо ни один обладатель гражданства восходящих Сатании, живущий в Иерусеме, не принял участия в бунте Люцифера.
\vs p053 7:11 Час за часом и день за днем станции вещания всего Небадона были переполнены озабоченными наблюдателями всех мыслимых классов небесного интеллекта, которые пристально изучали бюллетени о бунте Сатании и радовались, когда сообщения непрерывно информировали о неизменной верности восходящих смертных, которые под руководством своих Мелхиседеков успешно противостояли объединенным и длительным усилиям вкрадчивых сил зла, столь быстро собравшихся под знамена раскола и греха.
\vs p053 7:12 Прошло более двух лет системного времени от начала «войны на небе» до введения в должность преемника Люцифера. И наконец пришел новый Владыка, высадившись со своим штатом на стеклянном море. Я был в резерве, мобилизованном Гавриилом на Эдентии, и хорошо помню первое послание Ланафоржа Отцу Созвездия Норлатиадека. Оно гласило: «Ни один гражданин Иерусема не пропал. Каждый восходящий смертный пережил огненный опыт и вышел из решающего испытания ликующим и всецело победоносным». И далее --- в Спасоград, Уверсу и Рай пошло это послание уверенности в том, что опыт продолжения существования смертных путем восхождения является величайшей гарантией безопасности против бунта и надежнейшей защитой от греха. Эта благородная иерусемская группа верных смертных насчитывала ровно 187\,432\,811 человек.
\vs p053 7:13 \pc С прибытием Ланафоржа архибунтовщики были свергнуты и лишены всех своих властных полномочий, хотя им и было дозволено свободно передвигаться по Иерусему, моронтийным сферам и даже --- по отдельным обитаемым мирам. Они, однако, продолжали обманывать и соблазнять, стремясь посеять смуту в разумах людей и ангелов и ввести их в заблуждение. Но что касается их работы на административной вершине Иерусема, то там «для них места больше нет».
\vs p053 7:14 \pc Хотя Люцифер и был лишен всей административной власти в Сатании, тогда еще не существовало никакого суда или власти локальной вселенной, которая могла бы заключить под стражу или уничтожить нечестивого бунтовщика; в то время Михаил еще не был владыкой\hyp{}правителем. Древние Дней поддержали Отцов Созвездий в захвате правительства системы, но они никогда не передавали никаких дальнейших указаний сверху в отношении многих апелляций, все еще ожидающих разрешения, по вопросам как существующего статуса, так и будущего Люцифера, Сатаны и их сподвижников.
\vs p053 7:15 Таким образом, этим архибунтовщикам было позволено скитаться по всей вселенной и искать пути дальнейшего распространения своих доктрин недовольства и самоутверждения. Но в течение почти двухсот тысяч урантийских лет они больше не смогли обмануть ни один мир. Ни один мир Сатании не был потерян со времен падения тридцати семи миров, устояли даже те молодые миры, которые были заселены после начала бунта.
\usection{8. Сын Человеческий на Урантии}
\vs p053 8:1 Люцифер и Сатана свободно блуждали по системе Сатании до завершения миссии пришествия Михаила на Урантию. Последний раз они были в вашем мире во время их совместного нападения на Сына Человеческого.
\vs p053 8:2 В прошлом, когда собирались Планетарные Принцы, «Сыны Бога», «также приходил и Сатана», заявляя, что он представляет все изолированные миры падших Планетарных Принцев. Но со времени окончательного пришествия Михаила ему на Иерусеме такая свобода не предоставляется. После их попыток совратить Михаила, когда он был во плоти во время пришествия, то есть за пределами изолированных миров греха, во всей Сатании исчезло всякое сострадание по отношению к Люциферу и Сатане.
\vs p053 8:3 \pc Пришествие Михаила положило конец бунту Люцифера во всей Сатании, за исключением планет Планетарных Принцев\hyp{}отступников. И в этом заключалась значимость личного опыта Иисуса, когда однажды, почти перед самой своей смертью во плоти, он воскликнул, обращаясь к своим ученикам: «Я видел Сатану, спадшего с неба, как молнию». Он пришел с Люцифером на Урантию для последней решающей битвы.
\vs p053 8:4 Сын Человеческий был убежден в успехе и знал, что его триумф в вашем мире навсегда определит статус его вековечных врагов не только в Сатании, но и в других двух системах, куда проник грех. Когда ваш Учитель в ответ на предложения Люцифера хладнокровно и с божественной твердостью ответил: «Отойди от меня, Сатана», для смертных это означало продолжение существования, а для ангелов --- безопасность. Это и было, в принципе, концом бунта Люцифера. Правда, суды Уверсы все же не вынесли никаких подлежащих исполнению решений по поводу обращения Гавриила, умоляющего уничтожить бунтовщиков, но такой декрет, несомненно, появится в надлежащее время, так как первый шаг в слушании этого дела уже предпринят.
\vs p053 8:5 Сын Человеческий признавал Калигастию в качестве официального Принца Урантии, почти до самой своей смерти. Сказал Иисус: «Ныне суд миру сему; ныне князь мира сего изгнан будет вон». И потом, еще ближе к завершению дела своей жизни провозгласил: «Князь мира сего осужден». А это и есть тот самый свергнутый и опозоренный Принц, которого когда\hyp{}то называли «Богом Урантии».
\vs p053 8:6 \pc Последнее, что сделал Михаил перед тем, как покинуть Урантию, было предложение о помиловании Калигастии и Далигастии, но они с презрением отвергли его доброту. Калигастия, ваш Планетарный Принц\hyp{}отступник, еще пользуется на Урантии свободой осуществлять свои нечестивые замыслы, но у него нет абсолютно никакой возможности проникнуть в мысли людей, не может он и подобраться ближе к их душам, чтобы искушать или развращать их, если только те действительно не желают его нечестивого присутствия.
\vs p053 8:7 \pc Перед пришествием Михаила эти правители тьмы стремились укрепить свой авторитет на Урантии, и они настойчиво противостояли младшим и низшим небесным личностям. Но со дня Пятидесятницы этот предатель Калигастия и такой же презренный его сподвижник Далигастия раболепствуют перед божественным величеством Райских Настройщиков Мысли и защитником\hyp{}Духом Истины, духом Михаила, который излит на всю плоть.
\vs p053 8:8 Скорее, даже так: ни один падший дух никогда не имел возможности вторгнуться в разум или потревожить душу детей Бога. Ни Сатана, ни Калигастия никогда не могли даже притронуться или приблизиться к верным сынам Бога; вера --- действенная броня против греха и порока. Истинно: «Тот, кто держится Бога, сохраняет себя, и нечистый не коснется его».
\vs p053 8:9 В общем, когда предполагается, что слабые и распущенные смертные находятся под влиянием дьяволов и демонов, это означает, что в них просто\hyp{}напросто преобладают собственные врожденные и низкие стремления, их уводят с истинного пути их же собственные, свойственные их природе, пристрастья. Дьявола часто обвиняют в том зле, которого он не совершал. С момента распятия Христа Калигастия практически бессилен.
\usection{9. Состояние бунта в настоящее время}
\vs p053 9:1 В начале бунта Люцифера Михаил предложил всем бунтовщикам спасение. Всем, кто представит доказательство искреннего раскаяния, он предложил --- при достижении им полного вселенского владычества --- прощение и восстановление в неком виде вселенской службы. Никто из вождей не принял этого милосердного предложения. Но тысячи ангелов и низших чинов небесных существ, включая сотни Материальных Сынов и Дочерей, приняли это помилование, провозглашенное Паноптийцами, и были восстановлены в правах во время воскресения Иисуса тысячу девятьсот лет назад. С тех пор эти существа были переведены в иерусемский мир Отца, где они должны быть задержаны по формальным юридическим причинам до тех пор, пока суды Уверсы не спустят решение по делу Гавриил \bibemph{против} Люцифера. Нет ни малейшего сомнения в том, что, когда будет вынесен смертный приговор, эти раскаявшиеся и спасенные личности будут исключены из декрета об уничтожении. Эти души, проходящие испытательный срок, в настоящее время трудятся с Паноптийцами, занимаясь работой по уходу за миром Отца.
\vs p053 9:2 \pc Архиобманщик никогда не был на Урантии с того времени, когда пытался отвратить Михаила от намерения завершить пришествие и затем поставить себя окончательно и бесповоротно неограниченным правителем Небадона. Когда Михаил стал законным главой вселенной Небадона, Люцифер был взят под стражу агентами Древних Дней на Уверсе, и с тех пор заключен на спутнике номер один переходных сфер Иерусема, входящих в группу Отца. И здесь правители других миров и систем увидят конец вероломного Владыки Сатании. Павел знал о положении этих руководителей\hyp{}бунтовщиков после пришествия Михаила, ибо он писал о начальниках Калигастии как о «сонмах духа злобы в местах небесных».
\vs p053 9:3 \pc Михаил, принимая верховное владычество над Небадоном, обратился к Древним Дней с просьбой о разрешении интернировать все личности, замешанные в бунте Люцифера, вплоть до постановления сверхвселенских судов по делу Гавриил \bibemph{против} Люцифера, принятому к рассмотрению верховным судом Уверсы почти двести тысяч лет назад по вашему летоисчислению. Что касается группы личностей в столице системы, то Древние Дней удовлетворили прошение Михаила, но с единственным исключением: Сатане было позволено периодически наносить визиты принцам\hyp{}отступникам в падших мирах до тех пор, пока другой Сын Бога не будет принят такими отступническими мирами, или до того времени, когда суды Уверсы начнут судебное разбирательство по делу Гавриил \bibemph{против} Люцифера.
\vs p053 9:4 Сатана мог прийти на Урантию, потому что у вас не было Сына, постоянно живущего на планете, --- ни Планетарного Принца, ни Материального Сына. С тех пор Махивента Мелхиседек был объявлен наместником Планетарного Принца Урантии, и открытие процесса Гавриил \bibemph{против} Люцифера ознаменовало начало временного планетарного режима во всех изолированных мирах. Верно, что Сатана периодически посещал Калигастию и остальных падших принцев вплоть до времени представления данных откровений, когда произошло первое слушание ходатайства Гавриила об уничтожении архибунтовщиков. В настоящее время Сатана безоговорочно содержится под стражей в одном из иерусемских тюремных миров.
\vs p053 9:5 \pc Со времени последнего пришествия Михаила никто во всей Сатании не пожелал отправиться в тюремные миры, чтобы принять на себя служение интернированным бунтовщикам. И с тех пор никто не присоединился к делу обманщика. В течение девятнадцати столетий это положение не менялось.
\vs p053 9:6 Мы не надеемся, что существующие ограничения, наложенные на Сатанию, будут сняты до того, как Древние Дней осуществят окончательное устранение архибунтовщиков. Контуры системы не будут восстановлены, пока Люцифер жив. Между тем он совершенно бездеятелен.
\vs p053 9:7 В Иерусеме бунт закончился. Он заканчивается в падших мирах, как только прибывает божественный Сын. Мы полагаем, что все бунтовщики, которые когда\hyp{}либо хотели принять помилование, сделали это. Мы ждем ослепительного возвещения, которое лишит этих предателей личностного существования. Мы предвидим, что вердикт Уверсы будет объявлен исполнительным возвещением, что осуществит уничтожение этих интернированных бунтовщиков. Тогда вы будете искать, где они, но их нельзя будет найти. «И те, кто знал тебя среди миров, будут удивляться тебе; ты внушал страх, но тебя больше не будет». И так все эти ничтожные изменники «станут, как будто их не было». Декрета Уверсы ждут все.
\vs p053 9:8 Но в течение веков семь тюремных миров духовной тьмы в Сатании служат серьезным предупреждением всему Небадону, действенно и красноречиво свидетельствуя великую истину, что «путь грешника тяжек»; «что внутри каждого греха таится семя его собственной погибели»; что «плата за грех --- смерть».
\vsetoff
\vs p053 9:9 [Представлено Мановандетом Мелхиседеком, некогда приданным исполнителям Урантии.]
