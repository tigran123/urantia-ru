\upaper{38}{Духи\hyp{}служители локальной вселенной}
\author{Мелхиседек}
\vs p038 0:1 Существуют три различных чина личностей Бесконечного Духа. Пламенный апостол понимал это, когда писал об Иисусе: «Который взошел на небо и пребывает одесную Бога и которому покорились ангелы и власти и силы». Ангелы --- духи\hyp{}служители времени; власти --- сонмы вестников пространства; силы --- высшие личности Бесконечного Духа.
\vs p038 0:2 \pc Подобно супернафимам в центральной вселенной и секонафимам в сверхвселенной, серафимы вместе со связанными с ними херувимами и сановимами составляют отряд ангелов локальной вселенной.
\vs p038 0:3 Все серафимы довольно единообразны по строению. От вселенной к вселенной, повсюду во всех семи сверхвселенных они обнаруживают минимум различий; это наиболее стандартные из всех духовных типов личностных существ. Их различные чины составляют отряд искусных и обычных служителей локальных творений.
\usection{1. Происхождение серафимов}
\vs p038 1:1 Серафимы создаются Духом\hyp{}Матерью Вселенной, и с ранних времен существования Небадона, еще во времена сотворения «ангелов\hyp{}паттернов» и отдельных ангельских прототипов, серафимы уже постоянно создавались группами из 41\,472. Сын\hyp{}Творец и вселенское представительство Бесконечного Духа совместно создают множество Сынов и других вселенских личностей. После завершения этих совместных действий Сын занимается сотворением Материальных Сынов, первых созданий, имеющих пол, в это же время Дух\hyp{}Мать Вселенной в одиночку прилагает первоначальные усилия, направленные на воспроизводство духов. Таким образом начинается сотворение ангельских сонмов локальной вселенной.
\vs p038 1:2 Эти ангельские чины замысливаются при планировании эволюции смертных созданий, обладающих волей. Сотворение серафимов относится ко времени достижения Духом\hyp{}Матерью Вселенной относительной личности --- не как равноправной с Сыном\hyp{}Мастером, как это свойственно более позднему периоду, а как творческой помощницы Сына\hyp{}Творца на раннем этапе. До этого события серафимов для несения службы в Небадоне на время предоставляла какая\hyp{}нибудь соседняя вселенная.
\vs p038 1:3 Серафимы по\hyp{}прежнему периодически создаются; вселенная Небадона все еще находится в процессе создания. Дух\hyp{}Мать Вселенной никогда не прекращает творческой деятельности в растущей и совершенствующейся вселенной.
\usection{2. Природа ангелов}
\vs p038 2:1 Ангелы не имеют материальных тел, но они конкретные и самобытные существа, имеющие духовную природу и происхождение. Хотя ангелы и невидимы для смертных, они воспринимают вас такими, какие вы есть во плоти, не нуждаясь при этом ни в посредниках, ни в толкователях; они интеллектуально понимают образ жизни смертных и разделяют все не плотские эмоции и чувства человека. Они ценят ваши труды в области музыки, искусства и настоящего юмора и получают от них большое удовольствие. Они полностью осведомлены о вашей нравственной борьбе и духовных трудностях. Они любят людей, и ваши усилия понять и полюбить их могут принести только благо.
\vs p038 2:2 \pc Хотя серафимы --- существа очень любящие и участливые, но им чужды сексуальные эмоции. Они такие, какими вы станете в мирах\hyp{}обителях, где не будете «ни жениться, ни выходить замуж, но будете как ангелы небесные». Ибо все, кто «будут сочтены достойными достигнуть миров\hyp{}обителей, ни женятся, ни замуж не выходят; и не умирают они уже больше, ибо они равны ангелам». Тем не менее когда мы общаемся с созданиями, имеющими пол, о существах, ведущих происхождение непосредственно от Отца и Сына, тогда у нас принято говорить как о сынах Бога, а детей Духа называть дочерьми Бога. Поэтому ангелов обычно обозначают местоимениями женского рода на планетах, где различают пол.
\vs p038 2:3 Серафимы созданы таким образом, чтобы функционировать и на духовном, и на материальном уровне. Немного найдется областей моронтийной или духовной деятельности, закрытых им для служения. Хотя по личностному статусу ангелы не так уж далеко отстоят от людей, по некоторым функциональным свойствам серафимы намного их превосходят. Они обладают многими способностями, совершенно недоступными человеческому пониманию. Например: вам было сказано, что «у вас и волосы на голове все сочтены», и это правда, но серафим не тратит свое время на их подсчет и не следит за точностью числа в каждый данный момент. Ангелы обладают прирожденными и автоматическими (то есть автоматическими на вашем уровне восприятия) способностями знать такие вещи; вы бы поистине сочли серафима математическим гением. Поэтому серафимы с чрезвычайной легкостью выполняют многочисленные обязанности, которые смертным показались бы неподъемными.
\vs p038 2:4 \pc Ангелы стоят выше вас по духовному статусу, но они не являются вашими судьями или обвинителями. Каковы бы ни были ваши проступки, «ангелы, превосходя вас мощью и могуществом, не выдвигают против вас никаких обвинений». Ангелы не выступают в качестве судей человечества, так же и отдельным смертным не следует преждевременно судить своих собратьев.
\vs p038 2:5 \pc Хорошо, что вы любите их, но вам не следует преклоняться перед ними; ангелы --- не объект для поклонения. Великий серафим Лоялация, когда ваш пророк «пал к ногам ангела в знак поклонения ей», сказала: «Смотри, больше так не делай; я служу равно с тобой и с твоими расами, и всем нам предписано поклоняться Богу».
\vs p038 2:6 По природе и личностным дарованиям серафимы лишь незначительно опережают человеческие расы в иерархии сотворенных существ. Действительно, когда вы освобождаетесь от плоти, то становитесь очень похожими на них. В мирах\hyp{}обителях вы начнете ценить серафимов, в сферах созвездий --- радоваться им, а в Спасограде они будут делить с вами свои места отдыха и богопочитания. На протяжении всего моронтийного и последующего духовного восхождения ваше братство с серафимами будет идеальным; дружба --- превосходной.
\usection{3. Нераскрытые ангелы}
\vs p038 3:1 Повсюду в сферах локальной вселенной функционируют многочисленные чины духовных существ, не раскрытые смертным потому, что они никоим образом не связаны с эволюционным планом Райского восхождения. В этом тексте слово «ангел» преднамеренно сужено и используется только в отношении тех серафимских и связанных с ними потомков Духа\hyp{}Матери Вселенной, которые имеют непосредственное отношение к осуществлению планов продолжения существования смертных. В локальной вселенной служат еще шесть чинов родственных существ --- нераскрытых ангелов, которые конкретно никак не связаны с вселенскими видами деятельности, имеющими отношение к Райскому восхождению эволюционных смертных. Эти шесть групп, связанные с ангелами, никогда не называются серафимами, никогда о них не говорят и как о духах\hyp{}служителях. Эти личности целиком заняты административными и другими делами Небадона --- занятиями, никак не связанными с поступательным путем духовного восхождения и достижения совершенства.
\usection{4. Миры серафимов}
\vs p038 4:1 Девятая группа, состоящая из семи первичных сфер в контуре Спасограда, --- это миры серафимов. Около каждого из этих миров есть по шесть спутников, на которых находятся специальные школы, посвященные всем аспектам ангельского обучения. Хотя серафимы имеют доступ ко всем сорока девяти мирам, из которых состоит эта группа сфер Спасограда, они занимают исключительно только первое из семи скоплений миров. Остальные шесть заняты другими шестью чинами ангельских сподвижников, которые не раскрыты на Урантии; каждая такая группа имеет свой центр в одном из этих шести первичных миров и осуществляет особые виды деятельности на шести подчиненных спутниках. Каждый ангельский чин имеет свободный доступ во все миры этих семи различных групп.
\vs p038 4:2 Эти центральные миры относятся к числу великолепнейших сфер Небадона; ангельские владения отличаются красотой и обширностью. Здесь каждый серафим имеет свой настоящий дом, а «дом» означает место проживания двух серафимов; они живут парами.
\vs p038 4:3 \pc Хотя нет серафимов мужского и женского пола, как у Материальных Сынов и человеческих рас, эти существа имеют или отрицательный, или положительный заряд энергии. При исполнении большинства заданий для выполнения задачи требуется два ангела. Когда они не объяты контуром, то могут действовать по одиночке; когда они стационарны, им также не нужны дополнители. Обычно они сохраняют своих изначальных дополнителей, но не всегда. Такие союзы требуются им, в первую очередь, для выполнения своих функций; им не свойственны сексуальные эмоции, хотя они и чрезвычайно личностные и поистине любящие существа.
\vs p038 4:4 Помимо предназначенных для них домов, у серафимов есть также центры группы, роты, батальоны и подразделения. Они собираются на встречи, на которых каждая присутствует раз в тысячелетие в соответствии с юбилеем своего сотворения. Если серафим исполняет обязанности, не позволяющие покинуть пост, то она присутствует на этих встречах попеременно со своим дополнителем, в это время ее подменяет серафим с другой датой рождения. Таким образом, каждый из серафимов\hyp{}партнеров присутствует на собраниях, по крайней мере, через раз.
\usection{5. Обучение серафимов}
\vs p038 5:1 Серафимы проводят первое тысячелетие своего существования как внештатные наблюдатели в Спасограде и в школах связанных с ним миров. Второе тысячелетие проводится в мирах серафимов в спасоградском контуре. Их центральной учебной школой руководят сейчас первые сто тысяч серафимов Небадона, а во главе стоит изначальный, или первородный ангел этой локальной вселенной. Группу серафимов Небадона, сотворенную первой, обучал отряд из тысячи серафимов из Авалона; впоследствии же наших ангелов учили их собственные старшие товарищи. Мелхиседеки также принимают большое участие в образовании и обучении всех ангелов локальной вселенной --- серафимов, херувимов и сановимов.
\vs p038 5:2 По завершении этого периода обучения в мирах серафимов в Спасограде серафимы мобилизуются в обычные группы и подразделения ангельской организации и назначаются в какое\hyp{}нибудь из созвездий. Они еще не облечены полномочиями духов\hyp{}служителей, хотя уже полностью вступили в стадию ангельского обучения, предшествующую возведению в ранг.
\vs p038 5:3 Серафимы посвящаются в духи\hyp{}служители через служение в качестве наблюдателей в низших эволюционных мирах. Обретя этот опыт, они возвращаются в миры, относящиеся к центру того созвездия, в которое они назначены, и начинают проходить более углубленное обучение и более целенаправленно готовиться к службе в какой\hyp{}то конкретной локальной системе. После получения такого общего образования они направляются на службу в одну из локальных систем. В архитектурных мирах, связанных со столицей какой\hyp{}нибудь системы Небадона, наши серафимы завершают свое обучение и облекаются полномочиями духов\hyp{}служителей времени.
\vs p038 5:4 Достигнув этого ранга, серафимы могут при выполнении заданий путешествовать по всему Небадону и даже Орвонтону. Их деятельность во вселенной не имеет границ и пределов; они тесно связаны с материальными созданиями миров и вечно служат низшим чинам духовных личностей, устанавливая контакт между этими существами духовного мира и смертными из материальных сфер.
\usection{6. Организация серафимов}
\vs p038 6:1 После второго тысячелетия пребывания в ангельских центрах серафимы под руководством глав объединяются в группы по двенадцать (12 пар, 24 серафима), и двенадцать таких групп составляют роту (144 пары, 288 серафимов), которой командует ведущий. Двенадцать рот под руководством командира составляют батальон (1\,728 пар, 3\,456 серафимов), а двенадцать батальонов под командованием управляющего образуют ангельское подразделение (20\,736 пар, или 41\,472 серафима), двенадцать же подразделений под командованием надзирателя составляют легион, в котором насчитывается 248\,832 пары, или 497\,664 серафима. Такую группу ангелов подразумевал Иисус, когда сказал в ту ночь в Гефсиманском саду: «Я и теперь могу просить Отца моего и он представит мне более, нежели двенадцать легионов ангелов».
\vs p038 6:2 Двенадцать легионов ангелов --- это сонм, в который входят 2\,985\,984 пары, или 5\,971\,968 серафимов, а двенадцать таких сонмов (35\,831\,808 пар, или 71\,663\,616 серафимов) образуют самую крупную действующую организацию серафимов --- ангельскую армию. Ангельским сонмом командует архангел или какая\hyp{}нибудь другая личность, равноправная с ним по статусу, ангельские же армии управляются Блестящими Вечерними Звездами или другими непосредственными заместителями Гавриила. Гавриил же является «верховным командующим небесными армиями», главным распорядителем у Владыки Небадона --- «Господа Бога сонмов».
\vs p038 6:3 Хотя серафимы служат под непосредственным руководством Бесконечного Духа, персонализированного в Спасограде, со времени пришествия Михаила на Урантию они и все прочие чины локальной вселенной подчинены владычеству Сына\hyp{}Мастера. Даже когда Михаил родился во плоти на Урантии, было послано сверхвселенское возвещение всему Небадону, которое гласило: «И пусть все ангелы поклоняются ему». Все ангелы подчинены его владычеству; они являются частью той группы, которая была названа «его могучие ангелы».
\usection{7. Херувимы и сановимы}
\vs p038 7:1 По всем основным дарованиям херувимы и сановимы подобны серафимам. Они имеют такое же происхождение, но не всегда такое же предназначение. Они поразительно умны, удивительно искусны, трогательно любящи и почти подобны людям. Они представляют собой низший чин ангелов и поэтому еще более родственны самым развитым типам людей на эволюционных мирах.
\vs p038 7:2 Херувимы и сановимы органически взаимосвязаны, функционально объединены. Одни --- личности положительной энергии, другие --- отрицательной. Дефлектором, отклоняющим вправо, или положительно заряженным ангелом является херувим --- старшая, или контролирующая личность. Дефлектором, отклоняющим влево, или отрицательно заряженным ангелом является сановим --- дополнитель. Каждый из этих типов ангелов, действуя в одиночку, имеет очень ограниченные возможности; поэтому они обычно служат парами. Когда они служат независимо от своих руководителей\hyp{}серафимов, тогда более чем когда\hyp{}либо зависимы от взаимного контакта и всегда действуют вместе.
\vs p038 7:3 \pc Херувимы и сановимы выступают как верные и искренние помощники серафимов\hyp{}служителей, и эти подчиненные помощники предоставлены всем семи чинам серафимов. Херувимы и сановимы служат в этом качестве веками, но при выполнении заданий они не сопровождают серафимов за пределы локальной вселенной.
\vs p038 7:4 Херувимы и сановимы выполняют повседневную духовную работу в конкретных мирах систем. При выполнении неличностного задания и в чрезвычайных ситуациях они могут нести службу вместо пары серафимов, но они никогда не действуют даже временно в качестве ангела, заботящегося о человеке; это исключительная привилегия серафимов.
\vs p038 7:5 \pc Назначенные на какую\hyp{}либо планету, херувимы поступают на местные учебные курсы, на которых изучают планетарные обычаи и языки. Все духи\hyp{}служители времени двуязычны и говорят на языке своей родной локальной вселенной и на языке родной сверхвселенной. В результате учебы в школах сфер они овладевают еще и другими языками. Херувимы и сановимы, подобно серафимам и всем прочим чинам духовных существ, постоянно поглощены стремлением к самосовершенствованию. Только подчиненные существа контроля мощи и управления энергией не способны к развитию; все существа, обладающие действительной или потенциальной личностной волей, стремятся к новым достижениям.
\vs p038 7:6 \pc Херувимы и сановимы по природе очень близки к моронтийному уровню существования, и они особенно эффективны в пограничной области между физической, моронтийной и духовной сферами. Эти дети Духа\hyp{}Матери локальной вселенной отличаются наличием «четвертых созданий» --- так же, как и Сервиталы Хавоны и примирительные комиссии. Каждый четвертый херувим и каждый четвертый сановим --- квазиматериальны и очень явственно напоминают моронтийный уровень существования.
\vs p038 7:7 Эти ангельские четвертые создания оказывают огромную помощь серафимам больше в материальных аспектах их вселенской и планетарной деятельности. Моронтийные херувимы выполняют также много необходимых пограничных задач в моронтийных учебных мирах и в больших количествах назначаются на службу в качестве Моронтийных Компаньонов. Для моронтийных сфер они являются примерно тем же, чем срединные создания --- для эволюционных планет. В обитаемых мирах эти моронтийные херувимы часто действуют во взаимосвязи с срединными созданиями. Херувимы и срединные создания --- это, несомненно, разные чины существ; у них различное происхождение, но в их природе и функционировании обнаруживается огромное сходство.
\usection{8. Эволюция херувимов и сановимов}
\vs p038 8:1 Херувимам и сановимам открыты многочисленные пути продвижения по службе, ведущие к повышению статуса, который может быть еще более повышен объятием Божественной Служительницы. Существует три больших класса херувимов и сановимов в соответствии с их эволюционным потенциалом:
\vs p038 8:2 \ublistelem{1.}\bibnobreakspace \bibemph{Кандидаты на восхождение.} Эти существа по своей природе являются кандидатами на получение статуса серафимов. Херувимы и сановимы этого чина --- блестящие, хотя и не равные серафимам по своим врожденным дарованиям; но благодаря прилежанию и опыту они могут достичь полноправного статуса серафима.
\vs p038 8:3 \ublistelem{2.}\bibnobreakspace \bibemph{Херувимы серединной фазы.} Не все херувимы и сановимы равны по своему потенциалу восхождения, и в эту группу входят ограниченные по своей природе существа из числа ангельских созданий. Большинство из них останутся херувимами и сановимами, хотя ограниченное число наиболее одаренных может добиться службы серафимов.
\vs p038 8:4 \ublistelem{3.}\bibnobreakspace \bibemph{Моронтийные херувимы.} Эти «четвертые существа» ангельских чинов всегда сохраняют свои квазиматериальные свойства. Вместе с большинством своих собратьев серединной фазы они останутся херувимами и сановимами вплоть до завершения фактуализации Верховного Существа.
\vs p038 8:5 \pc Если потенциал роста второй и третьей групп несколько ограничен, то кандидаты на восхождение могут достичь высот вселенской службы серафимов. Многие из более опытных херувимов прикрепляются к серафимам\hyp{}хранительницам предназначения и, таким образом, становятся кандидатами на обретение статуса Учителей Миров\hyp{}Обителей после того, как вышестоящие серафимы покинут их. Хранительницы предназначения не имеют херувимов и сановимов в качестве помощников, когда их смертные подопечные достигают моронтийной жизни. И когда другим типам эволюционных серафимов даруется допуск в Серафимоград и Рай, то при выходе за пределы Небадона они должны покинуть своих бывших подчиненных. Такие покинутые херувимы и сановимы обычно объемлются Духом\hyp{}Матерью Вселенной и приобретают статус серафима, достигая, таким образом, уровня, равнозначного уровню Учителя Миров\hyp{}Обителей.
\vs p038 8:6 После того, как некогда объятые херувимы и сановимы долго прослужат Учителями Миров\hyp{}Обителей в моронтийных сферах (от низших до высших) и их отряд в Спасограде полностью укомплектуется, Яркая и Утренняя Звезда призывает этих верных слуг созданий времени предстать перед ним. Принимается присяга на преобразование личности; и после этого группами по семь тысяч продвинутые и старшие херувимы и сановимы вновь объемлются Духом\hyp{}Матерью Вселенной. Из второго объятия они выходят уже полноправными серафимами. С этого момента для таких получивших новую жизнь херувимов и сановимов целиком и полностью открыт путь серафимов со всеми его Райскими возможностями. Такие ангелы могут назначаться хранительницами предназначения к какому\hyp{}нибудь смертному, и если смертный подопечный достигает продолжения существования, тогда они получают право на продвижение и в Серафимоград, и в семь кругов достижений серафимов, и даже в Рай и в Отряд Финалитов.
\usection{9. Срединные создания}
\vs p038 9:1 Срединные создания занимают троякое место в классификации: их справедливо классифицируют с восходящими Сынами Бога; фактически их объединяют в одну группу с чинами постоянных граждан, функционально же причисляют к духам\hyp{}служителям времени из\hyp{}за тесной и действенной связи с ангельскими сонмами, занимающимися служением смертному человеку в конкретных мирах пространства.
\vs p038 9:2 Эти уникальные создания появляются в большинстве обитаемых миров и всегда обнаруживаются на десятичных планетах --- тех, где проводится эксперимент с формами жизни, как, например, Урантия. Срединники бывают двух типов --- первого рода и второго рода, и они появляются следующим образом:
\vs p038 9:3 \ublistelem{1.}\bibnobreakspace \bibemph{Срединники первого рода ---} более духовная группа, несколько стандартизированный чин существ, которые единообразно происходят из модифицированных восходящих смертных, входящих в штаты Планетарных Принцев. Численность срединных созданий первого рода всегда пятьдесят тысяч, и группы большей числом нет ни на одной из планет, на которые распространяется их служение.
\vs p038 9:4 \ublistelem{2.}\bibnobreakspace \bibemph{Срединники второго рода ---} более материальная группа этих созданий --- сильно разнятся по численности в разных мирах, хотя в среднем их около пятидесяти тысяч. Они различным образом происходят от планетарных реализаторов биологического подъема, Адамов и Ев, или от их непосредственных потомков. Эти срединные создания второго рода создаются в эволюционных пространственных мирах не менее чем двадцатью четырьмя различными способами. На Урантии эта группа появилась необычным и исключительным образом.
\vs p038 9:5 \pc Возникновение и той, и другой группы не было эволюционной случайностью; обе группы --- существенные элементы заранее продуманных планов вселенских архитекторов, и их появление в развивающихся мирах в нужный момент происходит в соответствии с изначальными замыслами руководящих Носителей Жизни и с их планами развития.
\vs p038 9:6 Срединники первого рода ангельским способом заряжаются интеллектуальной и духовной энергией и одинаковы по интеллектуальному статусу. Семь духов\hyp{}помощников разума не вступают с ними в контакт; и только шестой и седьмой --- дух почитания и дух мудрости --- могут служить второй группе.
\vs p038 9:7 Срединники второго рода заряжаются физической энергией адамическим способом, духовно объемлются контуром серафима и интеллектуально одарены моронтийным переходным типом разума. Они подразделяются на четыре физических типа, духовно --- на семь чинов, а также на двенадцать уровней по интеллектуальной реакции на совместное служение последних двух духов\hyp{}помощников и моронтийного разума. Этим разнообразием определяются различия в их деятельности и планетарных назначениях.
\vs p038 9:8 Срединники первого рода больше напоминают ангелов, чем смертных; чины второго рода значительно более похожи на людей. Они оказывают неоценимую помощь друг другу в исполнении своих разнообразных планетарных заданий. Служители первого рода могут достигать взаимосвязи и сотрудничества с контролерами как моронтийной, так и духовной энергии и со штатом контура разума. Группа второго рода может устанавливать рабочие связи только с физическими контролерами и операторами материальных контуров. Но поскольку один чин срединников может устанавливать идеально согласованный контакт с другим, каждая из двух групп способна достигать практического использования всего спектра энергий --- от грубой физической мощи материальных миров до переходных фаз различных видов вселенских энергий и высших сил духовной реальности небесных сфер.
\vs p038 9:9 Последовательное взаимодействие смертного человека, срединника второго рода, срединника первого рода, моронтийного херувима, херувима серединной фазы и серафима прекрасно преодолевает разрыв между материальным и духовным мирами. В личном опыте смертного индивидуума эти различные уровни, несомненно, более или менее объединены и становятся лично значимыми посредством ненаблюдаемых и таинственных действий божественного Настройщика Мысли.
\vs p038 9:10 \pc В нормальных мирах срединники первого рода служат при Планетарном Принце как небесный информационный и развлекательный отряд, а служители второго рода продолжают свое сотрудничество с адамической системой содействия делу прогресса планетарной цивилизации. В случае отступничества Планетарного Принца и неудачи Материального Сына, как это случилось на Урантии, срединные создания становятся подопечными Владыки Системы и служат под направляющим руководством действующего хранителя планеты. Однако еще лишь в трех мирах в Сатании эти существа действуют одной группой под единым руководством, так же как это делают объединенные срединные служители Урантии.
\vs p038 9:11 Планетарная деятельность срединников как первого, так и второго рода, в многочисленных конкретных мирах вселенной различна и разнотипна, но на нормальных и обычных планетах их виды деятельности сильно отличаются от тех обязанностей, которые они выполняют на изолированных планетах, таких как Урантия.
\vs p038 9:12 Срединники первого рода являются планетарными историками, и со времени прибытия Планетарного Принца до периода установления планеты в свете и жизни они представляют живые картины и создают полотна на темы планетарной истории для экспозиций планет в центральных мирах систем.
\vs p038 9:13 \pc Срединники подолгу остаются в обитаемом мире, но если оправдают доверие, то, в конце концов, непременно получат признание за свою долгую службу по поддержанию владычества Сына Творца; они будут должным образом вознаграждены и за свое терпеливое служение материальным смертным в их мирах со временем и пространством. Рано или поздно все облеченные полномочиями срединные создания будут зачислены в ряды восходящих Сынов Бога и будут должным образом направлены на долгий путь Райского восхождения вместе с теми самыми смертными животного происхождения, их земными братьями, которых они так ревностно охраняли и которым так действенно служили на протяжении долгого пребывания на планете.
\vsetoff
\vs p038 9:14 [Представлено Мелхиседеком, действующим по просьбе Главы Сонмов Серафимов Небадона.]
