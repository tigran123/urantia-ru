\upaper{115}{Верховное Существо}
\author{Могучий Вестник}
\vs p115 0:1 По отношению к Богу сыновство --- само по себе великое родство. По отношению к Богу Верховному свершения --- это необходимое предварительное условие для получения статуса --- человек должен что\hyp{}то делать и кем\hyp{}то быть.
\usection{1. Относительность концептуальных построений}
\vs p115 1:1 Частичные, неполные и развивающиеся интеллекты были бы в главной вселенной беспомощными, неспособными сформировать первый рациональный паттерн мысли, если бы не врожденная способность всякого разума, будь то высокого или низкого формировать мысленные концепции \bibemph{построения вселенной.} Если разум не может постичь заключений, если он не может проникнуть к истинным источникам, то такой разум неизбежно будет заключения постулировать и источники изобретать, чтобы иметь средства логического мышления в границах этих созданных разумом постулатов. Хотя такие концепции построения вселенной тварному мышлению необходимы для рациональных интеллектуальных операций, все они без исключения в большей или меньшей степени ошибочны.
\vs p115 1:2 Концептуальные системы построения вселенной истинны лишь относительно; они --- полезные строительные леса, которые в конце концов должны отступить перед экспансией расширенного космического понимания. Понимания истины, красоты и добродетели, морали, этики, долга, любви, божественности, происхождения, бытия, цели, предназначения, времени, пространства и даже Божества истинны лишь относительно. Бог намного, намного больше Отца, но Отец является высшим представлением человека о Боге; тем не менее представление об отношениях между Творцом и творением как об отношениях между Отцом и Сыном будет усиливаться теми сверхсмертными представлениями о Божестве, которые будут достигнуты в Орвонтоне, в Хавоне и в Раю. Человек может понимать систему построения вселенной, как это доступно смертному, однако это не означает, что он не может представить себе иные и более возвышенные построения в которых может происходить мышление.
\vs p115 1:3 Чтобы облегчить смертным понимание вселенной вселенных, различные уровни космической реальности были определены как конечные, абсонитные и абсолютные. Из них лишь абсолютные являются неограниченно вечными, истинно экзистенциальными. Абсонитные и конечные же являются производными, модификациями, ограничениями и ослаблениями исходной и изначальной абсолютной реальности бесконечности.
\vs p115 1:4 Сферы конечного существуют в силу вечного замысла Бога. Конечные творения, высокие и низкие, могут выдвигать теории и именно так поступали объясняя необходимость конечного в устройстве космоса, однако в итоге конечное существует, потому что так было угодно Богу. Вселенную нельзя объяснить; не может и конечное творение предложить рациональную причину своего собственного индивидуального существования, не обращаясь к предшествующим ему деяниям и предсущей ему воле породивших его существ, Творцов или родителей.
\usection{2. Абсолютная основа для верховенства}
\vs p115 2:1 С экзистенциальной точки зрения, в галактиках не может произойти ничего нового, ибо полнота бесконечности, свойственная Я ЕСТЬ, вечно присутствует в семи Абсолютах, функционально соединена в триединствах и передаточно соединена в триодитах. Однако тот факт, что бесконечность таким образом экзистенциально присутствует в этих абсолютных соединениях, никоим образом не делает невозможным получение нового космического опыта. С точки зрения конечного творения, бесконечность содержит много потенциального, много такого, что скорее является будущей возможностью, нежели настоящей актуальностью.
\vs p115 2:2 Ценность --- это уникальный элемент во вселенской реальности. Мы не понимаем, как может быть увеличена ценность чего\hyp{}то бесконечного и божественного. Однако мы обнаруживаем, что \bibemph{значения} могут если не расширяться, то модифицироваться даже в отношениях бесконечного Божества. В основанных на опыте вселенных благодаря расширенному пониманию значений реальности даже божественные ценности, и те возрастают как актуальности.
\vs p115 2:3 Вся схема вселенского творения и эволюции на всех уровнях постижения опытом несомненно является вопросом преобразования потенциального в актуальное; причем это превращение в равной степени относится к областям потенции пространства, потенции разума и потенции духа.
\vs p115 2:4 Очевидный метод, посредством которого возможности космоса могут обрести подлинное существование, меняется от уровня к уровню, являясь основанной на опыте эволюцией в конечном и основанным на опыте выявлением в абсонитном. Экзистенциальная бесконечность в своем свойстве заключать в себе все поистине безгранична, и это свойство заключать в себе все должно при необходимости затрагивать даже возможность эволюционного конечного опыта. Причем возможность такого основанного на опыте роста становится вселенской актуальностью благодаря триодитным отношениям, соприкасающимся с Верховным и осуществляющимся в Верховном.
\usection{3. Изначальное, актуальное и потенциальное}
\vs p115 3:1 Абсолютный космос концептуально безграничен; определить величину и природу этой первичной реальности значит наложить ограничения на бесконечность и затемнить ясное понятие вечности. Идея о бесконечно\hyp{}вечном, вечно\hyp{}бесконечном, масштабно неограничена и фактически абсолютна. Ни в прошлом, ни в настоящем, ни в будущем на Урантии не было, нет и не будет языка, адекватного для выражения реальности бесконечности или бесконечности реальности. Человек, конечное творение в бесконечном космосе, должен довольствоваться искаженными проявлениями и нечеткими представлениями об этом необозримом, безграничном, никогда не начинающемся и никогда не заканчивающемся бытии, постижение которого, действительно, выходит за пределы его возможностей.
\vs p115 3:2 Разум не может надеяться постичь понятие об Абсолюте, не попытавшись вначале разрушить единство подобной реальности. Разуму свойственно объединять все различия, однако при отсутствии таких различий он не находит основы, на которой он бы мог попытаться сформулировать разумные понятия.
\vs p115 3:3 Первоначальный стаз бесконечности требует сегментации перед тем, как человек пытается достичь понимания. В бесконечности существует единство, отраженное в этих текстах как Я ЕСТЬ, --- главный постулат разума творения. Однако творение никогда не сможет понять, как получается, что это единство становится двойственностью, триединством и многообразием, оставаясь при этом неограниченным единством. С похожей проблемой человек сталкивается, когда приостанавливается, чтобы поразмышлять о нераздельном Божестве Троицы наряду с множественной персонализацией Бога.
\vs p115 3:4 Лишь отдаленность человека от бесконечности вынуждает выражать это понятие одним словом. Хотя, с одной стороны, бесконечность --- это ЕДИНСТВО, с другой --- это МНОГООБРАЗИЕ без конца и без границ. Бесконечность, как рассматривают ее конечные разумные существа, представляет собой самый большой парадокс философии творения и конечной метафизики. Хотя духовная природа человека в опыте богопочитания устремляется к Отцу, который бесконечен, способность человека к интеллектуальному постижению исчерпывается максимальным пониманием Верховного Существа. За пределами Верховного, понятия --- все больше названия и все меньше и меньше истинные обозначения реальности; они все больше и больше становятся осуществляемой творением проекцией конечного понимания на сверхконечное.
\vs p115 3:5 \pc Одна основная концепция абсолютного уровня влечет за собой постулат о трех фазах:
\vs p115 3:6 \ublistelem{1.}\bibnobreakspace \bibemph{Изначальная.} Неограниченное понятие о Первоисточнике и Центре, о том исходном проявлении Я ЕСТЬ, от которого берет начало всякая реальность.
\vs p115 3:7 \ublistelem{2.}\bibnobreakspace \bibemph{Актуальная.} Объединение трех Абсолютов актуальности, трех Источников и Центров: Второго, Третьего и Райского. Этот триодит Вечного Сына, Бесконечного Духа и Райского Острова составляет реальное откровение об изначальности Первоисточника и Центра.
\vs p115 3:8 \ublistelem{3.}\bibnobreakspace \bibemph{Потенциальная.} Объединение трех Абсолютов потенциальности --- Божественного, Неограниченного и Вселенского Абсолютов. Этот триодит экзистенциальной потенциальности образует потенциальное откровение об изначальности Первоисточника и Центра.
\vs p115 3:9 \pc Взаимосвязь Изначального, Актуального и Потенциального создает внутри бесконечности напряженности, которые приводят к возможности всякого роста во вселенной; причем рост является сущностью Семеричного, Верховного и Предельного.
\vs p115 3:10 В объединении Божественного, Вселенского и Неограниченного Абсолютов потенциальность абсолютна, а актуальность --- появляется; в объединении Второго, Третьего и Райского Источников и Центров актуальность абсолютна, а потенциальность\hyp{}появляется; в отношении же изначальности Первоисточника и Центра мы не можем сказать, являются ли актуальность и потенциальность существующими или появляющимися --- \bibemph{Отец есть.}
\vs p115 3:11 С точки зрения времени, Актуальное --- это то, что было, и то, что есть; Потенциальное --- то, что становится, и то, что будет, а Изначальное --- то, что есть. С точки же зрения вечности, различия между Изначальным, Актуальным и Потенциальным не очевидны. Эти триединые качества на Райско\hyp{}вечных уровнях не различаются таким образом. В вечности все есть --- только во времени и пространстве еще не все явлено.
\vs p115 3:12 С точки зрения создания, актуальность --- это сущность, а потенциальность --- возможность. Актуальность существует в самом центре и оттуда распространяется в периферийную бесконечность; потенциальность же идет внутрь от бесконечной периферии и сходится в центре всего. Изначальность есть то, что вначале вызывает, а затем уравновешивает двойные движения цикла превращения реальности из потенциального в актуальное и потенциализации существующего актуального.
\vs p115 3:13 Три Абсолюта потенциальности действуют исключительно на вечном уровне космоса и, следовательно, как таковые никогда не действуют на уровнях субабсолютных. На нисходящих уровнях реальности триодит потенциальности явлен у Предельного и на Верховном. Потенциал может потерять силу и не суметь актуализироваться во времени относительно части на некотором субабсолютном уровне, но в целом --- никогда. Воля Бога в итоге преобладает --- не всегда в отношении индивидуального, но неизменно в отношении тотального.
\vs p115 3:14 Все существующие в космосе имеют свой центр в триодите актуальности; дух ли это, разум или энергия --- все сосредоточивается в союзе Сына, Духа и Рая. Личность духовного Сына --- вот главный паттерн для всякой личности во всех вселенных. Сущность Райского Острова есть главный паттерн, в котором Хавона представляет собой совершенное откровение этого паттерна, а сверхвселенные --- совершенствующееся откровение. Носитель Объединенных Действий одновременно является разумной активацией космической энергии, концептуализацией цели, к которой стремится дух и интеграцией математических причин и следствий материальных уровней с волевыми целями и мотивами духовного уровня. В конечной вселенной и для нее Сын, Дух и Рай действуют в Предельном и на Предельном, поскольку он обусловлен и ограничен в Верховном.
\vs p115 3:15 Актуальность (Божества) есть то, чего ищет человек в восхождении к Раю. Потенциальность (человеческой божественности) есть то, что человек развивает в этом поиске. Изначальное же есть то, что делает возможным сосуществование и интеграцию человека актуального, человека потенциального и человека вечного.
\vs p115 3:16 \pc Окончательная динамика космоса связана с непрерывным переходом реальности от потенциальности к актуальности. Теоретически у такого преобразования может быть конец, фактически же подобное невозможно, так как и Потенциальное, и Актуальное вовлечены в контур Изначального (Я ЕСТЬ), и это отождествление навсегда устраняет возможность положить предел эволюционному развитию вселенной. Что бы ни отождествлялось с Я ЕСТЬ, оно никогда не может положить конец развитию, поскольку актуальность потенциалов Я ЕСТЬ абсолютна, и потенциальность актуальностей Я ЕСТЬ абсолютна тоже. Актуальности всегда будут открывать новые пути реализации прежде невозможных потенциалов --- каждое человеческое решение не только актуализирует новую реальность в человеческом опыте, но и открывает новую возможность для человеческого роста. Человек живет в каждом ребенке и моронтийный восходящий пребывает в зрелом человеке, знающем Бога.
\vs p115 3:17 В тотальном космосе остановка роста не может произойти никогда, поскольку основа для роста --- абсолютные актуальности --- не ограничена, а возможности для роста --- абсолютные потенциалы --- не имеют предела. С практической точки зрения, философы вселенной пришли к заключению, что такой вещи, как \bibemph{конец,} не существует.
\vs p115 3:18 С ограниченной точки зрения, действительно, существует много окончаний, множество завершений деятельности, но с более широкой точки зрения на более высоком вселенском уровне, окончаний нет, а есть лишь переходы от одной фазы развития к другой фазе. Основное постоянное свойство главной вселенной связано с несколькими вселенскими периодами, а именно: с периодами Хавоны, сверхвселенных и уровней внешнего пространства. Однако даже эти основные деления последовательных связей не могут быть чем\hyp{}то большим, нежели относительные ориентиры на нескончаемой стезе вечности.
\vs p115 3:19 Окончательное проникновение истины, красоты и добродетели Верховного Существа может лишь открыть развивающемуся созданию те абсонитные качества предельной божественности, которые лежат вне концептуальных уровней истины, красоты и добродетели.
\usection{4. Источники верховной реальности}
\vs p115 4:1 Любое обсуждение \bibemph{истоков} Бога Верховного должно начинаться с Райской Троицы, ибо Троица является Божеством изначальным, тогда как Верховный --- Божество производное. Любое обсуждение \bibemph{роста} Верховного должно принимать во внимание экзистенциальные триодиты, ибо они заключают в себе всю абсолютную актуальность и всю бесконечную потенциальность (в соединении с Первоисточником и Центром). Причем эволюционирующий Верховный является кульминационным и личностно волевым центром превращения --- преобразования --- потенциального в актуальное на конечном уровне бытия. Два триодита, актуальный и потенциальный, заключают в себе всю совокупность взаимосвязей роста во вселенных.
\vs p115 4:2 Источник Верховного находится в Райской Троице --- вечном, актуальном и нераздельном Божестве. Верховный --- это прежде всего духовная личность, и эта духовная личность происходит от Троицы. Но, во\hyp{}вторых, Верховный является Божеством роста --- эволюционного роста --- и этот рост происходит от двух триодитов, актуального и потенциального.
\vs p115 4:3 Если вам трудно осознать, что бесконечные триодиты могут действовать на конечном уровне, приостановитесь и подумайте о том, что сама их бесконечность должна содержать в себе потенциальность конечного; бесконечность включает в себя все --- от низшего и самого ограниченного конечного бытия до высших и неограниченно абсолютных реальностей.
\vs p115 4:4 Осознать, что бесконечное содержит конечное, не так трудно, как понять то, как это бесконечное в действительности явлено конечному. Однако Настройщики Мысли, пребывающие в смертном человеке, являются одним из вечных доказательств того, что даже абсолютный Бог (как абсолют) действительно может войти в прямой контакт даже с низшими и наименьшими из всех наделенных волей творений во вселенной и делает это.
\vs p115 4:5 Триодиты, совокупно заключающие в себе актуальное и потенциальное, явлены на конечном уровне в соединении с Верховным Существом. Способы такого проявления бывают и прямыми, и косвенными: прямыми --- поскольку триодитные отношения имеют последствия непосредственно в Верховном, косвенными же --- поскольку они производны через выявленный уровень абсонитного.
\vs p115 4:6 Верховная реальность, являющаяся совокупной конечной реальностью, находится в процессе динамичного роста между неограниченным потенциальным внешнего пространства и неограниченным актуальным центра всех вещей. Сфера конечного, таким образом, становится фактически существующей благодаря сотрудничеству абсонитных сил Рая и Верховных Личностей\hyp{}Творцов, существующих во времени. Акт развития ограниченных возможностей трех великих потенциальных Абсолютов является абсонитной функцией Архитекторов Главной Вселенной и их трансцендентальных сподвижников. Причем тогда, когда эти возможности достигают определенной точки развития, Верховные Личности\hyp{}Творцы выходят из Рая, дабы принять участие в извечном деле фактического создания развивающихся вселенных.
\vs p115 4:7 Рост Верховенства происходит от триодитов; духовная личность Верховного --- от Троицы; однако прерогативы мощи Всемогущего основаны на божественных достижениях Бога Семеричного, тогда как соединение прерогатив власти Всемогущего Верховного с духовной личностью Бога Верховного происходит благодаря служению Носителя Объединенных Действий, даровавшего разум Верховного в качестве объединяющего фактора в этом эволюционном Божестве.
\usection{5. Отношение Верховного к Райской Троице}
\vs p115 5:1 В реальности своей личной и духовной природы Верховное Существо абсолютно зависимо от существования и действия Райской Троицы. Хотя рост Верховного является вопросом триодитных отношений, духовная личность Бога Верховного зависит и происходит от Райской Троицы, которая всегда остается абсолютным центром\hyp{}источником совершенной и бесконечной стабильности, вокруг которой постепенно разворачивается эволюционный рост Верховного.
\vs p115 5:2 Действие Троицы связано с действием Верховного, ибо Троица действует на всех (тотальных) уровнях, включая и уровень действия Верховенства. Однако как век Хавоны уступает место веку сверхвселенных, так и различимое действие Троицы как непосредственного творца уступает место творческим актам детей Райских Божеств.
\usection{6. Отношение Верховного к триодитам}
\vs p115 6:1 Триодит актуальности продолжает действовать непосредственно в постхавонские эпохи; гравитация Рая захватывает основные единицы материального бытия, духовная гравитация Вечного Сына действует непосредственно на основные ценности духовного бытия, а гравитация разума Носителя Объединенных Действий безошибочно захватывает все жизненно важные значения бытия интеллектуального.
\vs p115 6:2 Однако по мере того, как каждый этап творческой деятельности продолжает распространяться в неисследованном пространстве, он функционирует и существует во все большем и большем удалении от прямого действия творческих сил и божественных личностей центрального местоположения --- абсолютного Райского Острова и бесконечных Божеств, на нем пребывающих. Эти следующие друг за другом уровни космического бытия становятся поэтому все более зависимыми от происходящего внутри трех Абсолютных потенциальностей бесконечности.
\vs p115 6:3 Верховное Существо заключает в себе возможности для космического служения, которые не явлены очевидно в Вечном Сыне, Бесконечном Духе или безличных реальностях Райского Острова. Это утверждение сделано с надлежащим учетом абсолютности этих трех основных актуальностей, однако рост Верховного не только основан на этих актуальностях Божества и Рая, но и вовлечен в происходящее внутри Божественного, Вселенского и Неограниченного Абсолютов.
\vs p115 6:4 \pc Верховный растет не только по мере того, как Творцы и создания развивающихся вселенных достигают Богоподобия; это конечное Божество также переживает рост вследствие совершенного овладения созданием и Творцом конечными возможностями великой вселенной. Движение Верховного --- двоякое: интенсивное --- по направлению к Раю и Божеству и экстенсивное --- по направлению к безграничности Абсолютов потенциального.
\vs p115 6:5 В современную эпоху вселенной это двоякое движение явлено в восходящих и нисходящих личностях великой вселенной. Верховные Личности\hyp{}Творцы и все их божественные сподвижники отражают направленное наружу расходящееся движение Верховного, тогда как идущие по пути восхождения пилигримы из семи сверхвселенных свидетельствуют о направленной внутрь, сходящейся тенденции Верховенства.
\vs p115 6:6 Конечное Божество всегда стремится к двойной корреляции: внутренней, направленной к Раю и Божествам, в нем пребывающим, и наружной, устремленной к бесконечности и Абсолютам, находящимся в ней. Мощное извержение Райской творческой божественности, персонализирующейся в Сынах\hyp{}Творцах и проявляющих мощь в контролерах мощи, говорит об огромном выплеске Верховенства в сферы потенциальности, тогда как бесконечная процессия идущих по пути восхождения творений великой вселенной свидетельствует о мощном устремлении Верховенства по направлению к единству с Райским Божеством.
\vs p115 6:7 Люди узнали, что движение невидимого иногда можно различить по его результатам в видимом; и мы во вселенных давно научились определять движения и тенденции Верховенства путем наблюдения последствий подобных эволюций в личностях и паттернах великой вселенной.
\vs p115 6:8 Хоть мы и не уверены, но все же думаем, что в качестве конечного отражения Райского Божества Верховный участвует в вечном продвижении во внешнее пространство; однако как ограниченное выражение трех Абсолютных потенциалов внешнего пространства это Верховное Существо вечно стремится к Райской согласованности. Причем это двойное движение, видимо, объясняет большую часть основной деятельности в настоящее время формированных вселенных.
\usection{7. Природа Верховного}
\vs p115 7:1 В Божестве Верховного Отец\hyp{}Я ЕСТЬ достиг относительно полного освобождения от ограничений, присущих бесконечности состояния, вечности бытия и абсолютности природы. Однако Бог Верховный освободился от всех экзистенциальных ограничений, лишь став подверженным относящимся к опыту ограничениям действия во вселенной. Обретая способность к постижению опыта, конечный Бог становится также зависимым от необходимости в нем; достигая освобождения от вечности, Всемогущий сталкивается с барьерами времени, причем Верховный смог познать рост и развитие лишь вследствие незавершенности существования и неполноты природы, неабсолютности бытия.
\vs p115 7:2 Все это должно соответствовать замыслу Отца, который основал прогресс конечного на усилии, достижение творения --- на упорстве, развитие личности --- на вере. Однако предопределив таким образом основанную на опыте эволюцию Верховного, Отец дал конечным созданиям возможность существовать во вселенных и благодаря основанному на опыте развитию достигать в будущем божественности Верховенства.
\vs p115 7:3 \pc Включая Верховного и даже Предельного, вся реальность, за исключением неограниченных ценностей семи Абсолютов, относительна. Факт Верховенства основан на мощи Рая, личности Сына и действии Носителя Объединенных Действий, однако рост Верховного заключен в Божественном Абсолюте, Неограниченном Абсолюте и Вселенском Абсолюте. Причем это синтезирующее и объединяющее Божество --- Бог Верховный --- олицетворяет собой конечную тень, отбрасываемую на великую вселенную бесконечным единством неисследимой природы Райского Отца, Первоисточника и Центра.
\vs p115 7:4 В степени, в которой триодиты действуют непосредственно на конечном уровне, они сталкиваются и с Верховным, который является средоточием Божества и космическим итогом конечных ограничений природ Абсолютного Актуального и Абсолютного Потенциального.
\vs p115 7:5 \pc Райская Троица считается абсолютной неизбежностью; Семь Духов\hyp{}Мастеров, очевидно, являются неизбежностями Троицы; актуализация же Верховного в мощи, разуме, духе и личности должна быть неизбежностью эволюционной.
\vs p115 7:6 Бог Верховный, по\hyp{}видимому, не является неизбежным в неограниченной бесконечности, однако, очевидно, неизбежен на всех относительных уровнях. Он --- (незаменимо сосредоточивает, подытоживает и заключает в себе эволюционный опыт, эффективно объединяющий результаты этого рода восприятия реальности в своей Божественной природе. Причем кажется, что все это он делает с целью способствовать феномену \bibemph{неизбежного выявления,} сверхопыта и сверхконечного проявления Бога Предельного.
\vs p115 7:7 Верховное Существо нельзя полностью оценить, не приняв в соображение источник, действие и предназначение: отношение к порождающей его Троице, вселенной действия и Троице Предельной непосредственного предназначения.
\vs p115 7:8 Благодаря процессу обобщения эволюционного опыта Верховный соединяет конечное с абсонитным так же, как разум Носителя Объединенных Действий интегрирует божественную духовность личностного Сына с неизменными энергиями Райского паттерна, и так же, как присутствие Вселенского Абсолюта объединяет активацию Божества с реактивностью Неограниченного. Причем это единство должно быть откровением невидимого действия изначального единства Первого Отца\hyp{}Причины и Паттерна\hyp{}Источника всех вещей и всех существ.
\vsetoff
\vs p115 7:9 [При поддержке Могучего Вестника, временно пребывающего на Урантии.]
