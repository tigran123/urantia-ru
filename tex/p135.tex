\upaper{135}{Иоанн Креститель}
\author{Комиссия срединников}
\vs p135 0:1 Иоанн Креститель родился 25 марта 7 года до н.э., как и обещал Елизавете Гавриил в июне предыдущего года. Пять месяцев Елизавета хранила посещение Гавриила в тайне; когда же она рассказала о нем своему мужу Захарии, тот впал в большое беспокойство и до конца поверил ее рассказу только после необычного сна, который приснился ему приблизительно за шесть недель до рождения Иоанна. Кроме посещения Гавриилом Елизаветы и сна Захарии в рождении Иоанна Крестителя не было ничего необычного и сверхъестественного.
\vs p135 0:2 На восьмой день, по еврейскому обычаю, Иоанну сделали обрезание. Он рос обыкновенным ребенком, день за днем и год за годом, в небольшом селении, известном в те дни под названием Города Иуды и расположенном в четырех милях к западу от Иерусалима.
\vs p135 0:3 Самым большим событием раннего детства Иоанна было посещение им вместе со своими родителями Иисуса и назаретского семейства. Произошло оно в июне 1 года до н.э., когда Иоанну было чуть больше шести лет.
\vs p135 0:4 Вернувшись из Назарета, родители Иоанна начали давать мальчику систематическое образование. В маленьком селении не было синагогальной школы, но Захария был священником, а потому хорошо образованным человеком; намного образованней средней иудейской женщины была и Елизавета, ибо также была из рода священников, происходя от «дочерей Аарона». Поскольку Иоанн был единственным ребенком в семье, его умственному и духовному воспитанию уделялось очень много времени. Захария служил в Иерусалимском храме лишь в определенные, короткие периоды времени, и поэтому посвящал обучению сына большую часть своего времени.
\vs p135 0:5 Захария и Елизавета владели небольшим хозяйством, где они разводили овец. Доход от него едва позволял сводить концы с концами, но Захария регулярно получал содержание из храмовой казны для священников.
\usection{1. Иоанн становится назореем}
\vs p135 1:1 Иоанн не посещал школу и поэтому к четырнадцати годам не смог ее закончить, но родители решили, что это самый подходящий возраст, чтобы дать официальный обет назорея. Поэтому Захария и Елизавета отвезли сына в Ен\hyp{}Геди, город на берегу Мертвого моря. Здесь находился южный центр назорейского братства, и здесь торжественно и по всем правилам мальчика приняли в орден пожизненно. После совершения церемоний и произнесения обетов не употреблять опьяняющих напитков, не стричь волосы и воздерживаться от прикосновения к мертвым семейство отправилось в Иерусалим, где перед храмом Иоанн принес жертвы, которые требовались от тех, кто дал назорейские обеты.
\vs p135 1:2 Подобно своим знаменитым предшественникам --- Самсону и пророку Самуилу, Иоанн дал обет на всю жизнь. Человек, давший такой обет, почитался очищенным от пороков и святой личностью. Евреи относились к назореям почти с таким же уважением и благоговением, как к первосвященникам, в чем не было ничего удивительного, ибо назореи, принявшие пожизненное посвящение, были единственными людьми, кому наряду с первосвященниками, позволялось входить во святая святых храма.
\vs p135 1:3 \pc Вернувшись из Иерусалима домой, Иоанн стал пасти овец отца и вырос сильным человеком с благородным характером.
\vs p135 1:4 В шестнадцать лет Иоанн прочитал об Илии, и то, что он узнал о пророке с горы Кармил, произвело на него столь сильное впечатление, что Иоанн решил подражать ему в одежде. С тех пор он всегда носил власяницу и кожаный пояс. В шестнадцать лет его рост превышал шесть футов, и он казался почти взрослым. С длинными волосами и в необычном одеянии юноша выглядел поистине живописно. Родители же ожидали от единственного сына, обетованного ребенка и пожизненного назорея, великих свершений.
\usection{2. Смерть Захарии}
\vs p135 2:1 После нескольких месяцев болезни Захария умер в июле 12 года н.э., когда Иоанну исполнилось восемнадцать лет. Иоанн оказался в сложном положении, поскольку обет назорея запрещал ему приближаться к умершим, даже если те были членами его семьи. Хотя Иоанн постарался соблюсти табу данного им обета избегать осквернения мертвыми, но он сомневался в том, что сумел соблюсти требования назорейского устава в целом; поэтому после погребения отца он отправился в Иерусалим, где в назорейской части двора для женщин принес жертвы, необходимые для очищения.
\vs p135 2:2 \pc В сентябре того же года Елизавета и Иоанн совершили путешествие в Назарет, чтобы повидать Марию и Иисуса. Иоанн уже почти решился приступить к делу всей своей жизни, однако не только пример Иисуса, но и слова, сказанные им, убедили его вернуться домой и заботиться о матери, ожидая «когда настанет час Отца». Когда чудесное время, проведенное в гостях, подошло к концу, Иоанн попрощался с Марией и Иисусом и больше не видел Иисуса до самого крещения его в Иордане.
\vs p135 2:3 Возвратившись домой, Иоанн с Елизаветой стали строить планы на будущее. Поскольку Иоанн отказался принимать содержание священника, положенное ему из казны храма, через два года они утратили все, и даже чуть не лишились своего дома, и решили со стадом овец идти на юг. Поэтому лето того года, когда Иоанну исполнилось двадцать лет, застало их в Хевроне. Он пас своих овец в так называемой «пустыне иудейской» возле притока, который вливался в большой ручей, впадавший в Мертвое Море близ Ен\hyp{}Геди. В ен\hyp{}гедийской колонии жили не только назореи, принявшие посвящение на всю жизнь либо на время, но и множество других пастухов\hyp{}аскетов, которые собирались здесь вместе со своими стадами и общались с назорейским братством. Жили они разведением овец и пожертвованиями богатых евреев ордену.
\vs p135 2:4 Со временем Иоанн все реже возвращался в Хеврон и все чаще посещал Ен\hyp{}Геди. От большинства назореев он отличался настолько разительно, что ему было очень трудно в полной мере сблизиться с братством. Но он очень любил Авенира, признанного лидера и главу ен\hyp{}гедийской колонии.
\usection{3. Жизнь пастуха}
\vs p135 3:1 В долине вдоль ручья Иоанн построил не меньше дюжины каменных укрытий и ночных загонов, сделанных из наваленных друг на друга камней, где он мог пасти и охранять свои стада овец и коз. Жизнь пастуха оставляла Иоанну много времени для размышлений. Он часто беседовал с Ездой, мальчиком\hyp{}сиротой из Беф\hyp{}Цура, которого почти усыновил и который следил за стадами, когда Иоанн уходил в Хеврон повидаться с матерью и продать овец или отправлялся в Ен\hyp{}Геди на субботнюю службу. Иоанн и мальчик жили очень просто, питаясь бараниной, козьим молоком, диким медом и съедобной саранчой, которая водилась в той местности. Свой обычный рацион они дополняли провизией, которую время от времени приносили из Хеврона и Ен\hyp{}Геди.
\vs p135 3:2 \pc Елизавета постоянно извещала Иоанна о событиях в мире и в Палестине, и он все больше утверждался в своем убеждении, что старый порядок стремительно подходит к концу и что ему предстоит стать глашатаем приближения новой эры, «царства небесного». Этот суровый пастух был особенно увлечен писаниями пророка Даниила. Он тысячу раз читал потрясающее иносказание Даниила, которое, как сказал сыну Захария, символизировало историю великих мировых царств от Вавилона и Персии до Греции и Рима. Иоанн понимал, что в Римскую империю входило так много различных рас и народов, что она не могла стать прочным и единым государством. Он считал, что Рим уже тогда был разделен на Сирию, Египет и другие провинции. Далее Иоанн читал: «И во дни тех царств Бог Небесный воздвигнет царство, которое вовеки не разрушится, и царство это не будет передано другому народу; оно сокрушит и разрушит все царства, а само будет стоять вечно». «И Ему дана власть, слава и царство, чтобы все народы, племена и языки служили Ему; владычество Его --- владычество вечное, которое не прейдет, и царство Его не разрушится». «Царство же и власть и величие царственное во всей поднебесной дано будет народу святых Всевышнего, которого царство --- царство вечное, и все властители будут служить и повиноваться ему».
\vs p135 3:3 \pc Иоанн так и не сумел до конца разобраться в противоречии, существующем между тем, что он слышал об Иисусе от родителей, и этими словами, которые он прочел в Писании. В книге Даниила он читал: «Видел я в ночных видениях, вот, с облаками небесными шел как бы Сын человеческий и Ему дана власть, слава и царство.» Но эти слова пророка не согласовывались с тем, чему учили его родители. Не соответствовало утверждениям Писания и то, что говорил Иисус во время встречи с Иоанном, когда ему было восемнадцать лет. Однако вопреки всем противоречиям, все время, пока Иоанн пребывал в сомнениях, его мать не переставала уверять его, что его далекий двоюродный брат Иисус из Назарета --- истинный Мессия, что он пришел в мир, дабы воцариться на престоле Давида, и что он, Иоанн, должен стать глашатаем его пришествия и его главной опорой.
\vs p135 3:4 Все, что Иоанн слышал о пороках и развращенности Рима, о беспутстве и нравственном убожестве империи, все, что он знал о злодеяниях Ирода Антипы и правителей Иудеи, склоняло его к вере в приближение конца эпохи. Этому суровому и благородному сыну природы казалось, что мир подошел к концу эпохи человека и стоит на пороге новой божественной эры --- царства небесного. Сердце Иоанна наполнялось чувством, что ему предстоит стать последним из старых пророков и первым из новых. И он трепетал от неудержимого желания выйти и провозгласить всем людям: «Покайтесь! Сделайтесь праведными перед Богом! Будьте готовы к концу; приготовьтесь к наступлению царства небесного, нового и вечного порядка вещей на земле».
\usection{4. Смерть Елизаветы}
\vs p135 4:1 7 августа 22 года н.э., когда Иоанну было двадцать восемь лет, его мать неожиданно умерла. Зная о табу, запрещавших назорею прикасаться к мертвым, даже если те были членами их семьи, друзья Елизаветы перед тем, как послать за Иоанном, сделали все приготовления к ее погребению. Узнав о смерти матери, Иоанн велел Езде отвести свои стада в Ен\hyp{}Геди, а сам отправился в Хеврон.
\vs p135 4:2 Вернувшись в Ен\hyp{}Геди с похорон матери, Иоанн передал стада братству и на какой\hyp{}то период уединился от внешнего мира, проводя время в посте и молитве. Иоанн знал только старые способы приближения к божественному и был знаком лишь с писаниями таких учителей, как Илия, Самуил и Даниил. Илия был для него идеалом пророка. Он был первым учителем Израиля, которого можно считать пророком, и Иоанн искренне верил, что ему, Иоанну, предстоит стать последним в этой долгой и славной череде вестников небес.
\vs p135 4:3 Два с половиной года жил Иоанн в Ен\hyp{}Геди и убедил большую часть братства в том, что «приблизился конец эпохи», а «царство небесное при дверях». Вначале все его учение было основано на современных ему представлениях евреев о Мессии как обещанном освободителе еврейского народа от господства правителей\hyp{}неевреев.
\vs p135 4:4 Все это время Иоанн читал и многое узнал из священных писаний, которые нашел в Ен\hyp{}Геди, в доме назореев. Особенно сильное впечатление на него произвели Исайя и Малахия, последние из пророков на то время. Иоанн читал и перечитывал последние пять глав книги Исайи и верил его пророчествам. Однако в книге Малахии Иоанн читал: «Вот, я пошлю к вам Илию пророка перед наступлением дня Господня, великого и страшного. И Он обратит сердца отцов к детям и сердца детей к отцам их, чтобы я пришел и не поразил земли проклятием». Иоанна удерживало только это обещание Малахии о возвращении Илии; оно не давало ему выступить с проповедью о будущем царстве и с призывом к собратьям\hyp{}евреям бежать от грядущего гнева. Иоанн был готов провозгласить весть о будущем царстве, но ожидание пришествия Илии задержало его более, чем на два года. Иоанн знал, что он не Илия. О чем же говорил Малахия? Каким было его пророчество, буквальным или образным? Как ему узнать истину? Наконец осмелился предположить: поскольку первого из пророков называли Илией, то и последний из них должен быть известен под тем же именем. Тем не менее у него были сомнения --- сомнения, достаточные для того, чтобы он не называл себя Илией никогда.
\vs p135 4:5 Влияние Илии побудило Иоанна перенять его методы прямо и резко порицать грехи и пороки своих современников. Он старался одеваться, как Илия, говорить, как Илия, и внешне во всех отношениях походил на древнего пророка. Он был таким же сильным и колоритным сыном природы, таким же бесстрашным и отважным проповедником праведности. Иоанн не был неграмотным и, действительно, хорошо знал еврейские священные писания, но едва ли мог считаться культурным человеком. Он был блестящим мыслителем, ярким оратором и яростным обличителем. Иоанн едва ли являл собою пример своим современникам, но зато служил красноречивым упреком им.
\vs p135 4:6 Наконец Иоанн придумал, как провозгласить новую эру, царство Божье; он решил, что должен стать глашатаем Мессии, и, отбросив все сомнения, покинул Ен\hyp{}Геди в один из дней марта 25 года н.э., чтобы вступить на свой короткий, но яркий путь народного проповедника.
\usection{5. Царство Бога}
\vs p135 5:1 Чтобы понять послание Иоанна, необходимо иметь представление о положении еврейского народа в то время, когда Иоанн вышел на арену событий. Почти сто лет весь Израиль находился в сложном положении; евреи не знали, как объяснить, почему они постоянно оказываются в подчинении у правителей\hyp{}неевреев. Разве не учил их Моисей, что праведность всегда вознаграждается процветанием и могуществом? Разве не были они народом, избранным Богом? Почему же престол Давида оставлен и пустует? В свете учения Моисея и заповедей пророков евреям было трудно объяснить свое непрекращающееся бедственное положение как нации.
\vs p135 5:2 Лет за сто до Иисуса и Иоанна в Палестине возникла новая школа религиозных учителей, называвших себя апокалиптиками. Новые учителя выработали систему верований, согласно которым страдание и угнетение евреев были карой за грехи нации. В своих рассуждениях они пускали в ход старые и хорошо известные доводы, которыми объяснялись вавилонское и другие пленения прежних времен. Но, как учили апокалиптики, Израиль должен крепиться сердцем: дни страданий почти подошли к концу, наказание богоизбранного народа скоро закончится, а терпение Бога по отношению к иноземцам\hyp{}неевреям почти иссякло. Конец римского правления в их представлении был тождествен концу эры и, в определенном смысле, концу света. Новые учителя в основном полагались на предсказания Даниила и, соответственно, учили, что творение вот\hyp{}вот перейдет в свою конечную стадию, а царства этого мира очень скоро превратятся в царство Божье. Для сознания евреев того времени выражение «царство небесное» означало именно это; то же самое характерно для учений и Иоанна, и Иисуса. Для евреев Палестины выражение «царство небесное» имело только одно значение: это абсолютно праведное государство, в котором Бог (Мессия) будет управлять народами земли с такой же совершенной властью, с какой правит он и на небе, --- «Да будет воля Твоя и на земле, как на небе».
\vs p135 5:3 Во дни Иоанна все евреи с надеждой спрашивали: «Скоро ли наступит царство?» Их охватило общее чувство, что конец господства нееврейских наций вот\hyp{}вот наступит. Среди еврейства распространилась радостная надежда, что его вечное желание осуществится уже при жизни современного поколения, и все напряженно ожидали этого.
\vs p135 5:4 Хотя евреи во многом расходились в своем понимании природы будущего царства, они как один верили, что это событие произойдет очень скоро, что оно приблизилось и уже у дверей. Многие из тех, кто воспринимал Ветхий Завет буквально, ожидали нового царя Палестины, возрождения еврейского народа, освобожденного от врагов и объединенного вокруг потомка царя Давида, Мессии, которого быстро признают законным и праведным правителем всего мира. Другая, менее многочисленная группа благочестивых евреев представляла себе царство совершенно иначе. Они учили, что грядущее царство будет не от мира сего, что мир приближается к определенному концу и что «новое небо и новая земля» явятся при установлении царства Божьего, что царство это будет стоять вечно, что грех будет искоренен и что граждане нового царства обретут бессмертие, радуясь такому нескончаемому блаженству.
\vs p135 5:5 Все соглашались, что установлению нового царства на земле будет обязательно предшествовать полное очищение либо очистительное наказание. Буквалисты учили, что вскоре разразится всемирная война, в которой погибнут все неверующие, верующие же доживут до всемирной и вечной победы. Сторонники духовного толкования учили, что наступлению царства будет предшествовать великий суд Божий, который вынесет неправедным заслуженный приговор, подвергнув их наказанию и окончательному уничтожению, и в то же время возведет верующих святых избранного народа на высокие престолы славы и власти и усадит их рядом с Сыном Человеческим, который будет править обретшими искупление народами во имя Бога. Представители этой группы верили даже, что в новое царство могут быть допущены многие благочестивые неевреи.
\vs p135 5:6 Некоторые евреи придерживались мнения, что новое царство, возможно, будет установлено прямым и божественным вмешательством Бога, но подавляющее большинство верило, что вначале Бог пошлет своего доверенного посредника, Мессию. Именно это значение слова «Мессия» и было единственно возможным в представлении поколения евреев, современного Иоанну и Иисусу. \bibemph{Мессией} не мог быть тот, кто просто учил исполнять волю Бога либо провозглашал необходимость праведной жизни. Таких святых людей евреи называли \bibemph{пророками.} Мессия же должен был быть больше пророка, он должен был установить новое царство, царство Бога. Никто, кому сделать этого не удалось, не мог быть Мессией в традиционном для евреев понимании.
\vs p135 5:7 Кому же предстояло стать этим Мессией? И снова еврейские учителя расходились во мнениях. Старейшие из них придерживались концепции сына Давидова. Новые учителя утверждали: поскольку новое царство должно быть царством небесным, постольку и новый правитель может быть божественной личностью, тем, кто давно занимает место одесную Бога на небесах. И, как ни странно, те, кто имел подобные представления о правителе нового царства, смотрели на него не как на Мессию\hyp{}человека, или как на простого \bibemph{человека,} но как на «Сына Человеческого» --- Сына Божьего --- небесного Принца, давно ожидающего, чтобы принять бразды правления обновленной землей. Такова была религиозная основа еврейского мира, когда Иоанн выступил, провозглашая: «Покайтесь, ибо приблизилось царство небесное!»
\vs p135 5:8 Несложно догадаться, что заявление Иоанна о новом царстве имело не менее полудюжины значений в представлениях тех, кто слушал его страстную проповедь. Но какой бы смысл ни приписывали выражениям, употреблявшимся Иоанном, каждая из этих различных групп, ожидающих еврейского царства, заинтересовалась словами искреннего, полного энтузиазма, не стесняющегося в выражениях проповедника праведности и покаяния, который столь торжественно призывал своих слушателей «бежать от грядущего гнева».
\usection{6. Иоанн начинает проповедовать}
\vs p135 6:1 В начале марта 25 г. н.э. Иоанн отправился вдоль западного берега Мертвого моря, затем вверх по течению реки Иордан и остановился напротив Иерихона, у древней переправы, через которую перешли Иисус Навин и дети Израиля, впервые вступая в обетованную землю; перейдя на другую сторону реки, он обосновался неподалеку у подхода к переправе и начал проповедовать людям, переправлявшимся через реку туда и обратно. Эта переправа была самой оживленной на всем Иордане.
\vs p135 6:2 Всем, кто слышал Иоанна, было ясно, что он не просто проповедник. Подавляющее большинство тех, кто внимали этому странному человеку, пришедшему из пустыни иудейской, уходили с верой, что услышали голос пророка. Неудивительно, что души этих истомившихся в ожидании людей глубоко тронуло подобное событие. Никогда за всю историю еврейского народа благочестивые дети Авраама не жаждали так «утешения Израилева» и не так страстно ожидали «восстановления царства». Никогда в истории еврейского народа послание Иоанна: «Приблизилось царство небесное» не могло приобрести такого глубокого и всеобъемлющего значения, как тогда, когда он столь таинственным образом появился на берегу у южной переправы через Иордан.
\vs p135 6:3 Подобно Амосу, Иоанн был пастухом. Он одевался, как древний пророк Илия, извергал предупреждения и изливал предостережения «в духе и силе Илии». Неудивительно, что этот странный проповедник возбудил сильное волнение во всей Палестине, куда путешественники приносили известия о его проповеди у Иордана.
\vs p135 6:4 В деятельности проповедника\hyp{}назорея была еще одна новая особенность: каждого, кто верил ему, он крестил в Иордане «для прощения грехов». Хотя омовение для евреев было отнюдь не новой церемонией, им еще не приходилось видеть, чтобы это делалось так, как теперь Иоанн. Крещальный обряд уже давно использовался для приобщения прозелитов\hyp{}неевреев к тем, кто допускался во внешние дворы храма, но никогда еще совершить омовение в знак покаяния не предлагали самим евреям. Всего пятнадцать месяцев прошло от момента, когда Иоанн начал проповедовать и крестить, до его ареста и заключения по приказу Ирода Антипы, но за этот короткий период он окрестил более ста тысяч кающихся.
\vs p135 6:5 Четыре месяца Иоанн проповедовал у вифанийской переправы, а затем отправился на север вверх по течению Иордана. Десятки тысяч слушателей, некоторые из любопытства, но большинство --- побуждаемые неподдельным интересом, приходили послушать его изо всех частей Иудеи, Переи и Самарии. Некоторые приходили даже из Галилеи.
\vs p135 6:6 В мае того же года, когда Иоанн еще был у вифанийской переправы, священники и левиты послали к нему делегацию, дабы узнать, считает ли он себя Мессией и какой властью он проповедует. Иоанн ответил вопрошающим, говоря: «Пойдите и скажите пославшим вас, что вы слышали „глас вопиющего в пустыне“, как сказано пророком, который говорит: „Приготовьте путь Господу, прямыми сделайте стези ему; всякий дол да заполнится, и всякая гора и холм да понизятся, кривизны выпрямятся и неровные пути сделаются гладкими; и узрит всякая плоть спасение Божье“».
\vs p135 6:7 Иоанн был отважным, но бестактным проповедником. Однажды, когда он проповедовал и крестил на западном берегу Иордана, к нему пришла группа фарисеев и несколько саддукеев, чтобы креститься. Перед тем, как отвести их в воду, Иоанн, обращаясь к ним, сказал: «Кто внушил вам, подобно ехиднам, бежать от будущего гнева? Я буду крестить вас, но предупреждаю: принесите достойный плод искреннего покаяния, если получите прощение грехов ваших. Не говорите мне, что ваш отец --- Авраам. Объявляю вам, что Бог может из сих двенадцати камней, лежащих перед вами, воздвигнуть достойных детей Аврааму. Уже и секира при корне деревьев лежит. Всякое дерево, не приносящее доброго плода, будет срублено и брошено в огонь». (Двенадцать камней, о которых говорил Иоанн, по преданию были теми мемориальными камнями, которые воздвиг Иисус Навин в память о переходе в этом самом месте «двенадцатью коленами» Израиля реки Иордан при вступлении их в обетованную землю.)
\vs p135 6:8 Иоанн преподавал своим ученикам уроки, в ходе которых давал подробные наставления об их новой жизни и старался ответить на их многочисленные вопросы. Он советовал учителям учить в духе так же, как и в букве закона, и призывал богатых давать пищу бедным; сборщикам налогов Иоанн говорил: «Ничего не требуйте более определенного вам». Солдат призывал: «Никого не обижайте и ничего не требуйте незаконно --- довольствуйтесь своим жалованьем». Всем советовал: «Приготовьтесь к концу века сего --- приблизилось царство небесное».
\usection{7. Иоанн отправляется на север}
\vs p135 7:1 У Иоанна все еще было смутное представление о будущем царстве и его царе. Чем больше он проповедовал, тем большие сомнения одолевали его, однако эта интеллектуальная неопределенность относительно природы будущего царства ни в малейшей степени не повлияла на его уверенность, что оно наступит очень скоро. Умом Иоанн мог быть смятен, но душой --- никогда. Он не испытывал сомнений относительно будущего царства, но был далек от уверенности, должен ли Иисус стать его правителем. Пока что Иоанн придерживался идеи о восстановлении престола Давида, учения его родителей о том, что Иисус, рожденный в городе Давидовом, должен стать долгожданным избавителем, казались ему последовательными, однако, когда Иоанн больше склонялся к доктрине духовного царства и конца временной эры на земле, он погружался в глубокие сомнения относительно роли, которую предстояло сыграть Иисусу в этих событиях. Иногда он сомневался вообще во всем, но такие периоды были короткими. Иоанну действительно хотелось обо всем поговорить с кузеном, но это противоречило их ясно высказанной договоренности.
\vs p135 7:2 \pc В пути на север Иоанн много думал об Иисусе. Продвигаясь вверх по течению Иордана, он сделал остановки более чем в двенадцати местах. В Адаме в ответ на прямой вопрос: «Ты ли Мессия?», который задали ему ученики, он впервые упомянул о «другом, кто придет после меня», и продолжал: «После меня придет больший меня, у которого я не достоин развязать ремень сандалий. Я крещу вас водой, а он будет крестить вас Духом Святым. Лопата его в руке его, и он очистит гумно свое; он соберет пшеницу в житницу свою, а солому сожжет в судном огне».
\vs p135 7:3 В ответ на вопросы учеников Иоанн продолжал развивать свое учение, день за днем прибавляя к нему то новое, что в отличие от первого и загадочного призыва «Покайтесь и креститесь» приносило пользу и утешение. К этому времени к Иоанну устремились толпы из Галилеи и Десятиградия. Множество искренне уверовавших следовало за обожаемым учителем день за днем.
\usection{8. Встреча Иисуса и Иоанна}
\vs p135 8:1 К декабрю 25 года н.э., когда Иоанн, путешествуя вверх по Иордану, достиг окрестностей Пеллы, его слава распространилась по всей Палестине, а его деяния стали главной темой обсуждения во всех городах, окружающих Галилейское море. Иисус благосклонно говорил о послании Иоанна, и это побудило многих из Капернаума присоединиться к Иоаннову культу покаяния и крещения. В декабре, вскоре после того, как Иоанн начал проповедовать неподалеку от Пеллы, к нему явились рыбаки Иаков и Иоанн, сыновья Зеведеевы, и попросили его крестить их. Они посещали Иоанна раз в неделю и приносили Иисусу свежие новости о служении евангелиста.
\vs p135 8:2 Братья Иисуса Иаков и Иуда уже говорили о том, чтобы пойти креститься к Иоанну; теперь же, когда Иуда пришел в Капернаум на субботнюю службу, он и Иаков, выслушав речь Иисуса в синагоге, решили посоветоваться с ним о своих планах. Было это ночью в субботу 12 января 26 года н.э. Иисус попросил братьев отложить разговор, пообещав дать ответ на следующий день. В эту ночь он почти не спал, пребывая в тесном общении с Отцом Небесным. Иисус договорился пообедать с братьями и дать им совет о крещении у Иоанна. Утром в воскресенье Иисус, как обычно, работал на верфи. Иаков и Иуда пришли с едой и стали ждать его на складе, поскольку время полуденного перерыва еще не наступило, а Иисус, как они знали, был в этом отношении очень пунктуален.
\vs p135 8:3 Перед самым началом полуденного отдыха Иисус сложил инструменты, снял рабочий фартук и просто объявил троим работникам, находившимся с ним в одном помещении: «Мой час настал». Затем он вышел к своим братьям Иакову и Иуде и повторил: «Мой час настал, пошли к Иоанну». Все трое тотчас отправились в Пеллу, обедая на ходу. Было это в воскресенье 13 января. Ночь братья провели в Иорданской долине и к месту, где Иоанн крестил, пришли около полудня следующего дня.
\vs p135 8:4 \pc Иоанн как раз начал крестить тех, кто в этот день должен был пройти обряд. В очереди собралось уже много кающихся, когда Иисус с двумя братьями заняли свое место среди мужчин и женщин, поверивших проповеди Иоанна о приближающемся царстве. Иоанн узнавал об Иисусе от сыновей Зеведея. Он уже слышал о его высказываниях о своей проповеди и со дня на день ожидал увидеть его самого, но никак не думал, что ему придется приветствовать Иисуса среди тех, кто ожидал своей очереди креститься.
\vs p135 8:5 Поглощенный тем, чтобы исполнить быстро обряд крещения над столь большим числом новообращенных, Иоанн не замечал Иисуса до тех пор, пока Сын Человеческий не предстал перед ним. Когда Иоанн узнал Иисуса, церемония приостановилась, и Иоанн поприветствовал кузена по плоти, спросив его: «Но зачем, приветствуя меня, ты вошел в воду?» Иисус ответил: «Чтобы креститься от тебя». Иоанн сказал: «Это мне нужно от тебя креститься. Почему ты пришел ко мне?» Тогда Иисус прошептал Иоанну: «Допусти меня сейчас, ибо так надлежит нам показать пример моим братьям, стоящим здесь рядом со мной, и чтобы народ знал, что мой час настал».
\vs p135 8:6 \pc В голосе Иисуса чувствовалась уверенность и власть. Иоанн вострепетал от волнения и приготовился крестить в Иордане Иисуса из Назарета. Было это в полдень в понедельник 14 января 26 года н.э. Итак, Иоанн окрестил Иисуса и двух его братьев Иакова и Иуду. Окрестив их, Иоанн отпустил остальных, объявив, что возобновит крещение в полдень на следующий день. Когда народ стал расходиться, все четверо, стоя в воде, услышали странный звук, вслед за которым над головой Иисуса на мгновение возникло видение, и они услышали голос, глаголющий: «Сей есть Сын Мой возлюбленный, в котором Мое благоволение». Лицо Иисуса преобразилось изменилось, и, выйдя из воды, он молча оставил их и отправился в горы на восток. И никто не видел его сорок дней.
\vs p135 8:7 Иоанн пошел вслед за Иисусом и прошел столько, сколько потребовалось, чтобы рассказать ему историю, которую он столько раз слышал из уст матери, историю о том, как Гавриил посетил Марию, когда никто из них еще не родился. Сказав Иисусу: «Теперь я знаю точно: ты --- Избавитель», он не стал его более задерживать. Но Иисус ничего не ответил ему.
\usection{9. Сорок дней проповеди}
\vs p135 9:1 Когда Иоанн вернулся к своим ученикам (теперь у него было двадцать пять или тридцать учеников, которые постоянно были с ним), он застал их увлеченными беседой о том, что же произошло в связи с крещением Иисуса. Все еще больше удивились, когда Иоанн поведал историю о посещении Гавриилом Марии до рождения Иисуса и сообщил, что Иисус не ответил ему ни слова даже после того, как он об этом ему рассказал. В тот вечер дождя не было, и тридцать или более человек беседовали еще долго после того, как на небе высыпали звезды. Все думали о том, куда же ушел Иисус и когда они его снова увидят.
\vs p135 9:2 \pc После всего, что случилось в этот день, в проповеди Иоанна появились новые и более уверенные ноты, которые по\hyp{}новому возвещали грядущее царство и ожидаемого Мессию. Наступило напряженное время, сорок дней ожидания возвращения Иисуса. Но Иоанн продолжал проповедовать с великой силой; в это же время огромным толпам, собиравшимся вокруг Иоанна у Иордана, начали проповедовать и его ученики.
\vs p135 9:3 В течение этих сорока дней ожидания всюду распространялись многочисленные слухи, которые достигали даже Тивериады и Иерусалима. Тысячи людей стекались к лагерю Иоанна посмотреть на предполагаемого Мессию, но Иисуса им увидеть не удалось. Когда же ученики Иоанна объявили, что странный Божий человек ушел в горы, многие усомнились во всем, что рассказывали о нем.
\vs p135 9:4 Недели через три после того, как Иисус оставил их, в Пеллу прибыла новая делегация священников и фарисеев из Иерусалима. Они прямо спросили Иоанна, он ли Илия или пророк, обещанный Моисеем; и когда Иоанн сказал: «Нет», они осмелились спросить у него: «Ты ли Мессия?» Иоанн ответил «Нет». Тогда пришедшие из Иерусалима спросили: «Если ты не Илия, не пророк и не Мессия, то зачем ты крестишь народ и возбуждаешь волнение?» Иоанн ответил: «Пусть те, кто слышал меня и принял крещение от меня, говорят, кто я, но я объявляю вам: когда я крестил водой, среди нас стоял тот, кто вернется крестить вас Святым Духом».
\vs p135 9:5 Эти сорок дней для Иоанна и его учеников были трудным временем. Какое отношение имеет Иоанн к Иисусу? Возникали и обсуждались сотни вопросов. Стали проявлять себя политиканство и эгоизм. Ожесточенные споры возникали вокруг различных идей и понятий о Мессии. Должен ли он быть военачальником и царем из потомков Давида? Сокрушит ли он римские армии, как некогда сокрушил Иисус Навин ханаанские племена? Или же он явится установить духовное царство? Иоанн вместе с меньшинством обсуждавших подобные вопросы решили, что Иисус пришел установить царство небесное, хотя сам Иоанн не совсем ясно представлял себе, в чем будет заключаться эта миссия установления царства небесного.
\vs p135 9:6 Для Иоанна настали напряженные дни, и он молился о возвращении Иисуса. Некоторые из учеников Иоанна организовали отряды, которые должны были искать Иисуса, но Иоанн запретил им, сказав: «Наше время в руках Бога Небес, он сам направляет своего избранного Сына».
\vs p135 9:7 \pc Рано утром в субботу 23 февраля во время утренней трапезы Иоанн и его друзья посмотрели на север и увидели Иисуса, идущего к ним. Когда Иисус приблизился, Иоанн встал на большой камень и, возвысив свой звучный голос, сказал: «Вот Сын Божий, спаситель мира! Это о нем я говорил: „После меня придет тот, кто стал впереди меня, потому что был прежде меня“. Вот почему я пришел из пустыни призывать к покаянию и крестить водой, возвещая о том, что приблизилось царствие небесное. И ныне идет тот, кто будет крестить вас Духом Святым. Я видел божественный дух, нисходящий на этого человека, и слышал глас Божий, глаголивший: „Сей есть Сын Мой возлюбленный, в котором Мое благоволение“».
\vs p135 9:8 Иисус попросил всех продолжать трапезу, а сам сел за трапезу рядом с Иоанном, так как его братья Иаков и Иуда уже вернулись в Капернаум.
\vs p135 9:9 \pc Рано утром на следующий день Иисус покинул Иоанна и его учеников и отправился назад в Галилею. Уходя, он ни слова не сказал, когда они снова увидят его. На вопрос Иоанна о его собственной проповеди и миссии Иисус сказал лишь: «Отец мой будет руководить тобой ныне и в будущем, как руководил и в прошлом». Итак, на берегу Иордана расстались в то утро два великих человека, чтобы во плоти не встречаться уже никогда.
\usection{10. Путешествие Иоанна на юг}
\vs p135 10:1 Когда Иисус ушел на север в Галилею, Иоанн почувствовал, что его влечет вернуться на юг. Поэтому утром в воскресенье 3 марта он с немногими своими учениками отправился на юг. Приблизительно четверть ближайших сторонников Иоанна тем временем пошла в Галилею искать Иисуса. Иоанн исполнился печалью смятения. Он уже никогда не проповедовал так, как до крещения Иисуса. Каким\hyp{}то образом он почувствовал, что ответственность за будущее царство более не лежит на его плечах. Он чувствовал, что его служение почти закончилось, был безутешен и одинок. Однако он проповедовал, крестил и продолжал идти на юг.
\vs p135 10:2 Неподалеку от селения Адам Иоанн остановился на несколько недель; именно здесь он выступил со своим знаменитым обличением Ирода Антипы, незаконно взявшего себе жену другого человека. К июню того же (26 г. н.э.) года Иоанн вернулся в Вифанию к переправе через Иордан, где он начал проповедовать о будущем царстве более года назад. В течение нескольких недель после крещения Иисуса его проповедь постепенно обрела характер провозглашения милости к простым людям и обличения с новой силой продажных политических и религиозных правителей.
\vs p135 10:3 Ирод Антипа, на чьей территории проповедовал Иоанн, встревожился, что он со своими учениками может поднять восстание. Ирод был также недоволен, что Иоанн публично осуждает его семейную жизнь. Поэтому Ирод решил заключить Иоанна в темницу, и ранним утром 12 июня до прибытия собравшихся послушать проповедь и посмотреть обряд крещения, агенты Ирода арестовали его. Шли недели, а Иоанна все не отпускали, и его ученики рассеялись по всей Палестине; многие из них ушли в Галилею, чтобы присоединиться к последователям Иисуса.
\usection{11. Иоанн в темнице}
\vs p135 11:1 В темнице Иоанн испытывал чувство горечи и одиночества. Нескольким его ученикам позволили приходить к нему на свидание. Иоанн хотел увидеть Иисуса, но вынужден был довольствоваться рассказами о его деяниях тех своих последователей, кто уверовал в Сына Человеческого. Он часто впадал в искушение и начинал сомневаться в Иисусе и его миссии. Если Иисус --- Мессия, то почему он ничего не делает, чтобы освободить его из невыносимого заточения? Более полутора лет томился суровый скиталец Божий в презренной темнице. И это время было великим испытанием его веры в Иисуса и верности ему. Более того, это было великим испытанием веры Иоанна в самого Бога. Много раз он впадал в искушение и начинал сомневаться в истинности своей собственной миссии и образа действий.
\vs p135 11:2 \pc Когда Иоанн провел в темнице несколько месяцев, к нему пришла группа учеников; рассказав ему об публичном служении Иисуса, они сказали: «Итак, Учитель, как видишь, тот, кто был с тобой в верховьях Иордана, благоденствует и принимает всех, кто приходит к нему. Он даже ест с мытарями и грешниками. Ты смело свидетельствовал о нем, а он ничего не делает, чтобы тебя спасти». Но Иоанн ответил друзьям: «Этот человек не может ничего делать, если не велено ему Отцом Небесным. Вы хорошо помните, как я говорил: „Я не Мессия, но тот, кого послали приготовить путь для него“. Это я сделал. Имеющий невесту есть жених; друг же его, находящийся с ним рядом и внимающий ему, радуется великой радостью, потому что слышит голос жениха. Моя радость, стало быть, исполнилась. Он должен возрастать, а я --- умаляться. Я --- человек земной и возвестил свое послание. Иисус же из Назарета пришел с неба. Он превыше всех нас. Сын Человеческий пришел от Бога, он провозгласит вам слова Бога. Ибо Отец не мерою дает дух Своему Сыну. Отец любит своего сына и все отдаст в руки его. Верующий в сына имеет вечную жизнь. Слова, которые я говорю вам, истинны, не отступайте от них».
\vs p135 11:3 \pc Учеников Иоанна настолько изумило его высказывание, что они ушли в полном молчании. Иоанн был также сильно взволнован, ибо понял, что изрек пророчество. Он больше ни разу не усомнился в миссии Иисуса и его божественности. Но Иоанн был огорчен тем, что Иисус не передал ему ни единого слова, что он сам не пришел повидаться с ним и не воспользовался своей великой силой, чтобы освободить его из темницы. Но Иисус об этом все знал. Он чувствовал к Иоанну большую любовь, но, сознавая свою божественную природу, зная все о великой награде, которая будет уготована Иоанну, когда тот покинет этот мир, а также о том, что земная миссия Иоанна исполнена, сдерживал себя и не вмешивался в естественное течение событий жизни великого проповедника и пророка.
\vs p135 11:4 \pc Столь долгое состояние неопределенности в темнице было для человека непереносимым. Всего за несколько дней до своей смерти Иоанн вновь послал нескольких человек, которым доверял, к Иисусу с вопросом: «Сделал ли я свое дело? Почему я томлюсь в темнице? Ты действительно Мессия, или нам ждать другого?» Когда же эти ученики передали послание Иоанна Иисусу, Сын Человеческий ответил: «Пойдите к Иоанну и скажите ему, что я не забыл о нем, но допускаю это, ибо так надлежит нам исполнить всякую добродетель. Скажите Иоанну, что видели и слышали --- бедным проповедано утешение --- и, наконец, скажите возлюбленному глашатаю моей земной миссии, что в грядущую эпоху он будет благословлен с избытком, если не усомнится и не впадет в заблуждение обо мне». Это было последние известие, которое Иоанн получил от Иисуса. Оно очень утешило его и сильно укрепило его веру, подготовив к трагическому концу его жизни во плоти, который последовал столь скоро после этого памятного события.
\usection{12. Смерть Иоанна Крестителя}
\vs p135 12:1 Поскольку во время ареста Иоанн проповедовал на юге Переи, его немедленно посадили в тюрьму крепости Махерон, где он и пробыл в заточении до самой казни. Ирод управлял как Галилеей, так и Переей, и в это время имел резиденции и в Юлии, и в Махероне в Перее. В Галилее же официальную резиденцию перенесли из Сефориса в новую столицу Тивериаду.
\vs p135 12:2 Ирод боялся отпустить Иоанна, опасаясь, как бы он не стал призывать к восстанию. Боялся он и казнить его, поскольку тогда бунт мог вспыхнуть в столице, ибо тысячи жителей Переи верили, что Иоанн --- святой человек, пророк. Поэтому Ирод держал проповедника\hyp{}назорея в темнице, не зная, как поступить с ним. Ирод несколько раз говорил с Иоанном, но тот не согласился ни покинуть его владения, ни отказаться от публичной деятельности, если его освободят. Новое же связанное с Иисусом из Назарета волнение, сила которого непрерывно возрастала, еще больше убедило Ирода в том, что отпускать Иоанна еще нельзя. К тому же Иоанн был жертвой сильной и жгучей ненависти со стороны Иродиады, незаконной жены Ирода.
\vs p135 12:3 Множество раз Ирод беседовал с Иоанном о царстве небесном, и хотя послание Иоанна порой производило на него сильное впечатление, но он боялся освободить Иоанна из тюрьмы.
\vs p135 12:4 Поскольку в Тивериаде еще продолжалось большое строительство, Ирод довольно много времени проводил в своих перейских резиденциях, особое предпочтение отдавая крепости Махерон. Прошло еще несколько лет, прежде чем строительство всех общественных зданий и официальной резиденции в Тивериаде закончилось.
\vs p135 12:5 \pc По случаю своего дня рождения Ирод устроил в Махеронском дворце великое пиршество для высших офицеров и других высокопоставленных особ из правительственных советов Галилеи и Переи. Так как добиться смерти Иоанна прямым обращением к Ироду Иродиаде не удалось, она решила прибегнуть к коварному замыслу.
\vs p135 12:6 Во время вечерних празднеств и увеселений Иродиада вывела свою дочь танцевать перед пирующими. Ироду очень понравилось выступление девицы, он подозвал ее к себе и сказал: «Ты очаровательна. Я очень доволен тобой. Сегодня мой день рождения, проси у меня чего пожелаешь, и я дам тебе до половины царства». Все это Ирод сделал, потому что был сильно пьян. Юная девушка отошла в сторону, чтобы узнать у матери, чего ей просить у Ирода. Иродиада сказала: «Иди к Ироду и проси голову Иоанна Крестителя». Вернувшись к праздничному столу, молодая женщина сказала Ироду: «Я прошу, чтобы ты тотчас подал мне на блюде голову Иоанна Крестителя».
\vs p135 12:7 Ирод исполнился страха и печали, но данная им клятва, а также присутствие всех, кто вместе с ним ел и пил, не позволили ему отказать в выполнении этой просьбы. Ирод Антипа призвал к себе солдата и приказал ему принести голову Иоанна. Так в эту ночь Иоанн был в тюрьме обезглавлен, и солдат принес на блюде голову пророка и отдал ее молодой женщине в дальней части трапезной. Девица же передала блюдо матери. Услышав об этом, ученики Иоанна пришли в тюрьму за телом Иоанна и, положив его в гробницу, они пошли и рассказали Иисусу.
