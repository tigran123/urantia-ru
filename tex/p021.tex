\upaper{21}{Райские Сыны\hyp{}Творцы}
\author{Совершенствователь Мудрости}
\vs p021 0:1 Сыны\hyp{}Творцы --- создатели и правители локальных вселенных пространства и времени. Эти вселенские творцы и владыки имеют двуединое происхождение, заключая в себе черты Бога Отца и Бога Сына. Но каждый Сын\hyp{}Творец отличается от любого другого; каждый уникален в природе так же, как и в личности; каждый является «единородным Сыном» совершенного божественного идеала своего происхождения.
\vs p021 0:2 В громадной работе по организации, развитию и совершенствованию локальной вселенной эти высокие Сыны всегда пользуются поддерживающим их одобрением Отца Всего Сущего. Связь Сынов\hyp{}Творцов с их Райским Отцом трогательна и восхитительна. Нет сомнения, что глубокая привязанность Божеств\hyp{}родителей к их божественному потомству является источником той самой прекрасной и почти божественной любви, которую даже смертные родители питают к своим детям.
\vs p021 0:3 Эти первичные Райские Сыны персонализированы как Михаилы. Как только они выходят из Рая и идут, чтобы основать свои вселенные, они становятся известны как Творцы\hyp{}Михаилы. Когда они облечены верховной властью, они называются Мастерами\hyp{}Михаилами. Иногда мы говорим о владыке вашей вселенной Небадона как о Христе\hyp{}Михаиле. Всегда и навечно они правят по «чину Михаила», что является названием первого Сына их чина и природы.
\vs p021 0:4 \pc Изначальный или перворожденный Михаил никогда не претерпевал воплощения как материальное существо, но семь раз он прошел через опыт восхождения духовного создания на семи контурах Хавоны, продвигаясь от внешних сфер к самому внутреннему контуру центрального творения. Чин Михаила знает великую вселенную от края и до края; не существует ни одного значительного события, переживаемого кем\hyp{}нибудь из детей пространства и времени, в котором Михаилы лично не принимали бы участие; они в действительности разделяют не только божественную природу, но также и вашу природу, значит, природы всех видов --- от наивысшей до наинизшей.
\vs p021 0:5 Изначальный Михаил --- глава\hyp{}председатель первичных Райских Сынов, когда они собираются на конференцию в центре всех вещей. Не так давно на Уверсе мы записали вселенское возвещение о чрезвычайном конклаве на вечном Острове, когда сто пятьдесят тысяч Сынов\hyp{}Творцов собрались в присутствии родителей и обсуждали дела, касающиеся развития, объединения и стабилизации вселенной вселенных. Это была избранная группа Владык\hyp{}Михаилов, Сынов семеричного пришествия.
\usection{1. Происхождение и природа Сынов\hyp{}Творцов}
\vs p021 1:1 Когда полнота абсолютного духовного замысла в Вечном Сыне соединяется с полнотой концепции личности в Отце Всего Сущего, когда такое творческое единение окончательно и полностью достигнуто, когда имеет место такая абсолютная идентичность духа и такое бесконечное единство концепции личности, тогда, именно тогда и там, не утрачивая ничего личностного или прерогатив любого из этих бесконечных Божеств, возникает, как вспышка, и мгновенно превращается в полностью сформировавшееся существо новый изначальный Сын\hyp{}Творец, единородный Сын совершенного идеала и мощной идеи, объединение которых порождает личность этого нового творца, личность мощи и совершенства.
\vs p021 1:2 Каждый Сын\hyp{}Творец является единородным и единственно могущим быть рожденным потомком совершенного объединения изначальных понятий двух бесконечных, вечных и совершенных разумов всегда существующих Творцов вселенной вселенных. Никогда не может быть другого такого Сына, потому что каждый Сын\hyp{}Творец есть неограниченное, завершенное и окончательное выражение и воплощение всего, что представляет собой каждая фаза каждой черты каждой возможности каждой божественной реальности, которая --- в течение всей вечности --- могла бы когда\hyp{}либо быть обнаружена или выражена при помощи этих божественных творческих потенциалов, объединившихся, чтобы породить этого Сына\hyp{}Михаила, или которая могла бы развиться из таких потенциалов. Каждый Сын\hyp{}Творец является абсолютом объединенных божественных понятий, которые составляют его божественное происхождение.
\vs p021 1:3 Божественная природа этих Сынов\hyp{}Творцов, в принципе, выводится равным образом из атрибутов обоих Райских родителей. Все разделяют полноту божественной природы Отца Всего Сущего и творческие прерогативы Вечного Сына, но мы, поскольку наблюдаем практический результат деятельности Михаилов во вселенной, обнаруживаем явные различия. Некоторые Сыны\hyp{}Творцы кажутся более похожими на Бога Отца, другие --- на Бога Сына. Например, тенденция управления во вселенной Небадона говорит о том, что природа и характер ее Создателя и правящего Сына больше напоминают природу и характер Вечного Сына\hyp{}Матери. Далее, необходимо подчеркнуть, что какие\hyp{}то вселенные возглавляют Райские Михаилы, равно напоминающие Бога Отца и Бога Сына. И эти наблюдения ни в коей мере не означают критику; это просто констатация факта.
\vs p021 1:4 Я не знаю точное число существующих Сынов\hyp{}Творцов, но у меня есть веские основания полагать, что их более семисот тысяч. Нам известно теперь, что существует ровно семьсот тысяч Объединяющих Дней и что больше их не создается. Заметим также, что согласно предписанным планам нынешнего вселенского периода, по\hyp{}видимому, в каждой локальной вселенной в качестве советника\hyp{}посланца Троицы назначено по одному Объединяющему Дней. Далее мы видим, что постоянно увеличивающееся число Сынов\hyp{}Творцов уже превзошло неизменное установленное число Объединяющих Дней. Но мы никогда не были осведомлены о предназначении тех Михаилов, которые выходят за пределы семисот тысяч.
\usection{2. Творцы локальных вселенных}
\vs p021 2:1 Райские Сыны первичного чина являются проектировщиками, творцами, строителями и руководителями своих соответствующих владений, локальных вселенных пространства и времени, основных творческих единиц семи развивающихся сверхвселенных. Сыну\hyp{}Творцу разрешено выбирать в пространстве область своей будущей космической деятельности, но прежде чем приступить к физическому формированию своей вселенной, он должен в течение длительного времени наблюдать и изучать результаты усилий его старших братьев в различных творениях, которые расположены в сверхвселенной его предполагаемых действий. А перед этим Сын\hyp{}Михаил завершит свой продолжительный и уникальный опыт наблюдения в Раю и обучения в Хавоне.
\vs p021 2:2 \pc Когда Сын\hyp{}Творец отправляется из Рая, чтобы приступить к созданию вселенной, чтобы стать главой --- по существу, Богом --- локальной вселенной, организованной им самим, здесь в первый раз он вступает в тесный контакт с Третьим Источником и Центром и во многих отношениях оказывается зависящим от него. Бесконечный Дух, пусть и оставаясь с Отцом и Сыном в центре всех вещей, предназначен действовать как реальный и эффективный помощник каждого Сына\hyp{}Творца. Следовательно, каждого Сына\hyp{}Творца сопровождает Творческая Дочь Бесконечного Духа --- существо, которому предназначено стать Божественной Служительницей, Духом\hyp{}Матерью новой локальной вселенной.
\vs p021 2:3 Отбытие Сына\hyp{}Михаила в этом случае навсегда освобождает его прерогативы творца от Райских Источников и Центров, эти прерогативы подпадают лишь под некоторые ограничения, присущие предсуществованию этих Источников и Центров и некоторым другим предшествующим присутствиям и видам мощи. К этим ограничениям прерогатив всемогущего во всех иных отношениях творца --- Отца локальной вселенной относятся:
\vs p021 2:4 \ublistelem{1.}\bibnobreakspace \bibemph{Энергией\hyp{}материей} распоряжается Бесконечный Дух. Прежде чем может быть создана любая форма вещей, больших или малых, прежде чем могут быть предприняты любые новые превращения энергии\hyp{}материи, Сын\hyp{}Творец должен обеспечить согласие и творческое сотрудничество Бесконечного Духа.
\vs p021 2:5 \ublistelem{2.}\bibnobreakspace \bibemph{Планы и типы созданий} контролируются Вечным Сыном. Прежде чем Сын\hyp{}Творец может заняться сотворением любого нового типа созданий, любым новым планом создания, он должен обеспечить согласие Вечного и Изначального Сына\hyp{}Матери.
\vs p021 2:6 \ublistelem{3.}\bibnobreakspace \bibemph{Личность} задумывается и одаряется Отцом Всего Сущего.
\vs p021 2:7 \pc Типы и паттерны \bibemph{разума} определяются факторами бытия, которые предшествуют созданию. После того, как эти факторы соединились, чтобы образовать создание (обладающее личностью или нет), разум становится даром Третьего Источника и Центра, вселенского источника служения разума для всех существ, находящихся ниже уровня Райских Творцов.
\vs p021 2:8 \pc Контроль за планами и типами \bibemph{духа} зависит от уровня их выражения. В конечном счете, духовные планы контролируются Троицей или духовными дарами личностей Троицы, существовавшими до Троицы, --- Отцом, Сыном и Духом.
\vs p021 2:9 \pc Когда такой совершенный и божественный Сын вступил во владение областью пространства избранной им вселенной; когда решены первоначальные проблемы вселенской материализации и предварительного равновесия, когда он вошел в эффективный и обеспечивающий творческое сотрудничество союз с дополняющей его Дочерью Бесконечного Духа --- тогда этот Вселенский Сын и этот Вселенский Дух вступают во взаимодействие, которому предназначено породить бесчисленные сонмы детей их локальной вселенной. В связи с этим событием Творческий Дух, являющийся средоточием райского Бесконечного Духа, преобразует свою природу и приобретает личностные качества Духа\hyp{}Матери локальной вселенной.
\vs p021 2:10 Несмотря на то, что все Сыны\hyp{}Творцы божественно подобны своим Райским родителям, ни один не похож в точности на другого; каждый является уникальным, отличным, исключительным и изначальным по \bibemph{природе} и как личность. А так как они --- архитекторы и исполнители планов жизни своих соответствующих владений, то благодаря этим самым различиям владения также будут различаться в каждой форме и фазе своего живого производного от Михаилов существования, которое может быть создано или развито в этих владениях впоследствии. Поэтому чины созданий, исконно существующих в локальных вселенных, весьма различны. Нет двух вселенных, управляемых или населенных исконными существами двуединого происхождения, которые были бы во всех отношениях идентичны. В пределах любой сверхвселенной половина из присущих им атрибутов, произведенных от единообразных Творческих Духов, совершенно одинакова, другая половина атрибутов различна, так как произведена от различных Творцов\hyp{}Сынов. Но такое различие не характерно ни для тех созданий, которые происходят только от Творческого Духа, ни для перемещенных существ, которые являются исконными жителями центральной вселенной или сверхвселенных.
\vs p021 2:11 \pc Когда Сын\hyp{}Михаил отлучается из своей вселенной, ее правительство возглавляет перворожденное исконное существо, Яркая и Утренняя Звезда, главный распорядитель локальной вселенной. Рекомендации и советы Объединяющего Дней в это время неоценимы. Во время этих отсутствий Сын\hyp{}Творец может наделять связанную с ним Дух\hyp{}Мать сверхконтролем своего духовного присутствия в обитаемых мирах и в сердцах смертных детей. И Дух\hyp{}Мать локальной вселенной всегда остается в ее центре, распространяя свою нежную заботу и духовное служение до самых отдаленных частей такого эволюционирующего владения.
\vs p021 2:12 Личное присутствие Сына\hyp{}Творца в его локальной вселенной не является необходимым для плавного хода развития учрежденного материального творения. Такие Сыны могут совершать путешествие в Рай, и все же их вселенные продолжают двигаться сквозь пространство. Они могут сложить с себя свои полномочия, чтобы воплотиться в детей времени; и все же их миры продолжают обращаться вокруг соответствующих центров. Не существует материальной структуры, независимой от абсолютно\hyp{}гравитационной власти Рая или от космического сверхконтроля присущего пространственному присутствию Неограниченного Абсолюта.
\usection{3. Владычество в локальной вселенной}
\vs p021 3:1 Сыну\hyp{}Творцу область вселенной дается с согласия Райской Троицы и по утверждению руководящим Духом\hyp{}Мастером соответствующей сверхвселенной. Такое действие утверждает право физического обладания, космической аренды. Но возвышение Сына\hyp{}Михаила с этой начальной и им самим ограниченной стадии правления до практического верховенства владычества, достигнутого им самим, происходит в результате его собственного личного опыта, полученного в работе сотворения вселенной и в результате опыта воплощения\hyp{}пришествия. До достижения владычества, заработанного пришествием, он правит как наместник Отца Всего Сущего.
\vs p021 3:2 \pc Сын\hyp{}Творец мог бы утвердить полное владычество над своим личным творением в любое время, но он мудро решает не делать этого. Если прежде, чем пройти через пришествия в образе созданий, он принял бы незаслуженное верховное владычество, Райские личности, постоянно пребывающие в его локальной вселенной, покинули бы ее. Но во всех творениях пространства и времени этого никогда не случалось.
\vs p021 3:3 Факт творчества подразумевает полноту владычества, но Михаилы предпочитают \bibemph{заслужить} его посредством приобретения опыта, тем самым обеспечивая полное сотрудничество всех Райских личностей, приданных руководству локальной вселенной. Мы не знаем ни одного Михаила, который когда\hyp{}нибудь поступал иначе; но любой из них может поступать иначе, они являются Сынами, истинно обладающими свободой воли.
\vs p021 3:4 \pc Владычество Сына\hyp{}Творца в локальной вселенной проходит через шесть, возможно --- семь, стадий выражения, основанного на опыте. Они появляются в следующем порядке:
\vs p021 3:5 \ublistelem{1.}\bibnobreakspace Начальное владычество наместника --- единоличная временная власть, осуществляемая Сыном\hyp{}Творцом до приобретения связанным с ним Творческим Духом личностных качеств.
\vs p021 3:6 \ublistelem{2.}\bibnobreakspace Объединенное владычество наместника --- совместное правление Райской пары после достижения личности Вселенской Духом\hyp{}Матерью.
\vs p021 3:7 \ublistelem{3.}\bibnobreakspace Увеличивающееся владычество наместника --- возрастающая власть Сына во время семи его пришествий в образе существа.
\vs p021 3:8 \ublistelem{4.}\bibnobreakspace Верховное владычество --- установленная власть вслед за завершением седьмого пришествия. В Небадоне верховное владычество ведет свое начало от завершения пришествия Михаила на Урантию. Оно существует чуть более девятнадцати столетий вашего планетарного времени.
\vs p021 3:9 \ublistelem{5.}\bibnobreakspace Увеличивающееся верховное владычество --- возросшая связь, вытекающая из установления большинства миров, населенных созданиями, в свете и жизни. Эта стадия относится к недостигнутому будущему вашей локальной вселенной.
\vs p021 3:10 \ublistelem{6.}\bibnobreakspace Тринитарное владычество --- осуществляемое вслед за установлением всей локальной вселенной в свете и жизни.
\vs p021 3:11 \ublistelem{7.}\bibnobreakspace Нераскрытое владычество --- неизвестные связи будущего вселенского периода.
\vs p021 3:12 \pc Принимая начальное владычество наместника над предполагаемой локальной вселенной, Творец\hyp{}Михаил дает клятву Троице не принимать на себя верховное владычество до тех пор, пока не будут завершены семь пришествий в образе созданий и пока они не будут удостоверены правителями сверхвселенной. Но если Сын\hyp{}Михаил не мог бы самостоятельно утвердить такое незаработанное владычество, тогда не было бы никакого смысла давать клятву не поступать таким образом.
\vs p021 3:13 Даже в периоды, предшествующие пришествию, Сын\hyp{}Творец почти верховно правит своим владением, когда ни в одной из его частей не существует раскола. Ограниченное правление вряд ли бы проявилось, если бы владычество никогда не оспаривалось. Владычество, осуществляемое Сыном\hyp{}Творцом до пришествия во вселенной, где нет бунта, не больше, чем во вселенной, где происходит бунт, но в первом случае ограничения владычества не видны, а во втором --- видны.
\vs p021 3:14 Если когда\hyp{}нибудь власти или руководству Сына\hyp{}Творца бросают вызов, на них нападают или им угрожает опасность, он навечно связан обещанием поддерживать, защищать, оборонять свое личное творение и, если нужно, вернуть его себе. Такие Сыны могут быть обеспокоены или встревожены лишь созданиями, которых они сами сотворили, или же --- высшими существами, которых они сами выбрали. Можно сделать вывод, что «высшие существа», те, кто по происхождению принадлежат уровням, находящимся выше локальной вселенной, едва ли будут причиной тревоги для Сына\hyp{}Творца, и это правда. Хотя они могли бы решиться на такой поступок. Добродетель есть волеизъявление личности; праведность не появляется автоматически в созданиях, обладающих свободой воли.
\vs p021 3:15 Перед завершением пути пришествия Сын\hyp{}Творец правит, в какой\hyp{}то мере ограничивая свое владычество, но по завершению своего служения\hyp{}пришествия он правит благодаря своему реальному опыту в облике и подобии своих многочисленных созданий. Когда Творец семь раз прожил среди своих созданий, когда путь пришествия завершен, тогда он утверждается в верховную вселенскую власть; он становится Сыном\hyp{}Мастером, владыкой и верховным правителем.
\vs p021 3:16 \pc Процесс получения верховного владычества над локальной вселенной включает семь опытных ступеней:
\vs p021 3:17 \ublistelem{1.}\bibnobreakspace Пройти на опыте через семь уровней бытия созданий посредством воплощения\hyp{}пришествия в точном подобии созданий соответствующего уровня.
\vs p021 3:18 \ublistelem{2.}\bibnobreakspace Осуществить на опыте посвящение каждой фазе семеричной воли Райского Божества, которое персонифицировано в Семи Духах\hyp{}Мастерах.
\vs p021 3:19 \ublistelem{3.}\bibnobreakspace Пройти каждое из семи опытных переживаний на уровнях созданий одновременно с исполнением одного из семи посвящений воле Райского Божества.
\vs p021 3:20 \ublistelem{4.}\bibnobreakspace На каждом уровне созданий, посредством опыта представить Райскому Божеству и всем вселенским разумным существам кульминацию жизни создания.
\vs p021 3:21 \ublistelem{5.}\bibnobreakspace На каждом уровне созданий, посредством опыта проявить одну фазу семеричной воли Божества уровню пришествия и всей вселенной.
\vs p021 3:22 \ublistelem{6.}\bibnobreakspace Посредством опыта объединить семеричный опыт созданий с семеричным опытом посвящения откровению природы и воли Божества.
\vs p021 3:23 \ublistelem{7.}\bibnobreakspace Достичь новой и более высокой связи с Верховным Существом. Последствия суммы этого опыта Творца и создания увеличивают сверхвселенскую реальность Бога Верховного и пространственно\hyp{}временное владычество Всемогущего Верховного и делает фактом верховное владычество Райского Михаила в локальной вселенной.
\vs p021 3:24 \pc Разрешая проблему владычества в локальной вселенной, Сын\hyp{}Творец не только демонстрирует свою собственную годность к правлению, но и открывает природу и выражает семеричную позицию Райских Божеств. Конечное понимание и признание созданиями главенства Отца проявляется в пришествии Сына\hyp{}Творца, когда он снисходит до того, чтобы принять на себя облик и переживания своих созданий. Эти первичные Райские Сыны --- подлинные открыватели любящей природы Отца и его благодетельной власти, того самого Отца, который в союзе с Сыном и Духом является вселенским главой всей мощи, всех личностей и правительств во всех мирах вселенных.
\usection{4. Пришествия Михаилов}
\vs p021 4:1 Существует семь групп Сынов\hyp{}Творцов пришествия, и они классифицируются в зависимости от числа пришествий, которыми они одаряли создания своих миров. Их различают, начиная от тех, кто обладает начальным опытом, далее следуют прошедшие через пять дополнительных сфер последовательного продвижения по пути пришествия и, наконец, те, кто достиг седьмого и заключительного этапа опыта Творца в образе создания.
\vs p021 4:2 Пришествие Авоналов всегда осуществляется в обличье смертной плоти, но семь пришествий Сына\hyp{}Творца включают его появление на семи уровнях бытия созданий и имеют отношение к откровению семи первичных выражений воли и природы Божества. Все без исключения Сыны\hyp{}Творцы проходят через это семь раз, отдавая себя своим сотворенным детям, прежде чем они принимают на себя установленную и верховную власть над вселенной их собственного творения.
\vs p021 4:3 Хотя эти семь пришествий различны в разных секторах и вселенных, они всегда содержат событие --- пришествие в образе смертного. В этом заключительном пришествии Сын\hyp{}Творец появляется в каком\hyp{}то обитаемом мире как член одной из высших смертных рас, обычно --- как член той расовой группы, которой в максимальной степени присущи наследственные черты Адамова рода, которые были предварительно внесены, чтобы поднять физический статус обитателей животного происхождения. Только однажды в его семеричной деятельности в качестве Сына пришествия Райского Михаила рождает женщина, как, например, у вас есть запись о Вифлеемском дитяти. Только однажды он живет и умирает как член самого низшего чина эволюционирующих созданий, обладающих волей.
\vs p021 4:4 После каждого своего пришествия Сын\hyp{}Творец переходит «одесную Отца», чтобы там снискать одобрение этого пришествия и получить дальнейшие наставления для очередного этапа вселенского служения. После седьмого и заключительного пришествия Сын\hyp{}Творец получает от Отца Всего Сущего верховную власть и полномочия над своей вселенной.
\vs p021 4:5 \pc Известно, что в последнем появлении на вашей планете божественный Сын был Райским Сыном\hyp{}Творцом, который завершил шесть фаз своей деятельности пришествия; следовательно, когда он сознательно отказался от жизни во плоти на Урантии, он мог сказать и воистину сказал: «Свершилось!», --- его миссия действительно закончилась. Его смерть на Урантии завершила его пришествие; это был последний акт в осуществлении священной клятвы Райского Сына\hyp{}Творца. И по обретению этого опыта такие Сыны становятся верховными владыками вселенных; они более не правят как наместники Отца, но --- в силу своего собственного права и от своего собственного имени --- как «Царь Царей и Господь Бог». За некоторыми указанными исключениями, эти Сыны семеричного пришествия --- неограниченные верховные во вселенных своего пребывания. Что касается его локальной вселенной, то этому торжествующему и возведенному на престол Сыну\hyp{}Мастеру передается «вся власть на небесах и на земле».
\vs p021 4:6 \pc После завершения своего пришествия Сыны\hyp{}Творцы считаются самостоятельным чином --- семеричные Сыны\hyp{}Мастера. Личностно Сыны\hyp{}Мастера идентичны Сынам\hyp{}Творцам, но они испытали столь уникальный опыт пришествия, что их обычно относят к другому чину. Если Творец замысливает осуществить пришествие, то этому действительно и непременно суждено случиться. Правда, Сын пришествия все же и тем не менее является Творцом, но он присовокупляет к своей природе опыт создания, который навсегда уводит его с божественного уровня Сына\hyp{}Творца и возносит на достигнутую с опытом ступень Сына\hyp{}Мастера, единственного, кто полностью заработал право властвовать во вселенной и управлять ее мирами. Такие существа вбирают в себя все возможное, что получено от божественных родителей и приобретено из опыта сделанных совершенными созданий. Почему человек должен сокрушаться из\hyp{}за своего низкого происхождения и вынужденного эволюционного пути развития, если самим Богам необходимо пройти тем же путем прежде, чем они приобретут надлежащий опыт и могут считаться окончательно и абсолютно компетентными, чтобы править своими вселенскими владениями!
\usection{5. Связь Сынов\hyp{}Мастеров со вселенной}
\vs p021 5:1 Мощь Мастера\hyp{}Михаила безгранична, потому что проистекает из его проверенного на опыте союза с Райской Троицей, она неоспорима, потому что проистекает из его актуального опыта, воплощения в жизнь созданий, подчиняющихся такой власти. Природа владычества семеричного Сына\hyp{}Творца верховна, потому что она:
\vs p021 5:2 \ublistelem{1.}\bibnobreakspace Охватывает семеричную точку зрения Райского Божества.
\vs p021 5:3 \ublistelem{2.}\bibnobreakspace Включает в себя семеричную позицию созданий, живущих в пространстве и времени.
\vs p021 5:4 \ublistelem{3.}\bibnobreakspace В полной мере обобщает позиции Рая и точки зрения созданий.
\vs p021 5:5 \pc Это опытное владычество, таким образом, включает в себя все, что относится к божественности Бога Семеричного, достигающего свой кульминации в Верховном Существе. И личное владычество семеричного Сына подобно будущему владычеству предназначенного когда\hyp{}нибудь завершиться Верховного Существа, охватывающему полную суть мощи и власти Райской Троицы, которая способна проявляться внутри соответствующих пространственно\hyp{}временных пределов.
\vs p021 5:6 \pc С достижением верховного владычества в локальной вселенной от Михаила исходит мощь и возможность создавать совершенно новые виды сотворенных существ в течение текущего вселенского периода. Но потеря Сыном\hyp{}Мастером способности порождать всецело новые чины существ никоим образом не мешает совершенствованию жизни, уже созданной и находящейся в процессе развития; эта грандиозная программа вселенской эволюции движется вперед без перерывов и сокращений. Обретение верховного владычества Сыном\hyp{}Мастером предполагает ответственность личной преданности делу воспитания и управления тем, что уже было задумано и создано, и тем, что впоследствии будет порождено теми, кто уже, таким образом, был задуман и создан. В свое время, может быть, разовьется почти бесконечная эволюция разнообразных существ, но впредь никакого всецело нового паттерна или вида разумного создания не произойдет непосредственно от Сына\hyp{}Мастера. Это есть первый шаг --- начало --- установленного руководства каждой вселенной.
\vs p021 5:7 Возвышение Сына семеричного пришествия до неоспариваемого владычества его вселенной означает начало конца вековечной неопределенности и относительного беспорядка. Вслед за этим событием то, что не может быть когда\hyp{}нибудь одухотворено, будет, в конце концов, деструктурировано; то, что не может быть когда\hyp{}нибудь согласовано с космической реальностью, будет, в конце концов, уничтожено. Когда запасы бесконечного милосердия и невыразимого терпения будут исчерпаны в попытке завоевать верность и преданность созданий мира сего, обладающих волей, восторжествует справедливость и праведность. То, что не сможет исправить милосердие, справедливость, в конце концов, уничтожит.
\vs p021 5:8 \pc Мастера\hyp{}Михаилы являются верховными в своих собственных локальных вселенных, если однажды они были там поставлены как владыки, правители. Те немногие ограничения, которые наложены на их правление, присущи космическому предсуществованию некоторых сил и личностей. В остальных отношениях Сыны\hyp{}Мастера являются верховными во власти, ответственности и в управленческой мощи в своих соответствующих вселенных; они как Творцы и Боги являются верховными, фактически, во всем. Не существует ничего превосходящего их мудрый замысел функционирования данной вселенной.
\vs p021 5:9 Райский Михаил после своего возвышения до установленного владычества в локальной вселенной осуществляет полный контроль над остальными Сынами Бога, функционирующими в его сфере, и может свободно править в соответствии со своим представлением о нуждах своих владений. Сын\hyp{}Мастер может по своей воле изменять порядок духовных диспенсаций и вносить корректировки в эволюцию обитаемых планет. И такие Сыны по своему выбору составляют и осуществляют планы по всем вопросам, касающимся особых планетарных нужд, прежде всего миров, на которых они жили как создания, еще большую заботу они проявляют о сфере заключительного пришествия --- о планете воплощения в обличье смертной плоти.
\vs p021 5:10 Сыны\hyp{}Мастера, по\hyp{}видимому постоянно сообщаются с мирами их пришествия, не только с теми, где они проживали сами, но и со всеми мирами, которые Сын\hyp{}Повелитель одарил своим пришествием. Этот контакт поддерживается их собственным духовным присутствием, Духом Истины, который они способны «изливать на всякую плоть». Эти Сыны\hyp{}Мастера также поддерживают непрерывную связь с Вечным Сыном\hyp{}Матерью в центре всех вещей. Они обладают полной сочувствия сферой влияния, которая простирается от Отца Всего Сущего в высях до низких рас, ведущих планетарную жизнь в областях времени.
\usection{6. Предназначение Мастеров\hyp{}Михаилов}
\vs p021 6:1 Никто не осмелится безапелляционно рассуждать ни о природе, ни о предназначении семеричных Мастеров\hyp{}Владык локальных вселенных; тем не менее мы много размышляем об этом. Нас учили, и мы верим, что каждый Райский Михаил есть \bibemph{абсолют} двуединых божественных понятий его происхождения; таким образом, он воплощает в себе актуальные аспекты бесконечности Отца Всего Сущего и Вечного Сына. Михаилы, должно быть, частичны по отношению к тотальной бесконечности, но они, вероятно, абсолютны по отношению к той части бесконечности, которая касается их происхождения. Но наблюдая их работу в современный вселенский период, мы не обнаруживаем никаких действий, которые явились бы большими, чем конечные; любые гипотетические сверхконечные способности, очевидно, скрыты в них самих и пока не раскрыты.
\vs p021 6:2 Завершение деятельности пришествия в образе созданий и возвышение до верховного вселенского владычества должно означать полное высвобождение способностей Михаила к конечным действиям, сопровождаемое появлением способностей к более\hyp{}чем\hyp{}конечному служению. И в этой связи мы отмечаем, что такие Сыны\hyp{}Мастера впоследствии ограничены в создании новых видов тварных существ, что, несомненно, явилось необходимым в результате освобождения их сверхконечных потенциалов.
\vs p021 6:3 Наиболее вероятно, что эти нераскрытые способности творцов останутся сокрытыми в них самих в течение всего настоящего вселенского периода. Но когда\hyp{}нибудь в очень отдаленном будущем, в пришедших теперь в движение вселенных внешнего пространства, мы полагаем, что связь между семеричным Сыном\hyp{}Мастером и Творческим Духом седьмой стадии, может быть, достигнет абсонитных уровней служения, что будет сопровождаться появлением новых вещей, значений и ценностей трансцендентальных уровней предельной вселенской значимости.
\vs p021 6:4 Как Божество Верховного актуализируется благодаря опытному служению, точно так же и Сыны\hyp{}Творцы достигают личностной реализации потенциалов Райской божественности, присущих их непостижимой природе. В свою бытность на Урантии Христос\hyp{}Михаил сказал однажды: «Я есть путь, и истина, и жизнь». И мы полагаем, что в вечности Михаилы действительно предназначены быть «путем, истиной и жизнью», вечно сияющим путем для всех вселенских личностей, поскольку он ведет от верховной божественности --- через предельную абсонитность --- к вечной божественной финальности.
\vsetoff
\vs p021 6:5 [Представлено Совершенствователем Мудрости из Уверсы.]
